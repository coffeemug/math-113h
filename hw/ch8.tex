\section{Least upper bounds}
\subsection*{Problem 3a}
Let $f$ be a continous function on $[a, b]$ with $f(a)<0<f(b)$. The
proof of Theorem 1 showed that there is a smallest $x$ in $[a,b]$ with
$f(x)=0$. Is there necessarily a second smallest $x$ in $[a,b]$ with
$f(x)=0$? Show that there is a largest $x$ in $[a,b]$ with $f(x)=0$.
(Try to give an easy proof by considering a new function closely
related to $f$.)

\subsection*{Solution}
There isn't necessarily a second smallest $x$ in $[a,b]$ with
$f(x)=0$. Consider $f(x)=x$ on $[-1,1]$. The smallest and only $x$
where $f(x)=0$ is $0$.

\vs

We want to find a function $g$ that varies from $b$ to $a$ the way $f$
varies from $a$ to $b$. In particular, we want:
\begin{itemize}
\item g(a)=f(b)
\item g(b)=f(a)
\end{itemize}

Observe that $g(x)=f(a+b-x)$ is such a function. By intermediate value
theorem there is a smallest $x$ in $[a,b]$ with $g(x)=0$. But that's
the largest $x$ in $[a,b]$ with $f(x)=0$.

\subsection*{Problem 5a}
Suppose that $y-x>1$. Prove that there is an integer $k$ such that
$x<k<y$. Hint: let $l$ be the largest integer satisfying $l\leq x$, and
consider $l+1$.

\subsection*{Solution}
\textbf{Lemma:} there exists the largest integer $l$ such that $l\leq x$.

\textbf{Proof:} since $\mathcal{N}$ is unbounded, there exists
$n\in\mathcal{N}$ such that $-n<x<n$. There is a finite number of integers between
$-n$ and $x$. Pick the largest.

\vs

Let $l$ be the largest integer such that $l\leq x$. Then
\begin{itemize}
\item $l+1>x$, otherwise $l$ wouldn't be the \textit{largest} integer
  such that $l\leq x$.
\item $l+1\leq x+1$. But $y>x+1$, thus $l+1<y$.
\end{itemize}

Therefore $x<l+1<y$ as desired.

\subsection*{Problem 5b}
Suppose $x<y$. Prove that there is a rational number $r$ such that
$x<r<y$. Hint: if $1/n<y-x$, then $ny-nx>1$. (Query: Why have parts
(a) and (b) been postponed until this problem set?)

\subsection*{Solution}
In 5a we've proven there exists an integer $l$ such that $nx<l<ny$.
Dividing each side by $n$ we get $x<\frac{l}{n}<y$, as desired.

\vs

For 5a we need the unboundedness of $\mathcal{N}$ and in 5b we need the
Archimedean property for the existence of $n$ such that $1/n<y-x$.
Both depend on least upper bound.

\subsection*{Problem 5c}
Suppose $r<s$ are rational numbers. Prove that there is an irrational
number between $r$ and $s$. Hint: As a start, you know that there is
an irrational number between $0$ and $1$.

\subsection*{Solution}
Let $x=r+\frac{\sqrt{2}(s-r)}{2}$. Obviously $x$ is irrational and
$x>r$. Also $r+(s-r)/2<s$, and multiplying $(s-r)/2$ by $0<\sqrt{2}<1$
will only make it smaller. Thus $x<s$.

\subsection*{Problem 5d}
Suppose that $x<y$. Prove that there is an irrational number between
$x$ and $y$. Hint: It is unnecessary to do any more work; this follows
from (b) and (c).

\subsection*{Solution}
By 5b there exists a rational $x<r<y$, and by 5b again there exists a
rational $r<s<y$. Then by 5c there exists an irrational $m$ such that
$r<m<s$.

\subsection*{Problems 6a, b}
A set $A$ of real numbers is said to be \textbf{dense} if every open
interval contains a point of $A$. For example, Problem 5 shows that
the set of rational numbers and the set of irrational numbers are each
dense.

\subsection*{Solution}

(a) Prove that if $f$ is continuous and $f(x)=0$ for all numbers $x$
in a dense set $A$, then $f(x)=0$ for all $x$.

\vs

Suppose for contradiction there exists $a\in\mathcal{R}$ such that
$f(a)\neq0$. Fix $\epsilon=|f(a)|/2$. By continuity definition, there exists
$\delta>0$ such that $|x-a|<\delta$ implies $|f(x)-f(a)|<\epsilon$. Since
$A$ is dense, there exists $x\in(a-\delta, a+\delta)$ such that
$x\in A$, and thus $f(x)=0$. But that would imply
$|0-f(a)|=|f(a)|<|f(a)|/2$, which is a contradiction. Thus $f(x)=0$
for all $x$ as desired.

\vs

(b) Prove that if $f$ and $g$ are continous and $f(x)=g(x)$ for all
$x$ in a dense set $A$, then $f(x)=g(x)$ for all $x$.

\vs

Since $f,g$ are continuous, so is $f-g$. We can thus apply 6a to
$f-g$, and we are done.

\subsection*{Problem 11a}
Suppose that $a_{1}, a_{2}, a_{3}, \ldots$ is a sequence of positive
numbers with $a_{n+1}\leq a_{n}/2$. Prove that for any $\epsilon>0$ there is
some $n$ with $a_{n}<\epsilon$.

\subsection*{Solution}


%%% Local Variables:
%%% TeX-master: "hw"
%%% End:
