\section{Three Hard Theorems}

\subsection*{Problem 1}
Decide which functions are bounded above or below on the given
interval, and which take on their min and max values. (Note, $f$ may
have these properties even if it isn't continuous or the interval
isn't closed.)

\subsection*{Solution}

(i) $f(x)=x^{2}$ on $(-1,1)$

Bounded above and below, and takes on min value. Note $f$ doesn't take
on the maximum value in $(-1,1)$. I.e. there is no $y$ in $(-1,1)$
such that $f(y)\geq f(x)$ for all $x$ in $(-1,1)$.

\vs

(ii) $f(x)=x^{3}$ on $(-1,1)$

Bounded above and below.

\vs

(iii) $f(x)=x^{2}$ on $\R$

Bounded below and takes on min value.

\vs

(iv) $f(x)=x^{2}$ on $[0, \infty)$

Bounded below and takes on min value.

\vs

(v) $f(x)=\begin{cases}
  x^{2}, x\leq a\\
  a+2, x>a
\end{cases}$ on $(-a-1, a+1)$.

First, we want $a+1>-a-1$, thus $a>-1$.

\vs

(vi) $f(x)=\begin{cases}
  x^{2}, x<a\\
  a+2, x\geq a
\end{cases}$ on $[-a-1, a+1]$.

\vs

(vii) $f(x)=\begin{cases}
  0, \text{$x$ is irrational}\\
  1/q \ \ \ x=p/q \text{ in lowest terms}
\end{cases}$ on $[0, 1]$.

\vs

(viii) $f(x)=\begin{cases}
  1, \text{$x$ is irrational}\\
  1/q \ \ \ x=p/q \text{ in lowest terms}
\end{cases}$ on $[0, 1]$.

\vs

(ix) $f(x)=\begin{cases}
  1, \text{$x$ is irrational}\\
  -1/q \ \ \ x=p/q \text{ in lowest terms}
\end{cases}$ on $[0, 1]$.

\vs

(x) $f(x)=\begin{cases}
  x, \text{$x$ is rational}\\
  0 \ \ \ \text{$x$ is irrational}
\end{cases}$ on $[0, a]$.


\subsection*{Problem 3}
Prove that there is some number $x$ such that

\subsection*{Solution}
(i) $x^{179}+\frac{163}{1+x^{2}+\sin^{2} x}=119$

\vs

Consider $f(x)=x^{179}+\frac{163}{1+x^{2}+\sin^{2} x}-119$. Observe
that $f(-1)<0$ and $f(0)>0$. Thus by intermediate value theorem there
exists $y$ in $[-1,0]$ such that $f(y)=0$.

\vs

(ii) $\sin x = x-1$

\vs

Consider $f(x)=\sin x-x+1$. Observe that $f(-5)>0$ and $f(5)<0$. Thus
by intermediate value theorem there exists $y$ in $[-5,5]$ such that
$f(y)=0$.

\subsection*{Problem 5}
Suppose that $f$ is continuous on $[a,b]$ and that $f(x)$ is always
rational. What can be said about $f$?

\subsection*{Solution}
$f(x)=c, c\in\mathcal{Q}$. For suppose this isn't the case. Then there exist
$x,y$ such that $f(x)=a$ and $f(y)=b$, and $a\neq b$. Let $d$ be
irrational and $a<d<b$. By intermediate value theorem there exists $z$
in $[x, y]$ such that $f(z)=d$. But $f$ is always rational, so we have
a contradiction.

\subsection*{Problem 17}
Suppose that $f$ is a continuous function with $f(x)>0$ for all $x$,
and $\lim_{x\to\infty} f(x)=0=\lim_{x\to-\infty} f(x)$. (Draw a picture.) Prove that
there is some number $y$ such that $f(y)\geq f(x)$ for all $x$.

\subsection*{Solution}
Let $a=f(0)$. By limit definition:
\begin{itemize}
\item There exists $N_1<0$ such that $f(x)<a$ for all $x$ in $(-\infty, N_1)$.
\item There exists $N_2>0$ such that $f(x)<a$ for all $x$ in $(N_2, \infty)$.
\end{itemize}

By extreme value theorem there is some $y$ in $[N_1, N_2]$ such that
$f(y)\geq f(x)$ for all $x$ in $[N_1, N_2]$. Observe that $f(y)\geq a$,
otherwise $f(y)$ wouldn't be a maximum. Thus $f(y)\ge f(x)$ for all
$x\in\mathcal{R}$, as desired.


%%% Local Variables:
%%% TeX-master: "hw"
%%% End:
