\section{Numbers of various sorts}
\subsection*{Problem 1i}
Prove the following formula by induction
\[1^2+\ldots+n^2=\frac{n(n+1)(2n+1)}{6}\]

\subsubsection*{Solution}
Observe that the base case $n=1$ holds:
\[1^2=\frac{1(1+1)(2\cdot1+1)}{6}=1\]

Suppose the equation holds for some integer $k$. Adding $(k+1)^2$ to both sides we get:
\begin{align*}
    1^2+\ldots+k^2+(k+1)^2&=\frac{k(k+1)(2k+1)}{6}+(k+1)^2\\
    &=\frac{(k^2+k)(2k+1)}{6}+k^2+2k+1\\
    &=\frac{2k^3+3k^2+k}{6}+\frac{6k^2+12k+6}{6}\\
    &=\frac{2k^3+9k^2+13k+6}{6}\\
\end{align*}

Consider the case $k+1$:
\begin{align*}
    1^2+\ldots+k^2+(k+1)^2&=\frac{(k+1)((k+1)+1)(2(k+1)+1)}{6}\\
    &=\frac{(k+1)(k+2)(2k+3)}{6}\\
    &=\frac{(k^2+3k+2)(2k+3)}{6}\\
    &=\frac{2k^3+3k^2+6k^2+9k+4k+6}{6}\\
    &=\frac{2k^3+9k^2+13k+6}{6}\\
\end{align*}

Thus the formula holds for any positive integer $n$ as desired.

\subsection*{Problem 1ii}
Prove the following formula by induction
\[1^3+\ldots+n^3=(1+\ldots+n)^2\]

\subsubsection*{Solution}
Observe that the base case $n=1$ holds:
\[1^3=1^2\]

Suppose the equation holds for some integer $k$. Recall that $1+\ldots+k=\frac{k(k+1)}{2}$. Adding $(k+1)^3$ to both sides we get:
\begin{align*}
    1^3+\ldots+k^3+(k+1)^3&=\left(\frac{k(k+1)}{2}\right)^2+(k+1)^3\\
    &=\frac{(k^2+k)^2}{4}+k^3+3k^2+3k+1\\
    &=\frac{k^4+2k^3+k^2}{4}+\frac{4k^3+12k^2+12k+4}{4}\\
    &=\frac{k^4+6k^3+13k^2+12k+4}{4}
\end{align*}

Now consider the case $k+1$:
\begin{align*}
    1^3+\ldots+k^3+(k+1)^3&=\left(\frac{(k+1)(k+2)}{2}\right)^2\\
    &=\frac{(k^2+3k+2)^2}{4}\\
    &=\frac{k^4+3k^3+2k^2+3k^3+9k^2+6k+2k^2+6k+4}{4}\\
    &=\frac{k^4+6k^3+13k^2+12k+4}{4}\\
\end{align*}

Thus the formula holds for any positive integer $n$ as desired.

\subsection*{Problem 5}
(a) Prove by induction on $n$ that
\[1+r+r^2+\ldots+r^n=\frac{1-r^{n+1}}{1-r}\]
if $r\neq 1$.

Observe that the base case $n=1$ holds:
\[1+r=\frac{1-r^2}{1-r}=\frac{(1-r)(1+r)}{1-r}=1+r\]

Suppose the equation holds for some integer $k$. Adding $r^{k+1}$ to both sides we get:
\begin{align*}
    1+r+r^2+\ldots+r^k+r^{k+1}&=\frac{1-r^{k+1}}{1-r}+r^{k+1}\\
    &=\frac{1-r^{k+1}+(1-r)r^{k+1}}{1-r}\\
    &=\frac{1-r^{k+1}+r^{k+1}-r^{k+2}}{1-r}\\
    &=\frac{1-r^{k+2}}{1-r}
\end{align*}

Thus the equation holds as desired.

\vs

(b) Derive this result by setting $S=1+r+\ldots+r^n$, multiplying this equation by $r$, and solving the two equations for $S$.

\begin{align*}
&rS=r+r^2+\ldots+r^{n+1}\\
&\implies S-rS=(1+r+\ldots+r^n)-(r+r^2+\ldots+r^{n+1})\\
&\implies S(1-r)=1-r^{n+1}\\
&\implies S=\frac{1-r^{n+1}}{1-r}
\end{align*}

\subsubsection*{Solution}

\subsection*{Problem 10}
Prove the principle of mathematical induction from the well-ordering principle.

\subsubsection*{Solution}
Let $P$ be a property indexed on natural numbers such that: $P_1$ is true, and $P_{k+1}$ is true if $P_k$ is true for $k\in\N$.
We must show $P_n$ is true for all $n\in\N$.

\vs

Let $S$ be a set of natural numbers for which $P$ is false, i.e. $S=\{j:P_j\text{ is false}\}$. Suppose for contradiction $S$ is not empty. Then by well ordering principle there exists $m\in S$ such that $m$ is the smallest element in $S$.

\vs

Since $m\in S$, $P_m$ is false. Further, since $m$ is the smallest element of $S$, $P_{m-1}$ is true (if it weren't, $m-1$ would be in $S$ and $m$ wouldn't be the smallest element). But $P_{m-1}$ being true implies by induction $P_m$ is true. We have a contradiction. Therefore $S$ is empty, and $P_n$ is true for all $n\in\N$ as desired.

\subsection*{Problem 10-modified}
Prove the well-ordering principle from the principle of mathematical induction.

[\textit{Adding for my own understanding.}]

\subsubsection*{Solution}
Let $S$ be a set of natural numbers with no least element. We show that $S$ must be empty (and thus every non-empty set of natural numbers has a least element).

\vs

Let $B$ be the set of natural numbers $n$ such that $1\ldots n\not\in S$. We will show by induction that all $n\in\N$ are in $B$ (and thus $S$ is empty).

\vs

First, $1\in B$ (if $1$ were in $S$, then $S$ would have $1$ as its smallest member). Thus the base case holds.

\vs

Suppose $k\in B$ (and thus $1\ldots k\not\in S$). Consider $k+1$. Since $1\ldots k\not\in S$, then $k+1\not\in S$ (otherwise $k+1$ would be the smallest member of $S$). Therefore $k+1\in B$. Thus by induction $n\in B$ for all $n\in N$, and $S$ is empty as desired.

\subsection*{Problem 14a}
Prove that $\sqrt{2}+\sqrt{6}$ is irrational.

\subsubsection*{Solution}
Suppose $\sqrt{2}+\sqrt{6}$ is rational. Then its square $8+4\sqrt{3}$ must be rational as well. Recall that $\mathcal{Q}$ is closed under addition, thus $8+4\sqrt{3}+(-8)=4\sqrt{3}$ is rational. Similarly, $\mathcal{Q}$ is closed under multiplication, thus $4\sqrt{3}\cdot\frac{1}{4}=\sqrt{3}$ is rational. We have a contradiction, and thus $\sqrt{2}+\sqrt{6}$ is irrational as desired.

%%% Local Variables:
%%% TeX-master: "hw"
%%% End:
