\section{Basic Properties of Numbers}

\subsection*{Problem 1v}
Prove $x^n-y^n=(x-y)(x^{n-1}+x^{n-2}y+\ldots+xy^{n-2}+y^{n-1})$

\subsubsection*{Solution}
\begin{align*}
    &(x-y)(x^{n-1}+x^{n-2}y+\ldots+xy^{n-2}+y^{n-1})\\
    &=x(x^{n-1}+x^{n-2}y+\ldots+xy^{n-2}+y^{n-1})-y(x^{n-1}+x^{n-2}y+\ldots+xy^{n-2}+y^{n-1})\\
    &=(x^n+x^{n-1}y+\ldots+x^2y^{m-2}+xy^{n-1})-(x^{n-1}y+x^{n-2}y^2+\ldots+xy^{n-1}+y^n)\\
    &=x^n+(x^{n-1}y+\ldots+x^2y^{m-2}+xy^{n-1}-x^{n-1}y+x^{n-2}y^2+\ldots+xy^{n-1})+y^n\\
    &=x^n-y^n
\end{align*}

\subsection*{Problem 1vi}
Prove $x^3+y^3=(x+y)(x^2-xy+y^2)$

\subsubsection*{Solution}
\begin{align*}
    (x+y)(x^2-xy+y^2)&=x(x^2-xy+y^2)+y(x^2-xy+y^2)\\
    &=(x^3-x^2y+xy^2)+(x^2y-xy^2+y^3)\\
    &=x^3+(-x^2y+xy^2+x^2y-xy^2)+y^3\\
    &=x^3+y^3
\end{align*}

\subsection*{Problem 5iii}
Prove that if $a<b$ and $c>d$ then $a-c<b-d$.

\subsubsection*{Solution}
Observe that $a<b\implies b-a>0$ and $c>d\implies c-d>0$. Then
\begin{align*}
    (b-d)-(a-c)&=b-d-a+c\\
    &=(b-a)+(c-d)>0
\end{align*}
Thus $a-c<b-d$ as desired.

\subsection*{Problem 5vii}
Prove that if $0<a<1$ then $a^2<a$.

\subsubsection*{Solution}
Observe that $a-a^2=a(1-a)$. Further $a>0$ and $1-a>0$, thus $a(1-a)>0$. Therefore $a-a^2>0$ and thus $a^2<a$ as desired.

\subsection*{Problem 5viii}
Prove that if $0\leq a<b$ and $0\leq c<d$, then $ac<bd$.

\subsubsection*{Solution}
Suppose $c=0$. Then $ac=0$. Since $b>0$ and $d>0$, $bd>0$, and thus $ac<bd$.

\vs

Alternatively, suppose $c>0$. Since $a<b$  and $c>0$, it follows $ac<bc$. Similarly since $c<d$ and $b>0$ it follows $bc<bd$. So, $ac<bc, bc<bd$, therefore $ac<bd$ as desired.

\subsection*{Problem 11}
Find all numbers $x$ for which

\subsubsection*{Solution}
(i) $|x-3|=8$

\begin{itemize}
    \item $x-3=8\implies x=11$
    \item $x-3=-8\implies x=-5$    
\end{itemize}

\vs

(ii) $|x-3|<8$

\begin{align*}
    -8<x-3<8\\
    -5<x<11
\end{align*}

\vs

(iii) $|x+4|<2$

\begin{align*}
    -2<x+4<2\\
    -6<x<-2
\end{align*}

\vs

(iv) $|x-1|+|x-2|>1$

Checking the intervals $(-\infty, 1), (1, 2), (2, \infty)$, the two intervals that work are $x<1$ and $x>2$.

\vs

(v) $|x-1|+|x+1|<2$

Checking the intervals $(-\infty, 1), (-1, 1), (1, \infty)$, there is no solution that satisfies the equation.

\vs

(vi) $|x-1|+|x+1|<1$

Since there is no solution for (v), there is certainly no solution for (vi) as it's a stricter inequality.

\vs

(vii) $|x-1|\cdot|x+1|=0$

$x=1, x=-1$

\vs

(viii) $|x-1|\cdot|x+2|=3$

There are three cases:
\begin{itemize}
    \item $x<-2$: The equation becomes $(1-x)(-x-2)=3$.
    \item $-2<x<1$: The equation becomes $(1-x)(x+2)=3$
    \item $x>1$ The equation becomes $(x-1)(x+2)=3$.
\end{itemize}

The first and last equations simplify to $x^2+x-5=0$ and have two obvious roots (based on quadratic formula). The middle equation has no solutions. Plotting confirms this.

\subsection*{Problem 13}
Prove that
\[\max(x,y)=\frac{x+y+|y-x|}{2}\]
\[\min(x,y)=\frac{x+y-|y-x|}{2}\]

Derive a formula for $\max(x,y,z)$ and $\min(x,y,z)$.

\subsubsection*{Solution}
First we prove $\max(x,y)=\frac{x+y+|y-x|}{2}$. Suppose $x>y$, i.e. $\max(x,y)=x$. Then $|y-x|=x-y$. Therefore
\[\frac{x+y+|y-x|}{2}=\frac{x+y+x-y}{2}=x\]
as desired. Conversely suppose $x<y$, i.e. $\max(x,y)=y$. Then $|y-x|=y-x$. Therefore
\[\frac{x+y+|y-x|}{2}=\frac{x+y+y-x}{2}=y\]
as desired.

\vs

Similarly we prove $\min(x,y)=\frac{x+y-|y-x|}{2}$. Suppose $x>y$, i.e. $\min(x,y)=y$. Then $|y-x|=x-y$. Therefore
\[\frac{x+y-|y-x|}{2}=\frac{x+y-x+y}{2}=y\]
as desired. Conversely suppose $x<y$, i.e. $\min(x,y)=x$. Then $|y-x|=y-x$. Therefore
\[\frac{x+y-|y-x|}{2}=\frac{x+y-y+x}{2}=x\]
as desired.

\vs

It's easy to see that if $x=y$, both formulas compute the correct result
\[\min(x,y)=\max(x,y)=x=y\]

\vs

We now derive $\max(x,y,z)$:
\begin{align*}
    \max(x,y,z)&=\max(x, \max(y, z))\\
    &=\frac{x+\max(y,z)+|\max(y,z)-x|}{2}\\
    &=\frac{x+\frac{y+z+|z-y|}{2}+|\frac{y+z+|z-y|}{2}-x|}{2}\\
    &=\frac{\frac{2x+y+z+|z-y|}{2}+|\frac{y+z+|z-y|-2x}{2}|}{2}\\
    &=\frac{2x+y+z+|z-y|+|y+z+|z-y|-2x|}{4}\\
\end{align*}

Similarly
\begin{align*}
    \min(x,y,z)&=\min(x,\min(y,z))\\
    &=\frac{x+\frac{y+z-|z-y|}{2}-|\frac{y+z-|z-y|}{2}-x|}{2}\\
    &=\frac{\frac{2x+y+z-|z-y|}{2}-|\frac{y+z-|z-y|-2x}{2}|}{2}\\
    &=\frac{2x+y+z-|z-y|-|y+z-|z-y|-2x|}{4}
\end{align*}

%%% Local Variables:
%%% TeX-master: "hw"
%%% End:
