\section{Derivatives}

\subsection*{Problem 2}
(a) Prove that if $f(x)=1/x^2$, then $f'(a)=-2/a^3$ for $a\neq0$.

\begin{align*}
  f'(a)&=\lim_{h\to0}\frac{f(a+h)-f(a)}{h}\\
       &=\lim_{h\to0}\frac{\frac{1}{(a+h)^2}-\frac{1}{a^2}}{h}\\
       &=\lim_{h\to0}\frac{\frac{1}{a^2+2ah+h^2}-\frac{1}{a^2}}{h}\\
       &=\lim_{h\to0}\frac{h(-2a-h)}{a^2h(a^2+2ah+h^2)}\\
       &=\lim_{h\to0}\frac{-2a-h}{a^2(a^2+2ah+h^2)}\\
       &=\frac{-2a}{a^4}=\frac{-2}{a^3}
\end{align*}

(b) Prove that the tangent line to $f$ at $(a, 1/a^2)$ intersects $f$
at one other point, which lies on the opposide side of the vertical
axis.

\vs

The linear equation of a line tangent to $f$ at $a$ is
\[g(x)=(-2/a^3)(x-a)+1/a^2\]

Solving for $x$ such that $f(x)=g(x)$:
\begin{align*}
  &1/x^2=(-2/a^3)(x-a)+1/a^2\\
  &\implies 1/x^2=-2x/a^3+2a/a^3+a/a^3\\
  &\implies 1/x^2=\frac{-2x+3a}{a^3}\\
  &\implies x^2=\frac{a^3}{3a-2x}\\
  &\implies 2x^3-3ax^2+a^3=0\\
  &\implies (x-a)(2x^2-ax-a^2)&\text{we know $a$ is a root}\\
  &\implies (x-a)^2(2x+a)
\end{align*}

Thus $(-\frac{a}{2}, \frac{4}{a^2})$ is another solution, and
$-\frac{a}{2}$ is on the opposide side of the vertical axis from $a$.

\subsection*{Problem 3}
Prove that if $f(x)=\sqrt{x}$, then $f'(a)=1/(2\sqrt{a})$, for $a>0$.
(The expression you obtain for $[f(a+h)-f(a)]/h$ will require some
algebraic face lifting, but the answer should suggest the right
trick.)

\subsection*{Solution}
\begin{align*}
  f'(a)&=\lim_{h\to0}\frac{f(a+h)-f(a)}{h}\\
       &=\lim_{h\to0}\frac{\sqrt{a+h}-\sqrt{a}}{h}\\
       &=\lim_{h\to0}\frac{(\sqrt{a+h}-\sqrt{a})(\sqrt{a+h}+\sqrt{a})}{h(\sqrt{a+h}+\sqrt{a})}\\
       &=\lim_{h\to0}\frac{h}{h(\sqrt{a+h}+\sqrt{a})}\\
       &=\lim_{h\to0}\frac{1}{\sqrt{a+h}+\sqrt{a}}=\frac{1}{2\sqrt{a}}
\end{align*}

\subsection*{Problem 11}
(a) Prove that Galileo was wrong: if a body falls a distance $s(t)$ in
$t$ seconds, and $s'$ is proportional to $s$, then $s$ cannot be a
function of the form $s(t)=ct^2$.

\begin{align*}
  s'(a)&=\lim_{h\to0}\frac{s(a+h)-s(a)}{h}\\
       &=\lim_{h\to0}\frac{c(a+h)^2-ca^2}{h}\\
       &=\lim_{h\to0}\frac{ca^2+2ach+ch^2-ca^2}{h}\\
       &=\lim_{h\to0}2ac+ch=2ac
\end{align*}
i.e. $s'$ is clearly not proportional to $s$, as $s'$ is not a factor
of $s$.

\vs

(b) Prove that the following facts are true about $s$ if
$s(t)=(a/2)t^2$ (the first fact will show why we switched from $c$ to
$a/2$).

\vs

\textbf{1.} $s''(t)=a$ (the acceleration is constant).

\vs

We've already seen that $s'(t)=2(a/2)t=at$. Then
\[s''(t)=\lim_{h\to0}\frac{s'(t+h)-s'(t)}{h}=\lim_{h\to0}\frac{at+ah-at}{h}=a\]

\vs

\textbf{2.} $[s'(t)]^2=2as(t)$.

\vs

First,
\[[s'(t)]^2=(at)^2=a^2t^2\]

Second,
\[2as(t)=2a(a/2)t^2=a^2t^2\]

Thus $[s'(t)]^2=2as(t)$ as desired.

\vs

(c) If $s$ is measured in feet, the value of $a$ is 32. How many
seconds do you have to get out of the way of a chandelier which falls
from a 400-foot ceiling? If you don't make it, how fast will the
chandelier be going when it hits you? Where was the chandelier when it
was moving with half that speed?

\begin{enumerate}
\item Solving for $s(t)=16t^2=400$, we get $t=5$ seconds before the
  chandelier hits the floor.
\item Velocity at $5$ seconds is $s'(5)=at=32*5=160$ feet per second.
\item We can solve for $t$ when speed is $80$ feet per second:
  $s'(t)=at=32*t=80$, and so $t=2.5$ seconds. Plugging that back in we
  get $s(2.5)=16*(2.5)^2=100$, and so the chandalier is at
  $400-100=300$ feet.
\end{enumerate}

\subsection*{Problem 20}
Let $f$ be any polynomial function; we will see in the next chapter
that $f$ is differentiable. The tangent line to $f$ at $(a, f(a))$ is
the graph of $g(x)=f'(a)(x-a)+f(a)$. Thus $f(x)-g(x)$ is the
polynomial function $d(x)=f(x)-f'(a)(x-a)-f(a)$. We have already seen
that if $f(x)=x^2$, then $d(x)=(x-a)^2$, and if $f(x)=x^3$, then
$d(x)=(x-a)^2(x+2a)$.

\vs

\textbf{(a)} Find $d(x)$ when $f(x)=x^{4}$, and show that it is
divisible by $(x-a)^2$.

\vs

We've seen that $f'(x)=4x^3$. Thus
\[d(x)=x^4-4x^3(x-a)-a^4=-3x^4+4ax^3-a^4\]

Indeed,
\[(x-a)^2(-3x^2-2ax-a^2)=-3x^4+4ax^3-a^4\]

\textbf{(b)} There certainly seems to be some evidence that $d(x)$ is
always divisible by $(x-a)^2$. Figure 22 provides an intuitive
argument: usually, lines parallel to the tangent line will intersect
the graph at two points; the tangelnt line intersects the graph only
once near the point, so the intersection should be a ``double
intersection''. To give a rigorous proof, first note that
\[\frac{d(x)}{x-a}=\frac{f(x)-f(a)}{x-a}-f'(a)\]

Now answer the following questions.

\vs

\textbf{1.} Why is $f(x)-f(a)$ divisible by $(x-a)$?

This is the slope of $f$ near $a$, and is equivalent to the quotient
of the derivative. Since $f$ is differentiable at $a$ the limit must
exist, and thus $\frac{f(x)-f(a)}{x-a}$ must be defined.

\vs

\textbf{2.} Why is there a polynomial function $h$ such that
$h(x)=d(x)/(x-a)$ for $x\neq a$?

Because for all $x$ such that $d(x)=0$ we have the tangent line $g$
intersect $f$ at $(x, f(x))$. Since we know $(a, f(a))$ is a point of
intersection, $d(a)=0$, and thus $x-a$ is a factor.

\vs

\textbf{3.} Why is $\lim_{x\to a}h(x)=0$? Why is $h(a)=0$?

\begin{align*}
  \lim_{x\to a}h(x)&=\lim_{x\to a}\left(\frac{f(x)-f(a)}{x-a}-f'(a)\right)\\
                 &=\lim_{x\to
                   a}\frac{f(x)-f(a)}{x-a}-\lim_{x\to
                   a}f'(a)\\
                 &=\lim_{x\to a}\frac{f(x)-f(a)}{x-a}-f'(a)\\
                 &=f'(a)-f'(a)=0
\end{align*}

Since $h$ is continuous, $h(a)=\lim_{x\to a}h(x)=0$.

\vs

\textbf{4.} Why does this solve the problem?

Since $h(a)=0$, it follows $x-a$ is another factor. Thus $d(x)$ is
divisible by $x-a$ twice.

\subsection*{Problem 26}
Find $f''(x)$ if

\vs

(i) $f(x)=x^3$

We already know $f'(x)=2x^2$. Thus
\begin{align*}
  f''(x)&=\lim_{h\to0}\frac{2(x+h)^2-2x^2}{h}\\
        &=\lim_{h\to0}\frac{2x^2+4xh+2h^2-2x^2}{h}\\
        &=4x
\end{align*}

\vs

(ii) $f(x)=x^5$
\begin{align*}
  f'(x)&=\lim_{h\to0}\frac{(x+a)^5-x^5}{h}\\
       &=\lim_{h\to0}\frac{h^5 + 5 h^4 x + 10 h^3 x^2 + 10 h^2 x^3 + 5
  h x^4}{h}\\
       &=5x^4
\end{align*}

We now find $f''(x)$:
\begin{align*}
  f''(x)&=\lim_{h\to0}\frac{5(x+h)^4-5x^4}{h}\\
        &=\lim_{h\to0}\frac{5 h^4 + 20 h^3 x + 30 h^2 x^2 + 20 h
          x^3}{h}\\
        &=20x^3
\end{align*}

\vs

(iii) $f'(x)=x^4$

\begin{align*}
  f''(x)&=\lim_{h\to0}\frac{(x+h)^4-x^4}{h}\\
        &=\lim_{h\to0}\frac{h^4+4h^3x+6h^2x^2+4hx^3+x^4-x^4}{h}\\
        &=\lim_{h\to0}h^3+4h^2x+6hx^2+4x^3\\
        &=4x^3
\end{align*}

\vs

(iv) $f(x+3)=x^5$
\begin{align*}
  f'(x)&=\lim_{h\to0}\frac{(x+h-3)^5-(x-3)^5}{h}\\
       &=5(x-3)^4&\text{Ok, I used Mathematica}\\
\end{align*}

Then
\begin{align*}
  f''(x)&=\lim_{h\to0}\frac{5((x+h)-3)^4-5(x-3)^4}{h}\\
        &=20(x-3)^3&\text{Ok, I used Mathematica again}
\end{align*}


%%% Local Variables:
%%% TeX-master: "hw"
%%% End:
