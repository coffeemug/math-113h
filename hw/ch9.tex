\section{Derivatives}

\subsection*{Problem 2}
(a) Prove that if $f(x)=1/x^2$, then $f'(a)=-2/a^3$ for $a\neq0$.

(b) Prove that the tangent line to $f$ at $(a, 1/a^2)$ intersects $f$
at one other point, which lies on the opposide side of the vertical
axis.

\subsection*{Solution}

\subsection*{Problem 3}
Prove that if $f(x)=\sqrt{x}$, then $f'(a)=1/(2\sqrt{a})$, for $a>0$.
(The expression you obtain for $[f(a+h)-f(a)]/h$ will require some
algebraic face lifting, but the answer should suggest the right
trick.)

\subsection*{Solution}

\subsection*{Problem 11}
(a) Prove that Galileo was wrong: if a body falls a distance $s(t)$ in
$t$ seconds, and $s'$ is proportional to $s$, then $s$ cannot be a
function of the form $s(t)=ct^2$.

\vs

(b) Prove that the following facts are true about $s$ if
$s(t)=(a/2)t^2$ (the first fact will show why we switched from $c$ to
$a/2$):
\begin{itemize}
\item $s''(t)=a$ (the acceleration is constant).
\item $[s'(t)]^2=2as(t)$.
\end{itemize}

(c) If $s$ is measured in feet, the value of $a$ is 32. How many
seconds do you have to get out of the way of a chandelier which falls
from a 400-foot ceiling? If you don't make it, how fast will the
chandelier be going when it hits you? Where was the chandelier when it
was moving with half that speed?

\subsection*{Solution}

\subsection*{Problem 20}
Let $f$ be any polynomial function; we will see in the next chapter
that $f$ is differentiable. The tangent line to $f$ at $(a, f(a))$ is
the graph of $g(x)=f'(a)(x-a)+f(a)$. Thus $f(x)-g(x)$ is the
polynomial function $d(x)=f(x)-f'(a)(x-a)-f(a)$. We have already seen
that if $f(x)=x^2$, then $d(x)=(x-a)^2$, and if $f(x)=x^3$, then
$d(x)=(x-a)^2(x+2a)$.

\vs

(a) Find $d(x)$ when $f(x)=x^{4}$, and show that it is divisible by
$(x-a)^2$.

\vs

(b) There certainly seems to be some evidence that $d(x)$ is always
divisible by $(x-a)^2$. Figure 22 provides an intuitive argument:
usually, lines parallel to the tangent line will intersect the graph
at two points; the tangelnt line intersects the graph only once near
the point, so the intersection should be a ``double intersection''. To
give a rigorous proof, first note that
\[\frac{d(x)}{x-a}=\frac{f(x)-f(a)}{x-a}-f'(a)\]

Now answer the following questions. Why is $f(x)-f(a)$ divisible by
$(x-a)$? Why is there a polynomial function $h$ such that
$h(x)=d(x)/(x-a)$ for $x\neq a$? Why is $\lim_{x\to a}h(x)=0$? Why is
$h(a)=0$? Why does this solve the problem?

\subsection*{Solution}

\subsection*{Problem 26}
Find $f''(x)$ if

\vs

(i) $f(x)=x^3$

\vs

(ii) $f(x)=x^5$

\vs

(iii) $f'(x)=x^4$

\vs

(iv) $f(x+3)=x^5$


\subsection*{Solution}


%%% Local Variables:
%%% TeX-master: "hw"
%%% End:
