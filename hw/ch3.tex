\section{Chapter 3}
\subsection*{Problem 3ii}
Find the domain of the function defined by the following formula
\[f(x)=\sqrt{1-\sqrt{1-x^2}}\]

\subsubsection*{Solution}
There are two constraints: $x^2\leq1$ and $\sqrt{1-x^2}\leq1$. Taking a square root of the first constraint we get $x\leq1$ and $x\geq-1$. With the second constraint:
\begin{align*}
    &\sqrt{1-x^2}\leq1\\
    &\implies 1-x^2\leq1\\
    &\implies x^2\geq0\\
\end{align*}

Observe that $x^2\geq 0$ is true for all $x$. Thus the domain of $f$ is $[-1,1]$.

\subsection*{Problem 9a}
If $A$ is any set of real numbers, define a function $C_A$ as follows:
\[C_A(x)=\begin{cases}
    1, x\in A\\
    2, x\not\in A
\end{cases}\]

Find expressions for $C_{A\cap B}$ and $C_{A\cup B}$ and $C_{\R-A}$, in terms of $C_A$ and $C_B$. 

\subsubsection*{Solution}
\begin{align*}
    &C_{A\cap B}=C_A\cdot C_B\\
    &C_{A\cup B}(x)=C_A+C_B-C_A\cdot C_B\\
    &C_{\R-A}=1-C_A
\end{align*}

\subsection*{Problem 12a,b,c}
A function $f$ is \textbf{even} if $f(x)=f(-x)$ and \textbf{odd} if $f(x)=-f(-x)$. For example, $f$ is even if $f(x)=x^2$ or $f(x)=|x|$ or $f(x)=\cos x$, while $f$ is odd if $f(x)=x$ or $f(x)=\sin x$.

\subsubsection*{Solution}
(a) Determine whether $f+g$ is even, odd, or not not necessarily either, in the four cases obtained by choosing $f$ even or odd and $g$ even or odd.

\vs

Suppose both $f,g$ are even. Then $f+g$ is even:
\[(f+g)(x)=f(-x)+g(-x)=(f+g)(-x)\]

Suppose both $f,g$ are odd. Then $f+g$ is odd:
\[(f+g)(x)=-f(-x)-g(-x)=-(f+g)(-x)\]

Suppose $f$ is even, $g$ is odd. Then $f+g$ is neither:
\[(f+g)(x)=f(-x)-g(-x)=(f-g)(-x)\]

Suppose $f$ is odd, $g$ is even. Then $f+g$ is neither:
\[(f+g)(x)=-f(-x)+g(-x)=(g-f)(-x)\]

(b) Do the same for $f\cdot g$.

\vs

Suppose both $f,g$ are even. Then $f\cdot g$ is even:
\[(f\cdot g)(x)=f(-x)\cdot g(-x)=(f\cdot g)(-x)\]

Suppose both $f,g$ are odd. Then $f\cdot g$ is even:
\[(f\cdot g)(x)=-f(-x)\cdot -g(-x)=(f\cdot g)(-x)\]

Suppose $f$ is even, $g$ is odd. Then $f\cdot g$ is odd:
\[(f\cdot g)(x)=f(-x)\cdot -g(-x)=-(f\cdot g)(-x)\]

Suppose $f$ is odd, $g$ is even. Then $f\cdot g$ is odd:
\[(f\cdot g)(x)=-f(-x)\cdot g(-x)=-(f\cdot g)(-x)\]


(c) Do the same for $f\circ g$.

\vs

Suppose both $f,g$ are even. Then $f\circ g$ is even:
\[(f\circ g)(x)=f(g(x))=f(g(-x))=(f\circ g)(-x)\]

Suppose both $f,g$ are odd. Then $f\circ g$ is odd:
\[(f\circ g)(x)=f(g(x))=-f(g(-x))=(-f\circ g)(-x)=-(f\circ g)(-x)\]

Suppose $f$ is even, $g$ is odd. Then $f\circ g$ is even:
\[(f\circ g)(x)=f(g(x))=f(-g(-x))=f(g(-x))=(f\circ g)(-x)\]

Suppose $f$ is odd, $g$ is even. Then $f\circ g$ is even:
\[(f\circ g)(x)=f(g(x))=f(g(-x))=(f\circ g)(-x)\]

%%% Local Variables:
%%% TeX-master: "index"
%%% End:
