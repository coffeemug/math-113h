\section{Continuity, Part II}

\subsection{Intermediate Value Theorem}\label{ivt}
\textbf{Theorem:} if $f$ is continuous on $[a,b]$ and $f(a)<0<f(b)$,
then there exists $x\in[a,b]$ such that $f(x)=0$.

\vs

Or intuitively, if $f(a)$ is below zero and $f(b)$ is above zero, $f$
must cross the $x$-axis somewhere.

\vs

\textbf{Proof:} intuitively, we will locate the smallest number $x$ on
the $x$-axis where $f(x)$ first crosses from negative to positive, and
show that $f(x)$ must be zero.

\vs

First, we define a set $A$ that contains all inputs to $f$ before $f$
crosses from negative to positive for the first time:
\[A=\{x:a\leq x\leq b, \text{and $f$ is negative on the interval
    $[a,x]$}\}\]

We know $A\neq\emptyset$ since $a\in A$, and $b$ is an upper bound of
$A$. Thus $A$ has a least upper bound $\alpha$ such that
$a\leq\alpha\leq b$. By nonzero neighborhood lemma (see
\ref{subsubsec:nonzero-lemma}) we know there is some interval around
$a$ on which $f$ is negative, and some interval around $b$ on which
$f$ is positive. Thus we can further refine the bound on $\alpha$ to
$a<\alpha<b$.

\vs

We now show $f(\alpha)=0$ by eliminating the possibilities
$f(\alpha)<0$ and $f(\alpha)>0$.

\vs

\textbf{Case 1.} Suppose for contradiction $f(\alpha)<0$. By nonzero
neighborhood lemma there exists $\delta>0$ such $|x-\alpha|<\delta$ implies
$f(x)<0$ for all $x$. But that means numbers in
$(\alpha-\delta, \alpha+\delta)$ are in $A$. E.g.
$(\alpha+\delta/2)\in A$. Since $\alpha+\delta/2>\alpha$, $\alpha$ is not an upper bound of
$A$, and is thus not the least upper bound.

\vs


\textbf{Case 2.} Suppose for contradiction $f(\alpha)>0$. By nonzero
neighborhood lemma there exists $\delta>0$ such $|x-\alpha|<\delta$ implies
$f(x)>0$ for all $x$. But that means numbers in
$(\alpha-\delta, \alpha+\delta)$ are \textit{not} in $A$, and there exist many upper
bounds of $A$ less than $\alpha$. E.g. $\alpha-\delta/2$ is an upper bound of
$A$, and since $\alpha-\delta/2<\alpha$, $\alpha$ is not the \textit{least} upper bound.

\vs

Both cases lead to contradiction, therefore $f(\alpha)=0$. QED.


\subsection{Extreme Value Theorem}
\subsection{Appendix: IVT and EVT consequences}

%%% Local Variables:
%%% TeX-master: "notes"
%%% End:
