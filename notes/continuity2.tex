\section{Continuity, Part II}

\subsection{Intermediate Value Theorem}\label{ivt}
\textbf{Theorem:} if $f$ is continuous on $[a,b]$ and $f(a)<0<f(b)$,
then there exists $x\in[a,b]$ such that $f(x)=0$.

\vs

Or intuitively, if $f(a)$ is below zero and $f(b)$ is above zero, $f$
must cross the $x$-axis somewhere.

\vs

\textbf{Proof:} intuitively, we will locate the smallest number $x$ on
the $x$-axis where $f(x)$ first crosses from negative to positive, and
show that $f(x)$ must be zero.

\vs

First, we define a set $A$ that contains all inputs to $f$ before $f$
crosses from negative to positive for the first time:
\[A=\{x:a\leq x\leq b, \text{and $f$ is negative on the interval
    $[a,x]$}\}\]

We know $A\neq\emptyset$ since $a\in A$, and $b$ is an upper bound of
$A$. Thus $A$ has a least upper bound $\alpha$ such that
$a\leq\alpha\leq b$. By nonzero neighborhood lemma (see
\ref{subsubsec:nonzero-lemma}) we know there is some interval around
$a$ on which $f$ is negative, and some interval around $b$ on which
$f$ is positive. Thus we can further refine the bound on $\alpha$ to
$a<\alpha<b$.

\vs

We now show $f(\alpha)=0$ by eliminating the possibilities
$f(\alpha)<0$ and $f(\alpha)>0$.

\vs

\textbf{Case 1.} Suppose for contradiction $f(\alpha)<0$. By nonzero
neighborhood lemma there exists $\delta>0$ such $|x-\alpha|<\delta$ implies
$f(x)<0$ for all $x$. But that means numbers in
$(\alpha-\delta, \alpha+\delta)$ are in $A$. E.g.
$(\alpha+\delta/2)\in A$. Since $\alpha+\delta/2>\alpha$, $\alpha$ is not an upper bound of
$A$, and is thus not the least upper bound.

\vs


\textbf{Case 2.} Suppose for contradiction $f(\alpha)>0$. By nonzero
neighborhood lemma there exists $\delta>0$ such $|x-\alpha|<\delta$ implies
$f(x)>0$ for all $x$. But that means numbers in
$(\alpha-\delta, \alpha+\delta)$ are \textit{not} in $A$, and there exist many upper
bounds of $A$ less than $\alpha$. E.g. $\alpha-\delta/2$ is an upper bound of
$A$, and since $\alpha-\delta/2<\alpha$, $\alpha$ is not the \textit{least} upper bound.

\vs

Both cases lead to contradiction, therefore $f(\alpha)=0$. QED.

\subsubsection*{IVT generalization}

The intermediate value theorem is usually presented in a more
general way. If $f$ is continuous on $[a,b]$ and $f(a)<c<f(b)$ or
$f(a)>c>f(b)$ then there is some $x$ in $[a,b]$ such that $f(x)=c$.

\vs

Intuitively, $f$ takes on any value between $f(a)$ and $f(b)$ at some
point in the interval $[a,b]$.

\vs

\textbf{Proof.} This trivially follows from the the theorem as
initially stated. There are two cases:

\vs

\textit{Case 1:} $f(a)<c<f(b)$. Let $g=f-c$. Then $g$ is continuous
and $g(a)<0<g(b)$. Thus there is some $x$ in $[a,b]$ such that
$g(x)=0$. But that means $f(x)=c$.

\vs

\textit{Case 2:} $f(a)>c>f(b)$. Observe that $-f$ is continuous on
$[a,b]$ and $-f(a)<-c<-f(b)$. By case 1 there is some $x$ in $[a,b]$
such that $-f(x)=-c$, which means $f(x)=c$.

\vs

QED.

\subsection{Boundedness theorem}

The boundedness theorem states that if $f$ is continuous on $[a,b]$,
then $f$ is bounded above (i.e. $f$ lies below some line). Before we
prove this, we first prove a simple lemma.

\vs

\textbf{Bounded neighborhood lemma:} if $f$ is continuous at $a$, then
there is $\delta>0$ such that $f$ is bounded above on the interval
$(a-\delta, a+\delta)$.

\vs

Intuitively, if $f$ is continuous at $a$ then there is some interval
around $a$ on which $f$ is bounded above.

\vs

\textbf{Proof:} The proof is trivial. Inlining the definition of
continuity, for any $\epsilon>0$ there exists $\delta>0$ such that
$|x-a|<\delta$ implies $|f(x)-f(a)|<\epsilon$ for all $x$. Thus
$f(a)+\epsilon$ is the upper bound on $f$ within $(a-\delta, a+\delta)$, as desired.

\vs

(Note that we can pick any $\epsilon$ to concretize the proof, for example
$\epsilon=1$.)

\vs


\textbf{Boundedness theorem:} if $f$ is continuous on
$[a,b]$, then $f$ is bounded above on $[a,b]$. I.e. there is some
numbers $N$ such that $f(x)\leq N$ for all $x$ in $[a,b]$.

\vs

\textbf{Proof:} intuitively, we will try to find the smallest number
$x$ on the $x$-axis where $f(x)$ becomes unbounded above, and discover
that there is no such number in $[a,b]$.

\vs

First, we define a set $A$ that contains all inputs to $f$ before $f$
stops being bounded above:
\[A=\{x:a\leq x\leq b \text{, and $f$ is bounded above on $[a,x]$}\}\]

By bounded neighborhood lemma $f$ is bounded above in the neighborhood
of $a$\footnote{We are being sloppy here as we actually need a
  left-sided and right-sided version of the bounded neighborhood
  lemma. I am papering over this for now, but will need to fix at some
  point by giving proper one sided proofs}. Thus we know
$A\neq\emptyset$ because $a\in A$. Further, $b$ is an upper bound of
$A$. Thus $A$ has a least upper bound.

\vs

Let $\alpha=\sup A$. To prove the boundedness theorem we must prove two
claims:
\begin{enumerate}
\item $\alpha=b$, i.e. $f$ does not ever stop being bounded above before
  $b$.
\item $(\alpha=b)\in A$, as $\sup A$ is not necessarily a member of $A$.
\end{enumerate}

\textit{First}, we prove $\alpha=b$. Suppose for contradiction $\alpha<b$. By
bounded neighborhood lemma there is some $\delta>0$ such that $f$ is
bounded above in $(\alpha-\delta, \alpha+\delta)$. But that means there are many upper
bounds greater than $\alpha$, for example $\alpha+\delta/2$. Thus $\alpha$ is not the
\textit{least} upper bound. We have a contradiction, and so $\alpha=b$.

\vs

\textit{Second}, we prove $(\alpha=b)\in A$. By bounded neighborhood lemma
there is some $\delta>0$ such that $f$ is bounded above in $(b-\delta, b]$. Pick
any $x_{0}$ such that $b-\delta<x_{0}<b$. Then:
\begin{itemize}
\item $x_{0}<b=\alpha$. Since $\alpha$ is the least upper bound it follows
  $x_{0}\in A$. Thus $f$ is bounded above on $[a,x_{0}]$.
\item $f$ is bounded above on $[x_{0}, b]$.
\end{itemize}
Since $f$ is bounded above on $[a,x_{0}]$ and on $[x_{0}, b]$, it
follows $f$ is bounded above on $[a,b]$ as desired. QED.

\subsubsection*{Boundedness theorem generalization}
The \textit{boundedness theorem} is usually presented slightly more
generally: it proves $f$ is bounded above \textit{and} below. We
already proved the former. Put more formally, the latter states:

\vs

If $f$ is continuous on $[a,b]$, then $f$ is bounded \textit{below} on
$[a,b]$. I.e. there is some numbers $N$ such that $f(x)\geq N$ for all
$x$ in $[a,b]$.

\vs

\textbf{Proof:} observe that $-f$ is continuous on $[a,b]$. By claim 2
there exists a number $M$ such that $-f(x)\leq M$ for all $x$ in
$[a,b]$. But that means $f(x)\geq -M$ for all $x$ in $[a,b]$. QED.


\subsection{Extreme Value Theorem}

The \textit{extreme value theorem} states that is $f$ is continuous on
$[a,b]$, then $f$ attains its maximum on $[a,b]$. To see why we need
the extreme value theorem, consider $f=\frac{1}{x}$. $f$ is
discontinuous at $0$ and approaches infinity. Thus $f$ does not attain
a maximum value on the interval $[0,1]$.

\vs

\textbf{Extreme value theorem:} If $f$ is continuous on $[a,b]$, then
there is some number $y$ in $[a,b]$ such that $f(y)\geq f(x)$ for all
$x$ in $[a,b]$.

\vs

\textbf{Proof:} Let $A$ be the set of $f$'s outputs on $[a,b]$:
\[A=\{f(x):x\text{ in }[a,b]\}\]

Since $[a,b]$ isn't empty, $A\neq\emptyset$. By boundedness theorem,
$f$ is bounded on $[a,b]$, and so $A$ has an upper bound. Thus $A$ has
a least upper bound. Let $\alpha=\sup A$. By definition
$\alpha\geq f(x)$ for $x$ in $[a,b]$. Thus it suffices to show
$\alpha\in A$ (i.e. $\alpha=f(y)$ for some $y$ in $[a,b]$).

\vs

Let's consider a function $g$\footnote{$g$ is a bit of a rabbit pulled
  out of a magic hat, but to quote a great British statesman, them's
  the breaks}:
\[g=\frac{1}{\alpha-f(x)},\ \ \ x \text{ in }\ [a,b]\]

Suppose for contradiction $\alpha\notin A$. Then the denominator is never zero
and $g$ is continuous. Therefore:
\begin{align*}
&\frac{1}{\alpha-f(x)}<M&&\text{by boundedness theorem}\\
& &&\text{for some bound $M$}\\
&\implies \alpha-f(x)>\frac{1}{M}&&\text{take reciprocal}\\
&\implies -f(x)>\frac{1}{M}-\alpha\\
&\implies f(x)<\alpha-\frac{1}{M}&&\text{times $-1$}
\end{align*}
But this contradicts that $\alpha$ is the \textit{least} upper bound. Thus
$\alpha\in A$ as desired. QED.

\subsubsection*{EVT generalization}

The extreme value theorem is usually presented slightly more
generally: a continuous $f$ attains both its maximum and its minimum.
We already proved the former. Put more formally, the latter states:

\vs

If $f$ is continuous on $[a,b]$, then there is some number $y$ in
$[a,b]$ such that $f(y)\leq f(x)$ for all $x$ in $[a,b]$.

\vs

\textbf{Proof:} Observe that $-f$ is continous on $[a,b]$. By claim 3
there is some $y$ in $[a,b]$ such that $-f(y)\geq-f(x)$ for all $x$ in
$[a,b]$. But that means that $f(y)\leq f(x)$ for all $x$ in $[a,b]$. QED.


\subsection{Appendix: IVT and EVT consequences}

\textbf{Claim 1a:} Every positive number has a square root. I.e. if
$\alpha>0$, then there is some number $x$ such that $x^{2}=\alpha$.

\vs

\textbf{Proof:} Consider the function $f(x)=x^{2}$. If $f$ takes on
the value of $\alpha$ as its output, then $x=\sqrt{\alpha}$ is the input (i.e.
$x^{2}=\alpha$). Thus all we must show is that $f$ takes on the value of
$\alpha$.

\vs

We can do it as follows. Show there exist $a,b$ such that
$f(a)<\alpha<f(b)$. Since $f$ is continuous, by intermediate value theorem
there exists $x$ such that $f(x)=\alpha$. So, let's find $a$ and $b$:
\begin{itemize}
\item First, find $a$ such that $f(a)<\alpha$. Observe that
  $f(0)=0<\alpha$, thus fix $a=0$.
\item Second, find $b$ such that $\alpha<f(b)$.
  \begin{itemize}
  \item If $\alpha<1$ then $f(1)=1>\alpha$. Thus fix $b=1$.
  \item If $\alpha>1$ then $f(\alpha)=\alpha^{2}>\alpha$. Thus fix $b=\alpha$.
  \end{itemize}
\end{itemize}

By intermediate value theorem, there is some $x$ in $[0,b]$ such that
$f(x)=\alpha$. QED.

\vs

\textbf{Claim 1b:} Every positive number has an $n$th root. I.e. if
$\alpha>0$, then there is some number $x$ such that $x^{n}=\alpha$.

\vs

\textbf{Proof:} We can use the exact same argument as 1a, just
consider $f(x)=x^{n}$.

\vs

\textbf{Claim 1c:} Let $n$ be odd. Then every number has an $n$th
root. I.e. there is some number $x$ such that $x^{n}=\alpha$ for all $\alpha$.

\vs

\textbf{Proof:} This is also easy:
\begin{itemize}
\item \textit{Case} $\alpha>0$. By claim 2b, there is an $x$ such that $x^{n}=\alpha$.
\item \textit{Case} $\alpha<0$. By claim 2b, there is an $x$ such that
  $x^{n}=-\alpha$. Then $(-x)^{n}=\alpha$.
\end{itemize}

QED.

\vs

\textbf{Claim 2:} If $n$ is odd, then any equation of the form
\[x^{n}+a_{n-1}x^{n-1}+\ldots+a_{0}=0\]
has a root.

\vs

\textbf{Proof:} Let $f(x)=x^{n}+a_{n-1}x^{n-1}+\ldots+a_{0}$. Here is an
intuitive outline of the proof:
\begin{enumerate}
\item We will show that $f$ must take on negative and positive values.
  Thus by the intermediate value theorem, there exists some $x$ such
  that $f(x)=0$.
\item To do that we will show that as $|x|$ gets large, $x^{n}$
  completely dominates other terms. (This is obvious if you consider
  Big-Oh of each term.)
\item Since $n$ is odd, $x^{n}$ takes on a negative value when $x$ is
  negative, and a positive value when $x$ is positive. And since
  $x^{n}$ dominates other terms, when $x$ is sufficiently large, $f$
  takes on both negative and positive values.
\end{enumerate}

We must find a way to bound the magnitude of
$a_{n-1}x^{n-1}+\ldots+a_{0}$ to show that for large enough $x$, it's
smaller than the magnitude of $x^n$. This way we guarantee $f(x)$ has
the same sign as $x^n$. This is trivial to do by adopting Big-Oh
notation, but both math books I looked at do it the old-fashioned way,
so we will too.

\vs

Let's start with some obvious transformations we can make:
\begin{align*}
  |a_{n-1}x^{n-1}+\ldots+a_{0}| &\leq |a_{n-1}x^{n-1}|+\ldots+|a_{0}| &&\text{by
                                                            triangle
                                                            inequality}\\
                           &=|a_{n-1}||x^{n-1}|+\ldots+|a_{0}|
                                                         &&\text{in
                                                            general } |ab|=|a||b|\\
\end{align*}

If we only consider behavior of $f$ on large $x$ (i.e. when $|x|>1$),
we can further bound the expression. Observe that when $|x|>1$ then
$x^{n-1}>x^{n-2}>\ldots>x>1$. Therefore:
\begin{align*}
  |a_{n-1}x^{n-1}+\ldots+a_{0}| &\leq |a_{n-1}||x^{n-1}|+\ldots+|a_{0}|\\
                           &\leq |a_{n-1}||x^{n-1}|+\ldots+|a_{0}||x^{n-1}|\\
                           &= x^{n-1}(|a_{n-1}|+\ldots+|a_0|)
\end{align*}

Let $M=|a_{n-1}|+\ldots+|a_0|+1$, i.e. a bound on the sum of the
coefficients, plus a little extra to ensure $M>1$. Then
\begin{align*}
  |a_{n-1}x^{n-1}+\ldots+a_{0}| &\leq x^{n-1}(|a_{n-1}|+\ldots+|a_0|)\\
                           &< M|x^{n-1}|
\end{align*}

Given this bound it follows that for all $|x|>1$:
\[x^{n}-M|x^{n-1}|<x^{n}+(a_{n-1}x^{n-1}+\ldots+a_{0})<x^{n}+M|x^{n-1}|\]

or put differently:
\[x^{n}-M|x^{n-1}|<f(x)<x^{n}+M|x^{n-1}|\]

We will now find $x_1$ and $x_2$ such that $f(x_1)<0$ and $f(x_2)>0$.
Let $x_1=-2M$ (note that $x_1$ satisfies our condition $|x_1|>1$ since
$M>1$). Then for all $x\leq x_1$:
\begin{align*}
  f(x)&<x^{n}+M|x^{n-1}|\\
      &=x^n+Mx^{n-1}&&n \text{ is odd, thus } n-1\text{ is even, thus } x^{n-1}>0\\
      &=x^{n-1}(x+M)&&\text{factor out } x^{n-1}\\
      &\leq -2^{n-1}M^n&&\text{substitute $-2M$ and simplify}\\
      &<0
\end{align*}

Similarly, let $x_2=2M$. Then for all $x\geq x_2$:
\begin{align*}
  f(x)&>x^{n}-M|x^{n-1}|\\
      &=x^n-Mx^{n-1}\\
      &=x^{n-1}(x-M)\\
      &\geq 2^{n-1}M^n\\
      &>0
\end{align*}

QED.

\vs

\textbf{Claim 3:} If $n$ is even and
$f(x)=x^{n}+a_{n-1}x^{n-1}+\ldots+a_{0}$, then there is a number $y$ such
that $f(y)\leq f(x)$ for all $x$.

\vs

Intuitively, even degree polynomials achieve their minimum on
$\mathcal{R}$ because when you zoom out enough they are U-shaped (consider the
graph $f(x)=x^2$ as a simple example).

\vs

\textbf{Proof:} It's easy to intuitively see why the claim makes
sense. $x^n$ dominates the rest of the terms when $x$ is very large.
Since $n$ is even, $x^n>0$. Thus on very large $|x|$ the graph shoots
up (i.e. it has a U shape).

\vs

Here is the outline for our proof:

\begin{enumerate}
\item Observe that $f(0)=a_0$.
\item We will prove $f$ is U-shaped by proving there exist two points:
  \begin{itemize}
  \item $x_0<0$ such that $f(x)>a_0$ on $(-\infty, x_0]$.
  \item $x_1>0$ such that $f(x)>a_0$ on $[x_1, \infty]$.
  \end{itemize}
\item By extreme value theorem $f$ achieves a minimum $m$ on
  $[x_0, x_1]$. Note $m\leq a_{0}$ (otherwise it wouldn't be a minimum).
\item Thus $f$ achieves a minimum $m$ on $\mathcal{R}$, as we've shown that
  outside $[x_0, x_1]$, $f(x)>a_{0}$ (and thus $f(x)>m$) for all $x$.
\end{enumerate}

All we must do now to complete the proof is find $x_0<0<x_1$. Let
$M=|a_{n-1}|+\ldots+|a_0|+1$, i.e. a bound on the sum of the coefficients,
plus a little extra to ensure $M>1$. In Claim 2 we discovered that for
$|x|>1$
\[x^{n}-M|x^{n-1}|<f(x)<x^{n}+M|x^{n-1}|\]

Let $x_1=-2M$. Note that $x_1$ satisfies our condition $|x_1|>1$ since
$M>1$. Then for all $x<x_1$:
\begin{align*}
  f(x)&>x^{n}-M|x^{n-1}|\\
      &=x^n+Mx^{n-1}&&\text{$x$ is negative, and $n-1$ is odd}\\
      &=x^{n-1}(x+M)\\
      &\geq 2^{n-1}M^n&&\text{substitute $-2M$ and simplify}
\end{align*}

Similarly let $x_2=2M$. Then for all $x>x_1$:
\begin{align*}
  f(x)&>x^{n}-M|x^{n-1}|\\
      &=x^n+Mx^{n-1}&&\text{$x$ is positive}\\
      &=x^{n-1}(x+M)\\
      &\geq 2^{n-1}M^n&&\text{substitute $2M$ and simplify}
\end{align*}

Since $M>1$ we have
\[2^{n-1}M^n\geq M\geq |a_n+1|\geq a_n+1>a_n\]

Therefore for all $x<x_1$ and $x>x_2$, $f(x)>a_{n}$ as desired.

\vs

\textbf{Claim 4:} Consider the equation
\[x^{n}+a_{n-1}x^{n-1}+\ldots+a_{0}=c\]
and suppose $n$ is even. Then there is a number $m$ such that the
equation has a solution for $c\geq m$ and has no solution for $c<m$.

\vs

\textbf{Proof:} TBD.

\vs



%%% Local Variables:
%%% TeX-master: "notes"
%%% End:
