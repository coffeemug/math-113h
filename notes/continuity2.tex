\section{Continuity, Part II}

\subsection{Intermediate Value Theorem}\label{ivt}
\textbf{Theorem:} if $f$ is continuous on $[a,b]$ and $f(a)<0<f(b)$,
then there exists $x\in[a,b]$ such that $f(x)=0$.

\vs

Or intuitively, if $f(a)$ is below zero and $f(b)$ is above zero, $f$
must cross the $x$-axis somewhere.

\vs

\textbf{Proof:} intuitively, we will locate the smallest number $x$ on
the $x$-axis where $f(x)$ first crosses from negative to positive, and
show that $f(x)$ must be zero.

\vs

First, we define a set $A$ that contains all inputs to $f$ before $f$
crosses from negative to positive for the first time:
\[A=\{x:a\leq x\leq b, \text{and $f$ is negative on the interval
    $[a,x]$}\}\]

We know $A\neq\emptyset$ since $a\in A$, and $b$ is an upper bound of
$A$. Thus $A$ has a least upper bound $\alpha$ such that
$a\leq\alpha\leq b$. By nonzero neighborhood lemma (see
\ref{subsubsec:nonzero-lemma}) we know there is some interval around
$a$ on which $f$ is negative, and some interval around $b$ on which
$f$ is positive. Thus we can further refine the bound on $\alpha$ to
$a<\alpha<b$.

\vs

We now show $f(\alpha)=0$ by eliminating the possibilities
$f(\alpha)<0$ and $f(\alpha)>0$.

\vs

\textbf{Case 1.} Suppose for contradiction $f(\alpha)<0$. By nonzero
neighborhood lemma there exists $\delta>0$ such $|x-\alpha|<\delta$ implies
$f(x)<0$ for all $x$. But that means numbers in
$(\alpha-\delta, \alpha+\delta)$ are in $A$. E.g.
$(\alpha+\delta/2)\in A$. Since $\alpha+\delta/2>\alpha$, $\alpha$ is not an upper bound of
$A$, and is thus not the least upper bound.

\vs


\textbf{Case 2.} Suppose for contradiction $f(\alpha)>0$. By nonzero
neighborhood lemma there exists $\delta>0$ such $|x-\alpha|<\delta$ implies
$f(x)>0$ for all $x$. But that means numbers in
$(\alpha-\delta, \alpha+\delta)$ are \textit{not} in $A$, and there exist many upper
bounds of $A$ less than $\alpha$. E.g. $\alpha-\delta/2$ is an upper bound of
$A$, and since $\alpha-\delta/2<\alpha$, $\alpha$ is not the \textit{least} upper bound.

\vs

Both cases lead to contradiction, therefore $f(\alpha)=0$. QED.

\subsubsection*{IVT generalization}

The intermediate value theorem is usually presented in a more
general way. If $f$ is continuous on $[a,b]$ and $f(a)<c<f(b)$ or
$f(a)>c>f(b)$ then there is some $x$ in $[a,b]$ such that $f(x)=c$.

\vs

Intuitively, $f$ takes on any value between $f(a)$ and $f(b)$ at some
point in the interval $[a,b]$.

\vs

\textbf{Proof.} This trivially follows from the the theorem as
initially stated. There are two cases:

\vs

\textit{Case 1:} $f(a)<c<f(b)$. Let $g=f-c$. Then $g$ is continuous
and $g(a)<0<g(b)$. Thus there is some $x$ in $[a,b]$ such that
$g(x)=0$. But that means $f(x)=c$.

\vs

\textit{Case 2:} $f(a)>c>f(b)$. Observe that $-f$ is continuous on
$[a,b]$ and $-f(a)<-c<-f(b)$. By case 1 there is some $x$ in $[a,b]$
such that $-f(x)=-c$, which means $f(x)=c$.

\vs

QED.

\subsection{Extreme Value Theorem}
We will build up to the \textit{extreme value theorem} by proving
three progressively more important claims (the last one being the
theorem itself):
\begin{enumerate}
\item If $f$ is continuous at $a$ then there is some interval around
  $a$ on which $f$ is bounded above.
\item If $f$ is continuous on $[a,b]$, then $f$ is bounded above on
  $[a,b]$.
\item Finally, if $f$ is continuous on $[a,b]$, $f$ attains its
  maximum on $[a,b]$.
\end{enumerate}

To see why we need the extreme value theorem, consider
$f=\frac{1}{x}$. $f$ is discontinuous at $0$ and approaches infinity.
Thus $f$ does not attain a maximum value on the interval $[0,1]$.

\vs

\textbf{Claim 1 (bounded neighborhood lemma):} if $f$ is continuous at $a$, then there is $\delta>0$
such that $f$ is bounded above on the interval $(a-\delta, a+\delta)$.

\vs

\textbf{Proof:} The proof is trivial. Inlining the definition of
continuity, for any $\epsilon>0$ there exists $\delta>0$ such that
$|x-a|<\delta$ implies $|f(x)-f(a)|<\epsilon$ for all $x$. Thus
$f(a)+\epsilon$ is the upper bound on $f$ within $(a-\delta, a+\delta)$, as desired.

\vs

(Note that we can pick any $\epsilon$ to concretize the proof, for example
$\epsilon=1$.)

\vs

\textbf{Claim 2 (boundedness theorem):} if $f$ is continuous on
$[a,b]$, then $f$ is bounded above on $[a,b]$. I.e. there is some
numbers $N$ such that $f(x)\leq N$ for all $x$ in $[a,b]$.

\vs

Intuitively, the claim means the graph of $f$ lies below some line.

\vs

\textbf{Proof:} intuitively, we will try to find the smallest number
$x$ on the $x$-axis where $f(x)$ becomes unbounded above, and discover
that there is no such number in $[a,b]$.

\vs

First, we define a set $A$ that contains all inputs to $f$ before $f$
stops being bounded above:
\[A=\{x:a\leq x\leq b \text{, and $f$ is bounded above on $[a,x]$}\}\]

By bounded neighborhood lemma $f$ is bounded above in the neighborhood
of $a$\footnote{We are being sloppy here as we actually need a
  left-sided and right-sided version of the bounded neighborhood
  lemma. I am papering over this for now, but will need to fix at some
  point by giving proper one sided proofs}. Thus we know
$A\neq\emptyset$ because $a\in A$. Further, $b$ is an upper bound of
$A$. Thus $A$ has a least upper bound.

\vs

Let $\alpha=\sup A$. To prove the boundedness theorem we must prove two
claims:
\begin{enumerate}
\item $\alpha=b$, i.e. $f$ does not ever stop being bounded above before
  $b$.
\item $(\alpha=b)\in A$, as $\sup A$ is not necessarily a member of $A$.
\end{enumerate}

\textit{First}, we prove $\alpha=b$. Suppose for contradiction $\alpha<b$. By
bounded neighborhood lemma there is some $\delta>0$ such that $f$ is
bounded above in $(\alpha-\delta, \alpha+\delta)$. But that means there are many upper
bounds greater than $\alpha$, for example $\alpha+\delta/2$. Thus $\alpha$ is not the
\textit{least} upper bound. We have a contradiction, and so $\alpha=b$.

\vs

\textit{Second}, we prove $(\alpha=b)\in A$. By bounded neighborhood lemma
there is some $\delta>0$ such that $f$ is bounded above in $(b-\delta, b]$. Pick
any $x_{0}$ such that $b-\delta<x_{0}<b$. Then:
\begin{itemize}
\item $x_{0}<b=\alpha$. Since $\alpha$ is the least upper bound it follows
  $x_{0}\in A$. Thus $f$ is bounded above on $[a,x_{0}]$.
\item $f$ is bounded above on $[x_{0}, b]$.
\end{itemize}
Since $f$ is bounded above on $[a,x_{0}]$ and on $[x_{0}, b]$, it
follows $f$ is bounded above on $[a,b]$ as desired. QED.

\vs

\textbf{Claim 3: (extreme value theorem):} If $f$ is continuous on
$[a,b]$, then there is some number $y$ in $[a,b]$ such that
$f(y)\geq f(x)$ for all $x$ in $[a,b]$.

\vs

\textbf{Proof:} TBD.

\subsubsection*{EVT generalization}

Claim 2 above proves a continuous $f$ is bounded above. The
\textit{boundedness theorem} is usually presented slightly more
generally, i.e. it proves $f$ is bounded above \textit{and} below.
This is easy to derive from claim 2.

\vs

\textbf{Claim 2a (boundedness theorem):} if $f$ is continuous on
$[a,b]$, then $f$ is bounded \textit{below} on $[a,b]$. I.e. there is
some numbers $N$ such that $f(x)\geq N$ for all $x$ in $[a,b]$.

\vs

\textbf{Proof:} observe that $-f$ is continuous on $[a,b]$. By claim 2
there exists a number $M$ such that $-f(x)\leq M$ for all $x$ in
$[a,b]$. But that means $f(x)\geq -M$ for all $x$ in $[a,b]$. QED.

\vs

Similarly, claim 3 above proves a continuous $f$ attains its maximum.
The extreme value theorem is usually presented slightly more
generally, i.e. it states a continuous $f$ attains both its maximum
and its minimum. Again, this is easy to derive from claim 3.

\vs

\textbf{Cliam 3a (extreme value theorem)}: If $f$ is continuous on
$[a,b]$, then there is some number $y$ in $[a,b]$ such that
$f(y)\leq f(x)$ for all $x$ in $[a,b]$.

\vs

\textbf{Proof:} Observe that $-f$ is continous on $[a,b]$. By claim 3
there is some $y$ in $[a,b]$ such that $-f(y)\geq-f(x)$ for all $x$ in
$[a,b]$. But that means that $f(y)\leq f(x)$ for all $x$ in $[a,b]$. QED.


\subsection{Appendix: IVT and EVT consequences}

\textbf{Claim 1:} Every positive number has a square root. I.e. if
$\alpha>0$, then there is some number $x$ such that $x^{2}=\alpha$.

\vs

\textbf{Proof:} TBD.

\vs

\textbf{Claim 2:} If $n$ is odd, then any equation of the form
\[x^{n}+a_{n-1}x^{n-1}+\ldots+a_{0}=0\]
has a root.

\vs

\textbf{Proof:} TBD.

\vs

\textbf{Claim 3:} If $n$ is even and
$f(x)=x^{n}+a_{n-1}x^{n-1}+\ldots+a_{0}$, then there is a number $y$ such
that $f(y)\leq f(x)$ for all $x$.

\vs

\textbf{Proof:} TBD.

\vs

\textbf{Claim 4:} Consider the equation
\[x^{n}+a_{n-1}x^{n-1}+\ldots+a_{0}=c\]
and suppose $n$ is even. Then there is a number $m$ such that the
equation has a solution for $c\geq m$ and has no solution for $c<m$.

\vs

\textbf{Proof:} TBD.

\vs



%%% Local Variables:
%%% TeX-master: "notes"
%%% End:
