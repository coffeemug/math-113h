
\section{Differentiation, Part I (Fundamentals)}

We now prove theorems that make differentiation of a large class of
functions easy.

\subsection{Basic proofs}

\textbf{Theorem 1.} If $f(x)=c$ then $f'(a)=0$ for all $a$.

\textbf{Proof:} we already proved this in the previous chapter.

\vs\vs

\textbf{Theorem 2.} If $f(x)=x$ then $f'(a)=1$ for all $a$.

\textbf{Proof:}
\[f'(a)=\lim_{h\to0}\frac{f(a+h)-f(a)}{h}=\lim_{h\to0}\frac{a+h-a}{h}=1\]

\vs

\textbf{Theorem 3.} If $f,g$ are differentiable at $a$, then
$(f+g)'(a)=f'(a)+g'(a)$.

\textbf{Proof:}
\begin{align*}
  (f+g)'(a)&=\lim_{h\to0}\frac{(f+g)(a+h)-(f+g)(a)}{h}\\
           &=\lim_{h\to0}\frac{f(a+h)+g(a+h)-f(a)-g(a)}{h}\\
           &=\lim_{h\to0}\frac{f(a+h)-f(a)}{h} +
             \lim_{h\to0}\frac{g(a+h)-g(a)}{h}\\
           &=f'(a)+g'(a)
\end{align*}

\vs

\textbf{Theorem 4.} If $f,g$ are differentiable at $a$, then
\[(f\cdot g)'(a)=f'(a)\cdot g(a)+f(a)\cdot g'(a)\]

\textbf{Proof:}
\begin{align*}
(f\cdot g)'(a)=&=\lim_{h\to0}\frac{(f\cdot g)(a+h)-(f\cdot g)(a)}{h}\\
           &=\lim_{h\to0}\frac{f(a+h)g(a+h)-f(a)g(a)}{h}\\
           &=\lim_{h\to0}\frac{f(a+h)g(a+h)-f(a)g(a) + f(a+h)g(a)-f(a+h)g(a)}{h}\\
           &=\lim_{h\to0}\frac{f(a+h)(g(a+h)-g(a)) + g(a)(f(a+h) - f(a))}{h}\\
           &=\lim_{h\to0}\left(f(a+h)\frac{g(a+h)-g(a)}{h}+g(a)\frac{f(a+h)-f(a)}{h}\right)\\
           &=\lim_{h\to0}f(a+h)\cdot\lim_{h\to0}\frac{g(a+h)-g(a)}{h}+\lim_{h\to0}g(a)\cdot\lim_{h\to0}\frac{f(a+h)-f(a)}{h}\\
           &=\lim_{h\to0}f(a+h)\cdot g'(a)+g(a)\cdot f'(a)
\end{align*}
Recall from \ref{diff-implies-cont} that $\lim_{h\to0}f(a+h)=f(a)$. Thus
\[(f\cdot g)'(a)=f(a)\cdot g'(a)+g(a)\cdot f'(a)\]

\vs

\textbf{Theorem 5.} If $g(x)=cf(x)$ then $g'(a)=c\cdot f'(a)$.

\textbf{Proof:} Let $h(x)=c$ so $g=h\cdot f$. Then by theorem 4:
\begin{align*}
  g'(x)&=h'(x)f(x)+f'(x)g(x)\\
       &=0\cdot f(x)+cf'(x)\\
       &=cf'(x)
\end{align*}

\vs

\textbf{Note}, theorems 1-5 imply:
\[(-f)'(a)=(-1\cdot f')(a)=-f'(a)\]
\begin{center}and\end{center}
\[(f-g)'(a)=(f+(-g))'(a)=f'(a)+(-g)'(a)=f'(a)-g'(a)\]

\vs

\textbf{Theorem 6.} If $f(x)=x^n$ for $n\in\mathcal{N}$, then $f'(a)=na^{n-1}$ for
all $a$.

\textbf{Proof.} We prove this by induction. For $n=1$, $f'(a)=1$ by
theorem 2.

\vs

Assume if $f(x)=x^n$ then $f'(a)=na^{n-1}$ for all $a$.

\vs

Let $I(x)=x$ and let $g(x)=x^{n+1}=xx^n$. Then $g(x)=I(x)\cdot f(x)$, i.e.
$g=I\cdot f$. By theorem 4:
\begin{align*}
  g'(a)&=(I\cdot f)'(a)\\
       &=I'(a)f(a)+I(a)f'(a)\\
       &=1\cdot a^n+a\cdot na^{n-1}\\
       &=a^n+na^n\\
       &=a^n(1+n)\\
       &=(n+1)a^n
\end{align*}

\subsection{Chain rule}
\subsection{Consequences}
\subsection{Trig}

%%% Local Variables:
%%% TeX-master: "notes"
%%% End:
