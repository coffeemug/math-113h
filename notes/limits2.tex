\section{Limits, Part II (Edge Cases)}

\subsection{Absence of limits}

What does it mean to say $L$ is not a limit of $f(x)$ at $a$? It flows
out of the definition-- there exist some $\epsilon$ such that for any
$\delta$ there exists an $x$ in $0<|x-a|<\delta$ such that $|f(x)-L|\geq\epsilon$.

\vs

A stronger version is to say there is no limit of $f(x)$ at $a$. To do
that we must prove that \textit{any} $L$ is not a limit of $f(x)$ at
$a$.

\subsubsection*{Example: Absolute value fraction}

Consider $f(x)=\frac{x}{|x|}$. It's easy to see that

\[f(x)=\begin{cases}
    -1 & \text{if } x<0\\
    1 & \text{if } x>0\\
\end{cases}\]

We will show there is no limit of $f(x)$ near $0$.

\paragraph{Weak version.}

First, let's prove a weak version-- that $\lim_{x\to 0}f(x)\neq 0$. That is
easy. Pick some reasonably small epsilon, say $\epsilon=\frac{1}{10}$. We
must show that for any $\delta$ there exists an $x$ in
$0<|x-a|<\delta$ such that $|f(x)-0|\geq \frac{1}{10}$.

\vs

Let's pick some arbitrary $x$ out of our permitted interval, say $x=\delta/2$. Then
\[|f(x)-0|=|f(\delta/2)|=\left|\frac{\delta/2}{|\delta/2|}\right|=1\geq\frac{1}{10}\]


\paragraph{Strong version.}

Now we prove that $\lim_{x\to 0}f(x)\neq L$ for \textit{any} $L$. Sticking
with $\epsilon=\frac{1}{10}$ we proceed as follows.

\vs

If $L<0$ take $x=\delta/2$. Then

\[|f(x)-L|=|f(\delta/2)-L|=\left|\frac{\delta/2}{|\delta/2|}-L\right|=|1-L|>\frac{1}{10}\]

Similarly if $L\geq 0$ take $x=-\delta/2$. Then

\[|f(x)-L|=|f(-\delta/2)-L|=\left|\frac{-\delta/2}{|-\delta/2|}-L\right|=|-1-L|>\frac{1}{10}\]

\subsubsection*{Example: Dirichlet function} \label{subsubsec:dirichlet}
The \textit{dirichlet} function $f$ is defined as follows:
\[
f(x) = 
\begin{cases} 
1 & \text{for rational } x,\\
0 & \text{for irrational } x.
\end{cases}
\]

We prove $\lim_{x\to a}f(x)$ does not exist for any $a$.

\paragraph{Proof.} Let $\epsilon=\frac{1}{10}$. Suppose for contradiction
there exists $L$ such that $\lim_{x\to a}f(x)=L$. There are two
possibilities: either $L\leq\frac{1}{2}$ or $L>\frac{1}{2}$.

\vs

First suppose $L\leq\frac{1}{2}$. Pick any rational $x$ from the interval
$0<|x-a|<\delta$. Then $|f(x)-L|=|1-L|\geq\frac{1}{2}$. Thus
$|f(x)-L|\geq\frac{1}{10}$.

\vs

Similarly, suppose $L>\frac{1}{2}$. Pick any irrational $x$ from the
interval $0<|x-a|<\delta$. Then $|f(x)-L|=|0-L|>\frac{1}{2}$. Thus
$|f(x)-L|\geq\frac{1}{10}$.

\vs

Thus $\lim_{x\to a}f(x)$ does not exist for any $a$, as desired.

\subsection{One-sided limits}
TODO

\subsection{Infinite limits}
TODO

%%% Local Variables:
%%% TeX-master: "notes"
%%% End:
