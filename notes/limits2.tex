\section{Limits, Part II (Edge Cases)}

\subsection{Absence of limits}

What does it mean to say $L$ is not a limit of $f(x)$ at $a$? It flows
out of the definition-- there exist some $\epsilon$ such that for any
$\delta$ there exists an $x$ in $0<|x-a|<\delta$ such that $|f(x)-L|\geq\epsilon$.

\vs

A stronger version is to say there is no limit of $f(x)$ at $a$. To do
that we must prove that \textit{any} $L$ is not a limit of $f(x)$ at
$a$.

\subsubsection*{Example: Absolute value fraction}

Consider $f(x)=\frac{x}{|x|}$. It's easy to see that

\[f(x)=\begin{cases}
    -1 & \text{if } x<0\\
    1 & \text{if } x>0\\
\end{cases}\]

We will show there is no limit of $f(x)$ near $0$.

\paragraph{Weak version.}

First, let's prove a weak version-- that $\lim_{x\to 0}f(x)\neq 0$. That is
easy. Pick some reasonably small epsilon, say $\epsilon=\frac{1}{10}$. We
must show that for any $\delta$ there exists an $x$ in
$0<|x-a|<\delta$ such that $|f(x)-0|\geq \frac{1}{10}$.

\vs

Let's pick some arbitrary $x$ out of our permitted interval, say $x=\delta/2$. Then
\[|f(x)-0|=|f(\delta/2)|=\left|\frac{\delta/2}{|\delta/2|}\right|=1\geq\frac{1}{10}\]


\paragraph{Strong version.}

Now we prove that $\lim_{x\to 0}f(x)\neq L$ for \textit{any} $L$. Sticking
with $\epsilon=\frac{1}{10}$ we proceed as follows.

\vs

If $L<0$ take $x=\delta/2$. Then

\[|f(x)-L|=|f(\delta/2)-L|=\left|\frac{\delta/2}{|\delta/2|}-L\right|=|1-L|>\frac{1}{10}\]

Similarly if $L\geq 0$ take $x=-\delta/2$. Then

\[|f(x)-L|=|f(-\delta/2)-L|=\left|\frac{-\delta/2}{|-\delta/2|}-L\right|=|-1-L|>\frac{1}{10}\]

\subsubsection*{Example: Dirichlet function} \label{subsubsec:dirichlet}
The \textit{dirichlet} function $f$ is defined as follows:
\[
f(x) = 
\begin{cases} 
1 & \text{for rational } x,\\
0 & \text{for irrational } x.
\end{cases}
\]

We prove $\lim_{x\to a}f(x)$ does not exist for any $a$.

\paragraph{Proof.} Let $\epsilon=\frac{1}{10}$. Suppose for contradiction
there exists $L$ such that $\lim_{x\to a}f(x)=L$. There are two
possibilities: either $L\leq\frac{1}{2}$ or $L>\frac{1}{2}$.

\vs

First suppose $L\leq\frac{1}{2}$. Pick any rational $x$ from the interval
$0<|x-a|<\delta$. Then $|f(x)-L|=|1-L|\geq\frac{1}{2}$. Thus
$|f(x)-L|\geq\frac{1}{10}$.

\vs

Similarly, suppose $L>\frac{1}{2}$. Pick any irrational $x$ from the
interval $0<|x-a|<\delta$. Then $|f(x)-L|=|0-L|>\frac{1}{2}$. Thus
$|f(x)-L|\geq\frac{1}{10}$.

\vs

Thus $\lim_{x\to a}f(x)$ does not exist for any $a$, as desired.

\subsection{One-sided limits}
We have seen that the following function has no limit approaching $0$:
\[
f(x) = 
\begin{cases} 
-1 & x<0,\\
1 & x>0
\end{cases}
\]

However, $f$ has properties around $0$ we may want to be able to
formally describe. First, intuitively $f$ approaches $-1$ as we
approach zero from the left (from ``below''). Not surprisingly, a
notation for this exists:
\[\lim_{x\to0^-}f(x)=-1\]

If we take $l=-1$, this notation compiles down to the following
definition. For every $\epsilon>0$ there exists $\delta>0$ such that
$0<a-x<\delta$ implies $|f(x)-l|<\epsilon$ for all $x$. This is our usual limit
definition, except instead of looking at both sides of $a$ we say
$x<a$ (i.e. we look from left of $a$).

\vs

Second, intuitively $f$ approaches $11$ as we approach zero from the
right (from ``above''). The notation for this is:
\[\lim_{x\to0^+}f(x)=1\]

If we take $l=1$, the definition is as follows. For every $\epsilon>0$ there
exists $\delta>0$ such that $0<x-a<\delta$ implies $|f(x)-l|<\epsilon$ for all
$x$. Again, this is our usual limit definition, except instead of
looking at both sides of $a$ we say $x>a$ (i.e. we look from right of
$a$).

\subsection{Limits at infinity}
Consider the function $f(x)=\frac{1}{x}$. Clearly as $x$ gets very
large, $f(x)$ trends toward zero. Again, we have a notation that
encodes this property of $f$:
\[\lim_{x\to\infty}\frac{1}{x}=0\]

Take $l=0$, and this compiles down to the following definition. For
every $\epsilon>0$ there is a number $N$ such that $|f(x)-l|<\epsilon$ for all
$x>N$.

\vs

Intuitively, for any $\epsilon$, $f(x)$ will get within $\epsilon$ of the limit for
$x$ large enough. Here we simply produce a large enough $N$ instead of
$\delta$.

\subsection{Infinite limits}
Consider the function $f(x)=\frac{1}{x^2}$. Near zero $f$ shoots up,
and again, we want to be able to encode that. The notation for this
property is
\[\lim_{x\to0}f(x)=\infty\]

This compiles down to the following definition. Given any $M>0$ there
exists $\delta>0$ such that $0<|x-a|<\delta$ implies $f(x)>M$ for all
$x$. Intuitively, given an arbitrarily large $f(x)=M$ we can produce a
bound on the $x$-axis, within which $f(x)$ is never smaller than $M$.

\vs

Suppose we want to prove $\lim_{x\to0}\frac{1}{x^{2}}=\infty$. Let
$M>0$ be given. We must produce $\delta>0$ such that $0<|x|<\delta$ implies
$\frac{1}{x^2}>M$ for all $x$. Suppose we fix
$|x|<\frac{1}{\sqrt{M}}$. Then:
\begin{align*}
  &|x|<\frac{1}{\sqrt{M}}&&\text{note $M>0$}\\
  &\implies x^2<\frac{1}{M}\\
  &\implies \frac{1}{x^2}>M
\end{align*}

Thus $\delta\leq \frac{1}{\sqrt{M}}$ implies $\frac{1}{x^2}>M$ as desired.

%%% Local Variables:
%%% TeX-master: "notes"
%%% End:
