
\section{Preface}

I'm working through Spivak Calculus. Around the chapter on
epsilon-delta limits the details get pretty confusing. I started
supplementing with David Galvin's notes, which are often more clear
but are still confusing. This is surprising because the topic of
limits doesn't use anything beyond basic middle school math. Feels
like it should be simple! And so I started writing these notes to
properly understand the damned thing.

\vs

Some departures from the structure of Spivak's text:
\begin{itemize}
\item I start with limits here.
\item The very first chapter in these notes covers prerequisites
  necessary to study limits-- some really basic limits intuitions, and
  material on bounding values with inequalities.
\item In general each chapter in Spivak weaves between introducing
  concepts, exploring degenerate cases, showing examples of practice
  problems, and proving theorems. In my view this is delightful if you
  already understand the material, but distracting if you're trying to
  understand it for the first time. So instead I separate these
  categories into clear sections. I introduce concepts and proofs as
  quickly as possible (i.e. ``the blessed path''), then have a
  separate section on edge cases, etc. I tend to skip and backtrack a
  lot through Spivak's material. The order of these notes reflects the
  order in which I internalized Spivak's text.
\item This sometimes happens not only within a chapter, but also
  across chapters. Chapters 7 (Three Hard Theorems) and 8 (Least Upper
  Bounds) are swapped in these notes. Spivak first introduces the
  Intermediate Value theorem and the Extreme Value theorem as facts,
  then proves their consequences, then introduces completeness and its
  consequences, and finally proves IVT and EVT. I find it distracting
  and confusing. I introduce completeness and its consequences first.
  I then introduce and prove IVT and EVT, and finally cover their
  consequences. IMO this approach is much less confusing than
  Spivak's.
\item Spivak covers variations of trigonometric functions as he goes
  through the book. In the early chapters I found it distracting as I
  didn't know any trig. I eventually buckled down and learned enough,
  and then revisited everything I skipped. I go through this exercise
  in chapter 8 of these notes.
\item I break up derivatives into four chapters instead of three. The
  additional chapter is on the Leibniz notation. The issues of
  notation are sufficiently confusing that I found it difficult to
  study the concept of derivatives and two notational systems at the
  same time. Also, Leibniz notation requires considerable practice to
  internalize. So it gets its own chapter.
\end{itemize}


%%% Local Variables:
%%% TeX-master: "notes"
%%% End:
