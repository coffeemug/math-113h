
\section{Derivatives}

\subsection{Formal definitions}

\textbf{Definition:} the \textit{derivative} at $a$ of a function $f$,
denoted $f'(a)$, is defined as:
\[f'(a)=\lim_{h\to0}\frac{f(a+h)-f(a)}{h}\]

\textit{Intuitively}, draw a line through points $(a, f(a))$ and
$(a+h, f(a+h))$ for some small $h$. Then make $h$ ``infinitely small''.
Our $f'(a)$ is the slope of that line.

\vs

\textbf{Definition:} $f$ is called \textit{differentiable} at $a$ if
the limit $f'(a)$ exists.

\vs

The notation $f'(a)$ suggests $f'$ is a function. Indeed, we define
$f'$ as follows. Its domain is the set of all numbers $a$ where $f$ is
differentiable, and its value at such a point $a$ is the limit above.
Not surprisingly, we call $f'$ the \textit{derivative} of $f$. Note
that the domain of $f'$ could be much smaller than the domain of $f$.

\vs

We can apply the definition of the derivative to $f'$ yielding the
\textit{second derivative} $(f')'$, denoted $f''$ or $f^{(2)}$. The
domain of $f''$ is all points $a$ such that $f'$ is differentiable at
$a$. If $f''(a)$ exists, we say $f$ is \textit{twice differentiable}
at $a.$

\subsection{Leibniz notation}

The notation $f'$ is called Lagrange's notation.\footnote{Wikipedia
  claims the notation was invented by Euler and Lagrange only
  popularized it.} It's supposed to be modern, and Spivak's book
standardizes on it. Another convenient notation commonly in use is the
older (but often convenient, instructive, but also confusing) Leibniz
notation.

\vs

Leibniz didn't know about limits, and thought the derivative is the
value of the quotient $\frac{f(a+h)-f(a)}{h}$ when $h$ is
``infinitesimally small''. He denoted this infinitesimally small
quantity by $dx$, and the corresponding difference $f(x+dx)-f(x)$ by
$df(x)$. Thus for a given function $f$ the Leibniz notation for its
derivative $f'$ is:
\[\frac{df(x)}{dx}=f'\]

Intuitively, we can think of $d$ in a historical context as ``delta''
or ``change''. Then we can interpret this notation as Leibniz did-- a
quotient of a tiny change in $f(x)$ and a tiny change in $x$. But
formally we treat $\frac{df(x)}{dx}$ as a symbol denoting $f'$,
\textit{not} as a quotient of numbers.

\vs

Leibniz notation for the second derivative is
\[\frac{d\left(\frac{df(x)}{dx}\right)}{dx},\ \ \ \text{abbreviated
    to}\ \ \ \frac{d^2f(x)}{(dx)^2}, \ \ \ \text{or more often to}\ \
  \ \frac{d^2f(x)}{dx^2}.\]

Finally, there are two more ambiguities associated with the Leibniz
notation. First, $\frac{df(x)}{dx}$ is frequently abbreviated to
$\frac{df}{dx}$. Second, $\frac{df(x)}{dx}$ sometimes means the
function $f'$, and sometimes means the value $f'(x)$. The meaning of
the symbol often must be deteremined from the specific context.

\subsection{Differentiation}
  
\subsection{Examples of non-differentiable functions}
  
%%% Local Variables:
%%% TeX-master: "notes"
%%% End:
