
\section{Derivatives, Part IIb (Leibniz notation)}

The notation $f'$ that we've used so far is called the Lagrange
notation.\footnote{Wikipedia claims the notation was invented by Euler
  and Lagrange only popularized it.} However, there is another
notation for the derivative in common use. You may have already seen
something like $\frac{dy}{dx}$. This is called the Leibniz notation.

\vs

The Leibniz notation has many of what Spivak calls ``vagaries''. It
has multiple interpretations-- formal and informal. The informal
interpretation doesn't map to modern mathematics, but can
\textit{sometimes} be useful (while at other times misleading). The
full, unambigous Leibniz notation, at least as Spivak defines it, is
verbose, so in practice people end up taking liberties with it. As a
consequence, its meaning must often be discerned from the context.

\vs

This flexibility makes the notation very useful in science and
engineering, but also makes it difficult to learn. Spivak chose to
standardize on the Lagrange notation to maximize clarity, and banished
Leibniz notation to problem sections. But since the Leibniz notation
is so common, I take a different approach and explore it here in a
dedicated chapter.

\subsection{Historical motivation}

We start with the historical interpretation, where the notation began.
Leibniz didn't know about limits. He thought the derivative is the
value of the quotient
\[\frac{f(x+h)-f(x)}{h}\]

when $h$ is ``infinitesimally small''. He denoted this infinitesimally
small quantity of $h$ by $dx$, and the corresponding difference
$f(x+dx)-f(x)$ by $df(x)$. Thus for a given function $f$ the Leibniz
notation for its derivative $f'$ is:
\[\frac{df(x)}{dx}=f'\]

Intuitively, we can think of $d$ in a historical context as ``delta''
or ``change''. Then we can interpret this notation as Leibniz did-- a
quotient of a tiny change in $f(x)$ and a tiny change in $x$. However,
this explanation comes with two important disclaimers.

\vs

\textit{First}, $d$ is not a value. If it were a value, you could
cancel out $d$'s in the numerator and the denomenator. But you can't.
Instead think of $d$ as an operator. When applied to $f(x)$ or $x$, it
produces an infinitesimally small quantity. Alternatively you can
think of $df(x)$ and $dx$ as one symbol that happens to look like
multiplication, but isn't.\footnote{I read somewhere that in his
  notebooks Leibniz experimented with extending $d$ with a squiggle on
  top that went over $x$ to indicate that $d$ is not a value, but I
  haven't been able to verify if that's true.}

\vs

\textit{Second}, note that $\frac{df(x)}{dx}$ denotes a function
equivalent to $f'$, \textit{not} a value equivalent to $f'(x)$. To
denote the image of the derivative function at $a$ we use the
following notation:
\[\left. \frac{d f(x)}{dx} \right|_{x=a}=\lim_{h\to0}\frac{f(a+h)-f(a)}{h}=f'(a)\]

\subsection{Modern interpretation}

To summarize, the \textbf{full and unambiguous Leibniz notation} in
modern interpretation is:
\[\frac{df(x)}{dx}=f' \qquad\text{and}\qquad \left. \frac{d f(x)}{dx} \right|_{x=a}=f'(a)\]

Real numbers do not have a notion of infinitesimally small quantities.
Thus in a modern interpretation we treat $\frac{df(x)}{dx}$ as a
symbol denoting $f'$, \textit{not} as a quotient of numbers. Nothing
here is being divided, nothing can be canceled out. In a modern
interpretation $\frac{df(x)}{dx}$ is just one thing that
\textit{happens to look} like a quotient but isn't, anymore than $f'$
is a quotient.

\subsection{Second derivative}

A question arises for how to express the second (or nth) derivative in
the Leibniz notation. Let $g(x)=\frac{df(x)}{dx}$ (i.e. let $g$ be the
first derivative of $f$). Then it follows that the second derivative
in Leibniz notation is $\frac{dg(x)}{dx}=g'=f''$. Substituting the
definition of $g$ we get:
\[\frac{d\left(\frac{df(x)}{dx}\right)}{dx}=f''\]

Of course this is too verbose and no one wants to write it this way.
This is where the vagaries begin. For convenience people use the usual
algebraic rules to get a simpler notation, eventhough formally
everything is one symbol and you can't actually do algebra on it:
\[\frac{d\left(\frac{df(x)}{dx}\right)}{dx}=\frac{d^2f(x)}{dx^2}\]

Two questions arise here.

\vs

\textit{First}, why $dx^2$? Shouldn't it be $(dx)^2$? One way to
answer this question is to remember that $dx$ is one symbol, \textit{not} a
multiplication (because $d$ is not a value). And so we're just
squaring that one symbol $dx$, which doesn't require parentheses.

\vs

Another probably more honest way to answer this question is to recall
that this isn't real algebra-- we just use a simularcum of algebra out
of convenience. But convenience is a morally flexible thing, and
people decided to drop parentheses because they're a pain to write. So
$(dx)^2$ became $dx^2$.

\vs

\textit{Second}, we said before that $df(x)$ can be thought of as one
symbol. Then what is this $d^2$ business? The answer here is the same--
we aren't doing real algebra, but a simularcum of algebra out of
convenience. We aren't really squaring anything; we're overloading
exponentiation to mean ``second derivative''. The symbol $d^2f(x)$ is
again one symbol.

\subsection{Liberties and ambiguities}

There are a few more liberties people take with the Leibniz notation.
Let $f(x)=x^2$. If we want to denote the derivative of $f$ we can do
it in two ways:
\[\frac{df(x)}{dx} \qquad\text{or}\qquad \frac{dx^2}{dx}\]

Here $\frac{dx^2}{dx}$ is new, but the meaning should be clear. We're
just replacing $f(x)$ in $df(x)$ with the definition of $f(x)$. This
is a little confusing because in the particular case of $f(x)=x^2$,
it's visually similar to the notation for second derivative. There are
no ambiguities here so far-- it's just a visual artifact of the
notation we have to learn to ignore. But now the liberties come.

\vs

Suppose we wanted to state what the derivative of $f$ is. In Lagrange
notation we say $f'(a)=2a$. In Leibniz notation the proper way to say
it would be as follows:
\[\left.\frac{df(x)}{dx}\right|_{x=a}=2a\]

But this is obviously a pain, so people end up taking two liberties.
First, everyone drops the vertical line that denotes the application
at $a$. So in practice the form above becomes:
\[\frac{df(x)}{dx}=2x\]

This shouldn't ``compile'' because $\frac{df(x)}{dx}=f'$. Thus this
statement is equivalent to saying $f'=2x$, which doesn't make sense.
But this is the notation most people use, and you have to get used to
it.

\vs

Second, people decided that writing $\frac{df(x)}{dx}$ is too painful,
and in practice everyone writes $\frac{df}{dx}$. This also shouldn't
compile (it would be something like writing $\lim_{x\to a}f$, which also
doesn't make sense). But again, it's the notation most people use.

\vs

To summarize what we have so far:
\[\left.\frac{df(x)}{dx}\right|_{x=a}=2a \qquad\text{becomes}\qquad \frac{df}{dx}=2x\]

\subsection{Chain rule}

How do we express the chain rule $(f\circ g)'(x)=f'(g(x))\cdot g'(x)$ in
Leibniz notation? In the full and unambiguous version the chain rule
ought to look like this:
\[\frac{d f(g(x))}{dx} = \left. \frac{d f(y)}{dy} \right|_{y=g(x)} \cdot
  \frac{d g(x)}{dx}\]

But, surprise, nobody does it this way. Usually people say that if
$y=g(x)$ and $z=f(y)$ then:
\[\frac{dz}{dx}=\frac{dz}{dy}\cdot \frac{dy}{dx}\]

Let's go through some examples of using this formula, and then see
what's going on here. Let $z=\sin y, y=\cos x$. Then
\begin{align*}
  \frac{dz}{dx}&=\frac{dz}{dy}\cdot \frac{dy}{dx}\\
  &=\cos y \cdot (-\sin x)\\
  &=-\cos (\cos x) \cdot \sin x
\end{align*}

How about $z=\sin u, u=x+x^2$? Well,
\begin{align*}
  \frac{dz}{dx}&=\frac{dz}{du}\cdot \frac{du}{dx}\\
               &=\cos u \cdot (2x+1)\\
               &=\cos (x+x^2)\cdot(2x+1)
\end{align*}

How about a more complicated chain $z=\sin v, v=\cos u, u=\sin x$?
\begin{align*}
  \frac{dz}{dx}&=\frac{dz}{dv}\cdot \frac{dv}{dx}\\
               &=\frac{dz}{dv}\cdot \frac{dv}{du}\cdot \frac{du}{dx}\\
               &=\cos v \cdot (-\sin u)\cdot \cos x\\
               &=-\cos(\cos (\sin x)\cdot\sin (\sin x)\cdot \cos x
\end{align*}

Now, there are a bunch of notational liberties here:
\begin{itemize}
\item $y=\ldots$ implicitly defines a function $y(x)$ which is then used in
  e.g. $\frac{dy}{dx}$. But $y$ can also be references as a value
  (e.g. ``plot $y$ when $x$ is $\ldots$''). So the deliniation between
  functions and the values they take on is blurred.
\item $dz$ on the left side of the equations (e.g. in $\frac{dz}{dx}$)
  denotes $f\circ g$. But $dz$ on the right side of the equations (e.g. in
  $\frac{dz}{dy}$) denotes $f$. In other words, the denomenator has a
  bearing on the meaning of the numerator.
\item $\frac{dz}{dy}$ denotes the derivative function, but is also
  understood to be ``an expression involing $y$'' that must be
  substituted with the value of $y$ in the final answer. E.g. in the
  first example $\frac{dz}{dy}$ is equal to $\cos y$, and we must then
  substitute $y$ with $\cos x$.
\end{itemize}

Despite all these quirks and ambiguities, with some practice we begin
to see how easy and useful the Leibniz notation is. In the next two
sections we will refine this understanding as we deal with physical
problems involving the derivative.

\subsection{Implicit differentiation}
Suppose we have an equation for a unit circle $x^2+y^2=1$, and we want
to know $y$ changes with $x$. We will solve this problem in two ways.
First, using a ``brute force'' approach by explicitly solving for $y$
and then differentiating. Second, using a technique called
\textit{implicit differentiation} that considerably simplifies the
problem.

\subsubsection*{Brute force approach}
With the brute force approach we solve for $y$ and differentiate.
Observe that $y^2=1-x^2$, and thus

\[y=\sqrt{1-x^2}\qquad\text{and}\qquad y=-\sqrt{1-x^2}\]

\subsubsection*{Implicit differentiation approach}

\subsection{Related rates}


%%% Local Variables:
%%% TeX-master: "notes"
%%% End:
