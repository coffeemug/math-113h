\documentclass{article}
\usepackage{../style}

\title{Notes working through Spivak's Calculus}
\author{Slava Akhmechet}

\begin{document}

\maketitle

I'm working through Spivak Calculus. Around the chapter on
epsilon-delta limits the details get pretty confusing. I started
supplementing with David Galvin's notes, which are considerabley more
clear but are still confusing. This is surprising because the topic
doesn't use anything beyond basic middle school math. Feels like it
should be simple! And so I'm attempting to write these notes to
properly understand the damned thing.

\vs

Spivak's chapters 7 (Three Hard Theorems) and 8 (Least Upper Bounds)
are swapped in these notes. Spivak first introduces the Intermediate
Value theorem and the Extreme Value theorem as facts, proves their
consequences, then introduces completeness and its consequences, and
finally proves IVT and EVT. I find it distracting and confusing. I
introduce completeness and its consequences first. I then introduce
and prove IVT and EVT, and cover their consequences. IMO this approach
is much less confusing than Spivak's.

\tableofcontents

\section{Limits}

\subsection{A hand-wavy limits definition}

I'll first do a hand-wavy definition of limits, and then use it to
explain the mechanics of computing limits of functions in practice.
Basically, pre-calculus stuff. After that I'll do a proper definition
and use it to prove the theorems that make the mechanics work.

\vs

\textbf{A hand-wavy definition:} a limit of $f(x)$ at $a$ is the value
$f(x)$ approaches close to (but not necessarily at) $a$.

\vs

\textbf{A slightly less hand-wavy definition:} let $f:\R\to\R$, let
$a\in\R$ be some number on the x-axis, and let $l\in\R$ be some number on
the y-axis. Then as $x$ gets closer to $a$, $f(x)$ gets closer to $l$.

\vs

The notation for this whole thing is

\[\lim_{x\to a} f(x)=l\]

So for example $\lim_{x\to5}x^2=25$ because the closer $x$ gets to
$5$, the closer $x^2$ gets to $25$ (we'll prove all this properly
soon). Now suppose you have some fancy pants function like this one:
\begin{equation}
\label{eq:1}
\lim_{x\to 0}\frac{1-\sqrt{x}}{1-x}
\end{equation}

If you plot it, it's easy to see that as $x$ approaches $0$, the whole
shebang approaches $1$. But how do you algebraically evaluate the
limit of this thing? Can you just plug $0$ into the equation? It seems
to work, but once we formally define limits, we'll have to prove
somehow that plugging $a=0$ into $x$ gives us the correct result.

\subsection{Limits evaluation mechanics}

It turns out that it does in fact work because of a few theorems that
make practical evaluation of many limits easy. I'll first state these
theorems as facts, and then go back and properly prove them once I
introduce the formal definition of limits.

\begin{enumerate}
\item \textbf{Constants}. $\lim_{x\to a}c=c$, where $c\in\R$. In other
  words if the function is a constant, e.g. $f(x)=5$, then
  $\lim_{x\to a}f(x)=5$ for any $a$.
\item \textbf{Identity}. $\lim_{x\to a}x=a$. In other words if the
  function is an identity function $f(x)=x$, then
  $\lim_{x\to 6}f(x)=6$. Meaning we simply plug $a$ into $x$.
\item \textbf{Addition}\footnote{Spivak's book uses a slightly more
    verbose definition that assumes the limits of $f$ and $g$ exist
    near $a$, see p. 103}.
  $\lim_{x\to a}(f+g)(x)=\lim_{x\to a}f(x)+\lim_{x\to a}g(x)$. For example
  $\lim_{x\to a}(x+2)=\lim_{x\to a}x+\lim_{x\to a}2=a+2$.
\item \textbf{Multiplication}.
  $\lim_{x\to a}(f\cdot g)(x)=\lim_{x\to a}f(x)\cdot \lim_{x\to a}g(x)$. For example
  $\lim_{x\to a}2x=\lim_{x\to a}2\cdot \lim_{x\to a}x=2a$.
\item \textbf{Reciprocal}.
  $\lim_{x\to a}\left(\frac{1}{f}\right)(x)=\frac{1}{\lim_{x\to a}f(x)}$
  when the denominator isn't zero. For example
  $\lim_{x\to a}\frac{1}{x}=\frac{1}{\lim_{x\to a}x}=\frac{1}{a}$ for
  $a\neq0$.
\end{enumerate}

To come back to \ref{eq:1}, these theorems tells us that

\[\lim_{x\to 0}\frac{1-\sqrt{x}}{1-x}=\frac{\lim_{x\to 0}1-(\lim_{x\to 0}x)^{\frac{1}{2}}}{\lim_{x\to 0}1-\lim_{x\to 0}x}=\frac{1-0^{\frac{1}{2}}}{1-0}=1\]

\subsubsection*{Holes}

What happens if we try to take a limit as $x\to 1$ rather than $x\to 0$?

\[\lim_{x\to 1}\frac{1-\sqrt{x}}{1-x}\]

We can't use the same trick and plug in $1$ because we get a
nonsensical result $0/0$ as the function isn't defined at $0$. If we
plot it, we clearly see the limit approaches $1/2$ at $0$, but how do
we prove this algebraically? The answer is to do some trickery to find
a way to cancel out the inconvenient term (in this case $1-\sqrt{x}$)

\[\lim_{x\to 1}\frac{1-\sqrt{x}}{1-x}=\lim_{x\to 1}\frac{1-\sqrt{x}}{(1-\sqrt{x})(1+\sqrt{x})}=\lim_{x\to 1}\frac{1}{1+\sqrt{x}}=\frac{1}{2}\]

\vs

Why is it ok here to divide by $1-\sqrt{x}$? Good question! Recall
that the limit is defined \textit{close to} $a$ (or \textit{around}
$a$, or as $x$ \textit{approaches} $a$), but not \textbf{at} $a$. In
other words $f(a)$ need not even be defined (as is the case here).
This means that as we consider $1-\sqrt{x}$ at different values of $x$
as it approaches $a$, the limit never requires us to evaluate the
function at $x=a$. So we never have to consider $1-\sqrt{x}$ as $x=1$,
$1-\sqrt{x}$ never takes on the value of $0$, and it is safe to divide
it out.

\subsection{Formal limits definition}

Now we establish a rigorous definition of limits that formalizes the
hand-wavy version above.

\paragraph{Definition:} $\lim_{x\to a}f(x)=L$ when for any
$\epsilon\in\R$ there exists $\delta\in\R$ such that for all $x$,
$0<|x-a|<\delta$ implies $|f(x)-L|<\epsilon$. (Also $\epsilon>0, \delta>0$.)

\vs

Here is what this says. Suppose $\lim_{x\to a}f(x)=L$. You pick any
interval on the y-axis around $L$. Make it as small (or as large) as
you want. I'll produce an interval on the x-axis around $a$. You can
take any number from my interval, plug it into $f$, and the output
will stay within the bounds you specified.

\vs

So $\epsilon$ specifies the distance away from $L$ along the y-axis, and
$\delta$ specifies the distance away from $a$ along the x-axis. Take any
$x$ within $\delta$ of $a$, plug it into $f$, and the result is guaranteed
to be within $\epsilon$ of $L$. $\lim_{x\to a}f(x)=L$ just means there exists
such $\delta$ for any $\epsilon$.

\vs

This is quite simple, but the mechanics of the limit definition tend
to confuse people. I think it's because absolute value inequalities
are unfamiliar. Wtf is $0<|x-a|<\delta$ and $|f(x)-L|<\epsilon$?! Let's tease it
apart\footnote{The best answer is to go to Khan academy and do a bunch
  of absolute value inequalities until they become second nature.}

\vs

Here is the intuitive reading. $0<|x-a|<\delta$ means the difference
between $x$ and $a$ is between $0$ and $\delta$. And
$|f(x)-L|<\epsilon$ means the difference between $f(x)$ and $L$ is less than
$\epsilon$. This is actually all this means, but still, let's look at the
inequalities more closely.

\vs

First, consider $0<|x-a|<\delta$. There are two inequalities here. The left
side, $0<|x-a|$ is equivalent to $|x-a|>0$. But $|x-a|$ is an absolute
value, it's \textbf{always} true that $|x-a|\geq 0$. So this part of the
inequality says $x-a\neq 0$, or $x\neq a$. (Remember, we said the limit is
defined \textit{around} $a$ but not \textit{at} $a$). I don't know why
mathematicians say $0<|x-a|$ instead of $x\neq a$, probably because
confusing you brings them pleasure.

\vs

The right side is $|x-a|<\delta$. Intuitively this says that along the
x-axis the difference between $x$ and $a$ should be less than
$\delta$. Put differently, $x$ should be within $\delta$ of $a$. We can rewrite
this as $a-\delta<x<a+\delta$.

\vs

The second equation, $|f(x)-L|<\epsilon$ should now be easy to understand.
Intuitively, along the y-axis $f(x)$ should be within $\epsilon$ of
$L$, or put differently $L-\epsilon<f(x)<L+\epsilon$.

\subsubsection*{Limit uniqueness}

Suppose $\lim_{x\to a}f(x)=L$. It's easy to assume $L$ is the only limit
around $a$, but such a thing needs to be proved. We prove this here.
More formally, suppose $\lim_{x\to a}f(x)=L$ and
$\lim_{x\to a}f(x)=M$. We prove that $L=M$.

\vs

Suppose for contradiction $L\neq M$. Assume without loss of generality
$L>M$. By limit definition, for all $\epsilon>0$ there exists a positive
$\delta\in\R$ such that $0<|x-a|<\delta$ implies

\begin{itemize}
\item $|f(x)-L|<\epsilon\implies L-\epsilon<f(x)$
\item $|f(x)-M|<\epsilon\implies f(x)<M+\epsilon$
\end{itemize}
    
for all $x$. Thus

\begin{align*}
    &L-\epsilon<f(x)<M+\epsilon\\
    &\implies L-\epsilon<M+\epsilon\\
    &\implies L-M<2\epsilon\\
\end{align*}

The above is true for all $\epsilon$. Now let's narrow our attention and
consider a concrete $\epsilon=(L-M)/4$, which we easily find leads to a
contradiction\footnote{note we assumed $L>M$, thus $\epsilon=(L-M)/4>0$}:

\begin{align*}
    &L-M<2\epsilon\\
    &\implies (L-M)/4<\epsilon/2&&\text{dividing both sides by 4}\\
    &\implies \epsilon<\epsilon/2&&\text{recall we set $\epsilon=(L-M)/4$}
\end{align*}

We have a contradiction, and so $L=M$ as desired.

\subsubsection*{Half-Value Continuity Lemma} \label{subsubsec:half-value-lemma}

This lemma will come in handy later, so we may as well prove it now.
Suppose $M\neq0$ and $\lim_{x\to a}g(x)=M$. We show that there exists some
$\delta$ such that $0<|x-a|<\delta$ implies $|g(x)|\geq|M|/2$ for all $x$.

\vs

Intuitively, the lemma states the following: when a function $g$
approaches a nonzero limit $M$ near a point, there exists an interval
in which the values of $g$ are closer to $M$ than to zero.

\paragraph{Proof.} The claim that $|g(x)|\geq|M|/2$ is equivalent to
\[g(x)\leq-|M|/2 \text{\ \ \ or\ \ \ }g(x)\geq|M|/2\]

There are two possibilities: either $M>0$ or $M<0$. Let's consider
each possibility separately.

\vs

\textbf{Case 1}. Suppose $M>0$. Then to show $|g(x)|\geq|M|/2$ it is
sufficient to show \textit{either} $g(x)\leq-M/2$ or $g(x)\geq M/2$. We will
show $g(x)\geq M/2$. Fix $\epsilon=M/2$. By limit definition there is some
$\delta$ such that $0<|x-a|<\delta$ implies for all $x$
\begin{align*}
    &|g(x)-M|<M/2\\
    &\implies -M/2<g(x)-M\\
    &\implies M/2<g(x)&&\text{add $M$ to both sides}\\
    &\implies g(x)>M/2&&\text{note $\geq$ is correct but not tight}
\end{align*}

\textbf{Case 2}. Suppose $M<0$. We must show either $g(x)\leq M/2$ or
$g(x)\geq -M/2$. We will show $g(x)\leq M/2$. Fix $\epsilon=-M/2$. Then
\begin{align*}
    &|g(x)-M|<-M/2\\
    &\implies g(x)-M<-M/2\\
    &\implies g(x)<M/2&&\text{add $M$ to both sides;}\\
    & &&\text{note $\leq$ is correct but not tight}
\end{align*}
QED.

\subsection{Theorems that make evaluation work}

Armed with the formal definition, we can use it to rigorously prove
the five theorems useful for evaluating limits (constants, identity,
addition, multiplication, reciprocal). Let's do that now.

\subsubsection*{Constants}
Let $f(x)=c$. We prove that $\lim_{x\to a}f(x)=c$ for all $a$.

\vs

Let $\epsilon>0$ be given. Pick any positive $\delta$. Then for all
$x$ such that $0<|x-a|<\delta$, $|f(x)-c|=|c-c|=0<\epsilon$. QED.

\vs

(Note that we can pick any positive $\delta>0$, e.g.
$1, 10, \frac{1}{10}$.)

\subsubsection*{Identity}

Let $f(x)=x$. We prove that $\lim_{x\to a}f(x)=a$ for all $a$.

\vs

Let $\epsilon>0$ be given. We need to find $\delta>0$ such that for all
$x$ in $0<|x-a|<\delta$, $|f(x)-a|=|x-a|<\epsilon$. I.e. we need to find a
$\delta$ such that $|x-a|<\delta$ implies $|x-a|<\epsilon$. This obviously works for
any $\delta\leq\epsilon$. QED.

\vs

(Note the many options for $\delta$, e.g. $\delta=\epsilon$, $\delta=\frac{\epsilon}{2}$, etc.)

\subsubsection*{Addition}

Let $f,g\in\R\to\R$. We prove that

\[\lim_{x\to a}(f+g)(x)=\lim_{x\to a}f(x)+\lim_{x\to a}g(x)\]

Let $L_f=\lim_{x\to a}f(x)$ and let $L_g=\lim_{x\to a}g(x)$. Let
$\epsilon>0$ be given. We must show there exists $\delta>0$ such that for all
$x$ bounded by $0<|x-a|<\delta$ the following inequality holds:

\begin{equation*}
|(f+g)(x)-(L_f+L_g)|<\epsilon    
\end{equation*}

I.e. we're trying to show $\lim_{x\to a}(f+g)(x)$ equals to $L_f+L_g$,
the sum of the other two limits. Let's convert the left side of this
inequality into a more convenient form:

\begin{align*}
    |(f+g)(x)-(L_f+L_g)|&=|f(x)+g(x)-(L_f+L_g)|\\
    &=|(f(x)-L_f)+(g(x)-L_g)|\\
    &\leq |(f(x)-L_f)|+|(g(x)-L_g)|&&\text{by triangle inequality}
\end{align*}

\vs

By limit definition there exist positive $\delta_f, \delta_g$ such that for all $x$

\begin{itemize}
    \item $0<|x-a|<\delta_f$ implies $|f(x)-L_f|<\epsilon/2$
    \item $0<|x-a|<\delta_g$ implies $|g(x)-L_g|<\epsilon/2$
\end{itemize}

Recall that we can make $\epsilon$ as small as we like. Here we pick deltas
for $\epsilon/2$ because it's convenient to make the equations work, as you
will see in a second. For all $x$ bounded by
$0<|x-a|<\min(\delta_f, \delta_g)$ we have

\[|(f(x)-L_f)|<\epsilon/2 \ \ \ \text{ and }\ \ \  |(g(x)-L_g)|<\epsilon/2\]

Fix $\delta=\min(\delta_f, \delta_g)$. Then for all $x$ bounded by
$0<|x-a|<\delta$ we have

\begin{align*}
    |(f+g)(x)-(L_f+L_g)|&\leq |(f(x)-L_f)|+|(g(x)-L_g)|\\
    &<\epsilon/2+\epsilon/2=\epsilon
\end{align*}

as desired.

\subsubsection*{Multiplication}

Let $f,g\in\R\to\R$. We prove that

\[\lim_{x\to a}(fg)(x)=\lim_{x\to a}f(x)\cdot\lim_{x\to a}g(x)\]

Let $L_f=\lim_{x\to a}f(x)$ and let $L_g=\lim_{x\to a}g(x)$. Let
$\epsilon>0$ be given. We must show there exists $\delta>0$ such that for all
$x$ bounded by $0<|x-a|<\delta$ the following inequality holds:

\[|(fg)(x)-(L_fL_g)|<\epsilon\]

(i.e. we're trying to show $\lim_{x\to a}(fg)(x)$ equals to $L_fL_g$,
the product of the other two limits.) Let's convert the left side of
this inequality into a more convenient form:

\begin{align*}
    |(fg)(x)-(L_fL_g)|&=|f(x)g(x)-L_fL_g|\\
    &=|f(x)g(x)-L_fg(x)+L_fg(x)-L_fL_g|\\
    &=|g(x)(f(x)-L_f)+L_f(g(x)-L_g)|\\
    &\leq|g(x)(f(x)-L_f)|+|L_f(g(x)-L_g)|&&\text{by triangle inequality}\\
    &=|g(x)||f(x)-L_f|+|L_f||g(x)-L_g|&&\text{in general } |ab|=|a||b|
\end{align*}

We now need to show there exists $\delta$ such that $0<|x-a|<\delta$ implies

\[|g(x)||f(x)-L_f|+|L_f||g(x)-L_g|<\epsilon\]

We will do that by finding $\delta$ such that

\begin{enumerate}
    \item $|g(x)||f(x)-L_f|<\epsilon/2$
    \item $|L_f||g(x)-L_g|<\epsilon/2$
\end{enumerate}

\textbf{First}, we show $|g(x)||f(x)-L_f|<\epsilon/2$.

\vs

By limit definition we can find $\delta_1$ to make $|f(x)-L_f|$ as small as
we like. But how small? To make $|g(x)||f(x)-L_f|<\epsilon/2$ we must find a
delta such that $|f(x)-L_f|<\epsilon/2g(x)$. But to do that we need to get a
bound on $g(x)$. Fortunately we know there exists $\delta_2$ such that
$|g(x)-L_g|<1$ (we pick $1$ because we must pick some bound, and $1$
is as good as any). Thus $|g(x)|<|L_g|+1$. And so, we can pick
$\delta_1$ such that $|f(x)-L_f|<\epsilon/2(|L_g|+1)$.

\vs

\textbf{Second}, we show $|L_f||g(x)-L_g|<\epsilon/2$.

\vs

That is easy. By limit definition there exists a $\delta_3$ such that
$0<|x-a|<\delta_3$ implies $|g(x)-L_g|<\epsilon/2|L_f|$ for all $x$. Actually, we
need a $\delta_3$ such that $0<|x-a|<\delta_3$ implies
$|g(x)-L_g|<\frac{\epsilon}{2(|L_f|+1)}$ for all $x$ to avoid divide by zero, and of
course that exists too.

\vs

Fix $\delta=\min(\delta_1, \delta_2, \delta_3)$. Now

\begin{align*}
    |(fg)(x)-(L_fL_g)|&\leq |g(x)||f(x)-L_f|+|L_f||g(x)-L_g|\\
    &<e/2+e/2=e
\end{align*}

as desired.

\subsubsection*{Reciprocal}

Let $\lim_{x\to a}f(x)=L$. We prove $\lim_{x\to a}\left(\frac{1}{f}\right)(x)=1/L$ when $L\neq 0$.

\vs

First we show $\frac{1}{f}$ is defined near $a$. By half-value
continuity lemma (see \ref{subsubsec:half-value-lemma}) there exists
$\delta_{1}$ such that $0<|x-a|<\delta_{1}$ implies $|f(x)|\geq |L|/2$ where
$L\neq0$. Therefore $f(x)\neq 0$ near $a$, and thus $\frac{1}{f}$ near
$a$ is defined.

\vs


Now all we must do is find a delta such that
$\left|\frac{1}{f}(x)-\frac{1}{L}\right|<\epsilon$. Let's make the equation
more convenient:

\begin{align*}
  \left|\frac{1}{f}(x)-\frac{1}{L}\right|&=\left|\frac{1}{f(x)}-\frac{1}{L}\right|\\
                                         &=\left|\frac{L-f(x)}{Lf(x)}\right|\\
                                         &=\frac{|f(x)-L|}{|L||f(x)|}\\
                                         &=\frac{|f(x)-L|}{|L|}\cdot\frac{1}{|f(x)|}
\end{align*}

Above we showed there exists $\delta_{1}$ such that
$0<|x-a|<\delta_{1}$ implies $|f(x)|\geq |L|/2$. Raising both sides to
$-1$ we get $|\frac{1}{f(x)}|\leq \frac{2}{|L|}$. Continuing the chain of
reasoning above we get

\begin{align*}
  \frac{|f(x)-L|}{|L|}\cdot\frac{1}{|f(x)|}&\leq\frac{|f(x)-L|}{|L|}\cdot\frac{2}{|L|}\\
                                       &=\frac{2}{|L|^2}|f(x)-L|
\end{align*}

(if you're confused about why this inequality works, left-multiply both sides of
$|\frac{1}{f(x)}|\leq \frac{2}{|L|}$ by $\frac{|f(x)-L|}{|L|}$.) Thus we
must find $\delta_2$ such that

\[\frac{2}{|L|^2}|f(x)-L|<\epsilon\]

That is easy. Since $\lim_{x\to a}f(x)=L$ we can make $|f(x)-L|$ as
small as we like. Dividing both sides by $\frac{2}{|L|^2}$, we must
make $|f(x)-L|<\frac{|L|^2\epsilon}{2}$. Thus we must fix
$\delta=\min(\delta_1, \delta_2)$. QED.

\subsection{Absence of limits}

What does it mean to say $L$ is not a limit of $f(x)$ at $a$? It flows
out of the definition-- there exist some $\epsilon$ such that for any
$\delta$ there exists an $x$ in $0<|x-a|<\delta$ such that $|f(x)-L|\geq\epsilon$.

\vs

A stronger version is to say there is no limit of $f(x)$ at $a$. To do
that we must prove that \textit{any} $L$ is not a limit of $f(x)$ at
$a$.

\subsubsection*{Example: Absolute value fraction}

Consider $f(x)=\frac{x}{|x|}$. It's easy to see that

\[f(x)=\begin{cases}
    -1 & \text{if } x<0\\
    1 & \text{if } x>0\\
\end{cases}\]

We will show there is no limit of $f(x)$ near $0$.

\paragraph{Weak version.}

First, let's prove a weak version-- that $\lim_{x\to 0}f(x)\neq 0$. That is
easy. Pick some reasonably small epsilon, say $\epsilon=\frac{1}{10}$. We
must show that for any $\delta$ there exists an $x$ in
$0<|x-a|<\delta$ such that $|f(x)-0|\geq \frac{1}{10}$.

\vs

Let's pick some arbitrary $x$ out of our permitted interval, say $x=\delta/2$. Then
\[|f(x)-0|=|f(\delta/2)|=\left|\frac{\delta/2}{|\delta/2|}\right|=1\geq\frac{1}{10}\]


\paragraph{Strong version.}

Now we prove that $\lim_{x\to 0}f(x)\neq L$ for \textit{any} $L$. Sticking
with $\epsilon=\frac{1}{10}$ we proceed as follows.

\vs

If $L<0$ take $x=\delta/2$. Then

\[|f(x)-L|=|f(\delta/2)-L|=\left|\frac{\delta/2}{|\delta/2|}-L\right|=|1-L|>\frac{1}{10}\]

Similarly if $L\geq 0$ take $x=-\delta/2$. Then

\[|f(x)-L|=|f(-\delta/2)-L|=\left|\frac{-\delta/2}{|-\delta/2|}-L\right|=|-1-L|>\frac{1}{10}\]

\subsubsection*{Example: Dirichlet function} \label{subsubsec:dirichlet}
The \textit{dirichlet} function $f$ is defined as follows:
\[
f(x) = 
\begin{cases} 
1 & \text{for rational } x,\\
0 & \text{for irrational } x.
\end{cases}
\]

We prove $\lim_{x\to a}f(x)$ does not exist for any $a$.

\paragraph{Proof.} Let $\epsilon=\frac{1}{10}$. Suppose for contradiction
there exists $L$ such that $\lim_{x\to a}f(x)=L$. There are two
possibilities: either $L\leq\frac{1}{2}$ or $L>\frac{1}{2}$.

\vs

First suppose $L\leq\frac{1}{2}$. Pick any rational $x$ from the interval
$0<|x-a|<\delta$. Then $|f(x)-L|=|1-L|\geq\frac{1}{2}$. Thus
$|f(x)-L|\geq\frac{1}{10}$.

\vs

Similarly, suppose $L>\frac{1}{2}$. Pick any irrational $x$ from the
interval $0<|x-a|<\delta$. Then $|f(x)-L|=|0-L|>\frac{1}{2}$. Thus
$|f(x)-L|\geq\frac{1}{10}$.

\vs

Thus $\lim_{x\to a}f(x)$ does not exist for any $a$, as desired.


\subsection{Appendix: low-level proofs}

While high level theorems allow us to easily compute complicated
limits, it's instructive to compute a few limits for complicated
functions straight from the definition. We do that here.

\subsubsection*{Aside: inequalities}

We will often need to make an inequality of the following form work out:

\[|n||m|<\epsilon\]

Here $\epsilon$ is given to us, we have complete control over the upper bound
of $|n|$, and $|m|$ can take on values outside our direct control.
Obviously we can't make the inequality work without knowing
\textit{something} about $|m|$, so we'll try to find a bound for it in
terms of other fixed values, or values we control.

\vs

For example, suppose $\lim_{x\to a}f(x)=L$ and we've discovered that
$|m|<3|a|+4$. Given that we control $|n|$, how do we bound it in terms
of $\epsilon$ and $|a|$ in such a way that the inequality $|n||m|<\epsilon$ holds?

\vs

Since we control $|n|$ and $(3|a|+4)$ is fixed, we can find $|n|$
small enough so that $|n|(3|a|+4)<\epsilon$ holds. Then certainly any
inequality whose left side is smaller, e.g. $|n|(3|a|+3)<\epsilon$, will also
hold. And since $|m|$ is always smaller than $3|a|+4$, it follows
$|n||m|<\epsilon$ will hold as well.

\vs

All we have left to do is find a bound for $|n|$ such that
$|n|(3|a|+4)<\epsilon$ holds, which is of course easy:

\[|n|<\frac{\epsilon}{3|a|+4}\]

Having bound $|n|$ in this way, we can verify that
$|n|(3|a|+4)<\epsilon$ holds by multiplying both sides of the above
inequality by $3|a|+4$.

\subsubsection*{Limits of quadratic functions}

We will prove directly from the limits definition that
$\lim_{x\to a}x^2=a^2$. Let $\epsilon>0$ be given. We must show there exists
$\delta$ such that $|x^2-a^2|<\epsilon$ for all $x$ in $0<|x-a|<\delta$.

\vs

Observe that
\[|x^2-a^2|=|(x-a)(x+a)|=|x-a||x+a|\]

Thus we must pick $\delta$ such that $|x-a||x+a|<\epsilon$. Since
$0<|x-a|<\delta$, picking $\delta$ conveniently happens to bound
$|x-a|$, letting us make it as small as we want. But to know how
small, we need to find an upper bound on $|x+a|$. We can do it as
follows.

\vs

Pick an arbitrary $\delta=1$ (we may pick any arbitrary delta, e.g. $1/10$,
$10$, etc.) Then since $|x-a|<\delta$:
\begin{align*}
    &|x-a|<1\\
    &\implies -1<x-a<1\\
    &\implies 2a-1<x+a<2a+1&&\text{add $2a$ to both sides}
\end{align*}

We now have a bound on $x+a$, but we need one on $|x+a|$. It's easy to
see $|x+a|<\max(|2a-1|, |2a+1|)$. By triangle inequality
($|a+b|\leq|a|+|b|$):
\begin{align*}
    &|2a-1|\leq|2a|+|-1|=|2a|+1\\
    &|2a+1|\leq|2a|+|1|=|2a|+1
\end{align*}

Thus $|x+a|<|2a|+1$, provided $|x-a|<1$. Coming back to our original
goal, $|x-a||x+a|<\epsilon$ when

\begin{itemize}
    \item $|x-a|<1$ and
    \item $|x-a|<\frac{\epsilon}{|2a|+1}$
\end{itemize}

Putting these together, $\delta=\min(1, \frac{\epsilon}{|2a|+1})$.

\subsubsection*{Limits of fractions}

We will prove directly from the limits definition that
$\lim_{x\to 2}\frac{3}{x}=\frac{3}{2}$. Let $\epsilon>0$ be given. We must show
there exists $\delta>0$ such that $|\frac{3}{x}-\frac{3}{2}|<\epsilon$ for all
$x$ in $0<|x-2|<\delta$.

\vs

Let's manipulate $|\frac{3}{x}-\frac{3}{2}|$ to make it more convenient:
\[\left|\frac{3}{x}-\frac{3}{2}\right|=\left|\frac{6-3x}{2x}\right|=\frac{3}{2}\frac{|x-2|}{|x|}\]

Thus we need to find $\delta$ such that

\begin{align*}
&\frac{3}{2}\frac{|x-2|}{|x|}<\epsilon\\
&\implies \frac{|x-2|}{|x|}<\frac{2\epsilon}{3}\\
\end{align*}

Conveniently $0<|x-2|<\delta$ bounds $|x-2|$. But now we need to find a
bound for $|x|$. It would be extra convenient if we could show
$|x|>1$. Then we could set $\delta=\frac{2\epsilon}{3}$ (and thus bound
$|x-2|<\frac{2\epsilon}{3}$). A denominator greater than $1$ would only make
the fraction smaller than $\frac{2\epsilon}{3}$, ensuring
$\frac{|x-2|}{|x|}<\frac{2\epsilon}{3}$ holds.

\vs

We will do exactly that. Pick an arbitrary $\delta=1$ (we may pick any
arbitrary delta, e.g. $1/10$, $10$, etc.) Then since $|x-2|<\delta$
\begin{align*}
    &|x-2|<1\\
    &\implies -1<x-2<1\\
    &\implies 1<x<3\\
    &\implies 1<|x|<3
\end{align*}

Yes!! Luckily $\delta=1$ implies $|x|>1$! Thus, provided that
$|x-2|<1$ and $|x-2|<\frac{2\epsilon}{3}$, the inequality
$|\frac{3}{x}-\frac{3}{2}|<\epsilon$ holds. Putting the two constraints
together, we get $\delta=\min(1, \frac{2\epsilon}{3})$.

\subsection{Problems}
\subsubsection*{Problem 2}
Find the following limits.

\subsubsection*{Solution}
(i)

\[\lim_{x\to 1}\frac{1-\sqrt{x}}{1-x}=\lim_{x\to 1}\frac{1-\sqrt{x}}{(1-\sqrt{x})(1+\sqrt{x})}=\lim_{x\to 1}\frac{1}{1+\sqrt{x}}=\frac{1}{2}\]

(ii)
\begin{align*}
\lim_{x\to 0}\frac{1-\sqrt{1-x^2}}{x}&=\lim_{x\to 0}\frac{(1-\sqrt{1-x^2})(1+\sqrt{1-x^2})}{x(1+\sqrt{1-x^2})}\\
&=\lim_{x\to 0}\frac{1-(1-x^2)}{x(1+\sqrt{1-x^2})}\\
&=\lim_{x\to 0}\frac{x^2}{x(1+\sqrt{1-x^2})}\\
&=\lim_{x\to 0}\frac{x}{1+\sqrt{1-x^2}}=0
\end{align*}

(iii)
\begin{align*}
    \lim_{x\to0}\frac{1-\sqrt{1-x^2}}{x^2}&=\lim_{x\to 0}\frac{(1-\sqrt{1-x^2})(1+\sqrt{1-x^2})}{x^2(1+\sqrt{1-x^2})}\\
    &=\lim_{x\to 0}\frac{1-(1-x^2)}{x^2(1+\sqrt{1-x^2})}\\
    &=\lim_{x\to 0}\frac{x^2}{x^2(1+\sqrt{1-x^2})}\\
    &=\lim_{x\to 0}\frac{1}{1+\sqrt{1-x^2}}=\frac{1}{2}
\end{align*}

\subsubsection*{Problem 3i, ii}
In each of the following cases, find a $\delta$ such that $|f(x)-l|<\epsilon$ for all $x$ satisfying $0<|x-a|<\delta$.

\subsubsection*{Solution}
(i) $f(x)=x^4; l=a^4$

\vs

Let $\epsilon>0$ be given. We must find $\delta$ such that $0<|x-a|<\delta$ implies $|x^4-a^4|<\epsilon$ for all $x$. Observe that
\[|x^4-a^4|=|(x^2+a^2)(x+a)(x-a)|=|x^2+a^2||x+a||x-a|\]

We must find a bound on $|x+a|$ and $|x^2+a^2|$. Start by arbitrarily fixing $|x-a|<1$. Then
\begin{align*}
    &-1<x-a<1\\
    &\implies 2a-1<x+a<2a+1&&\text{add $2a$ to both sides}
\end{align*}
We now have a bound on $x+a$, but we need one on $|x+a|$. It's easy to see $|x+a|<\max(|2a-1|, |2a+1|)$. By triangle inequality ($|a+b|\leq|a|+|b|$):
\begin{align*}
    &|2a-1|\leq|2a|+|-1|=|2a|+1\\
    &|2a+1|\leq|2a|+|1|=|2a|+1
\end{align*}
Thus $|x+a|<|2a|+1$, provided $|x-a|<1$. Similarly, we find a bound for $|x^2+a^2|$:
\begin{align*}
    &-1<x-a<1\\
    &\implies a-1<x<a+1\\
    &\implies (a-1)^2<x^2<(a+1)^2&&\text{square each side}\\
    &\implies (a-1)^2+a^2<x^2+a^2<(a+1)^2+a^2
\end{align*}
Observe that $x^2+a^2=|x^2+a^2|$, thus $|x^2+a^2|<(a+1)^2+a^2$. Thus to make $|x^4-a^4|<\epsilon$ we must set
\[|x-a|<\frac{\epsilon}{(|2a|+1)((a+1)^2+a^2)}\]
provided $|x-a|<1$. Therefore
\[\delta=\min(1, \frac{\epsilon}{(|2a|+1)(2a^2+2a+1)})\]

\vs

(ii) $f(x)=\frac{1}{x}; a=1, l=1$

\vs

Let $\epsilon>0$ be given. We must find $\delta$ such that $0<|x-1|<\delta$ implies $|\frac{1}{x}-1|<\epsilon$ for all $x$. Observe that
\[\left|\frac{1}{x}-1\right|=\left|\frac{1}{x}-\frac{x}{x}\right|=\left|\frac{1-x}{x}\right|=\frac{|x-1|}{|x|}\]

Fix $|x-1|<\frac{1}{10}$. Then
\begin{align*}
    &-\frac{1}{10}<x-1<\frac{1}{10}\\
    &\implies \frac{9}{10}<x<\frac{11}{10}
\end{align*}
Thus we must set
\[|x-1|<\frac{\epsilon}{10}\] provided $|x-1|<\frac{1}{10}$. Therefore
\[\delta=\min(\frac{1}{10}, \frac{\epsilon}{10})\]

\subsubsection*{Problem 8}
Answer the following.

\subsubsection*{Solution}
(a) If $\lim_{x\to a}f(x)$ and $\lim_{x\to a}g(x)$ do not exist, can $\lim_{x\to a}[f(x)+g(x)]$ or $\lim_{x\to a}f(x)g(x)$ exist?

\vs

Yes. Consider
\[\begin{array}{cc}
f(x)=\begin{cases}
    -1 & \text{if } x\leq0\\
    1 & \text{if } x>0\\
\end{cases}
&
g(x)=\begin{cases}
    1 & \text{if } x\leq0\\
    -1 & \text{if } x>0\\
\end{cases}
\end{array}\]

Then $(g+f)(x)=0$ and $(gf)(x)=-1$, both of which have limits for all $a$.

\vs

(b) If $\lim_{x\to a}f(x)$ exists and $\lim_{x\to a}[f(x)+g(x)]$ exists, must $\lim_{x\to a}g(x)$ exist?

\vs

Yes. Let $\epsilon>0$ be given. Then there exists $\delta$ such that for all $x$ in $0<|x-a|<\delta$ the following inequalities hold:
\[\begin{array}{cc}
M-\epsilon/2<f(x)+g(x)<M+\epsilon/2\text{,} & L-\epsilon/2<f(x)<L+\epsilon/2
\end{array}\]

Then:
\begin{align*}
    &M-\epsilon/2<f(x)+g(x)<M+\epsilon/2\\
    &\implies M-\epsilon/2-f(x)<g(x)<M+\epsilon/2-f(x)\\
    &\implies M-\epsilon/2-L-\epsilon/2<g(x)<M+\epsilon/2-L+\epsilon/2\\
    &(M-L)-\epsilon<g(x)<(M-L)+\epsilon\\
    &-\epsilon<g(x)-(M-L)<\epsilon\\
    &|g(x)-(M-L)|<\epsilon\\
\end{align*}
Therefore $\lim_{x\to a}g(x)=M-L$ and must exist.

\vs

(c) If $\lim_{x\to a}f(x)$ exists and $\lim_{x\to a}g(x)$ does not exist, can $\lim_{x\to a}[f(x)+g(x)]$ exist?

\vs

No. Let $\lim_{x\to a}f(x)=L$. Since $\lim_{x\to a}g(x)$ does not exist, there exists $\epsilon$ such that $|g(x)-M|\geq\epsilon$ for all $M$. Suppose for contradiction $\lim_{x\to a}[f(x)+g(x)]=M$ exists. Then
\begin{align*}
    &M-\epsilon/2<f(x)+g(x)<M+\epsilon/2\\
    &\implies M-\epsilon/2-f(x)<g(x)<M+\epsilon/2-f(x)\\
    &\implies M-\epsilon/2-L-\epsilon/2<g(x)<M+\epsilon/2-L+\epsilon/2\\
    &(M-L)-\epsilon<g(x)<(M-L)+\epsilon\\
    &-\epsilon<g(x)-(M-L)<\epsilon\\
    &|g(x)-(M-L)|<\epsilon\\
\end{align*}
We have a contradiction, thus $\lim_{x\to a}[f(x)+g(x)]$ does not exist.

\vs

(d) If $\lim_{x\to a}f(x)$ exists and $\lim_{x\to a}f(x)g(x)$ exists, does it follow that $\lim_{x\to a}g(x)$ exists?

\vs

No. Consider $g(x)=1/x$ which has no limit at $0$, and $f(x)=0$. Then $f(x)g(x)=0$ which has a limit of $0$ as $x\to 0$.

\subsubsection*{Problem 13}
Suppose that $f(x)\leq g(x)\leq h(x)$ and that $\lim_{x\to a}f(x)=\lim_{x\to a} h(x)$. Prove that $\lim_{x\to a}g(x)$ exists, and that $\lim_{x\to a}g(x)=\lim_{x\to a}f(x)=\lim_{x\to a} h(x)$. (Draw a picture!)

\subsubsection*{Solution}
Let $L=\lim_{x\to a}f(x)=\lim_{x\to a} h(x)$. Let $\epsilon>0$ be given. We must find $\delta$ such that $0<|x-a|<\delta$ implies $|g(x)-L|<\epsilon$.

\vs

By limit definition there exists $\delta_1$ such that for all $x$ in $0<|x-a|<\delta_1$
\begin{align*}
    &|f(x)-L|<\epsilon\\
    &-\epsilon<f(x)-L<\epsilon\\
    &L-\epsilon<f(x)<L+\epsilon\\
\end{align*}

Similarly there exists $\delta_2$ such that for all $x$ in $0<|x-a|<\delta_2$
\begin{align*}
    &|h(x)-L|<\epsilon\\
    &-\epsilon<h(x)-L<\epsilon\\
    &L-\epsilon<h(x)<L+\epsilon\\
\end{align*}

By problem statement $f(x)\leq g(x)\leq h(x)$. Fix $\delta=\min(\delta_1, \delta_2)$. Then
\[L-\epsilon<f(x)\leq g(x)\leq h(x)<L+\epsilon\]

Therefore $L-\epsilon<g(x)<L+\epsilon$ which implies $|g(x)-L|<\epsilon$, as desired.

\subsubsection*{Problem 15}
Evaluate the following limits in terms of the number $\alpha=\lim_{x\to0}(\sin x)/x$.

\subsubsection*{Solution}

(i)

\[\lim_{x\to 0}\frac{\sin 2x}{x}=\lim_{x\to 0}\frac{2\sin x\cos x}{x}=2 \alpha \cos x=2\alpha\]

(iv)
\begin{align*}
\lim_{x\to 0}\frac{\sin^2 2x}{x^2}&=\lim_{x\to 0}\frac{(2\sin x\cos x)^2}{x^2}\\
&=\lim_{x\to 0}\frac{4\sin^2x\cos^2x}{x^2}\\
&=\lim_{x\to 0}4\alpha^2\cos^2x\\
&=4\alpha^2
\end{align*}

(vii)

\begin{align*}
    \lim_{x\to 0}\frac{x\sin x}{1-\cos x}&=\lim_{x\to 0}\frac{x\sin x(1+\cos x)}{(1-\cos x)(1+\cos x)}\\
    &=\lim_{x\to 0}\frac{x\sin x(1+\cos x)}{\sin^2 x}\\
    &=\lim_{x\to 0}\frac{x(1+\cos x)}{\sin x}\\
    &=\lim_{x\to 0}\frac{1+\cos x}{\alpha}=\frac{2}{\alpha}
\end{align*}

(ix)

\begin{align*}
    \lim_{x\to 1}\frac{\sin(x^2-1)}{x-1}&=\lim_{x\to 1}\frac{\sin(x^2-1)(x+1)}{(x-1)(x+1)}\\
    &=\lim_{x\to 1}\frac{\sin(x^2-1)(x+1)}{x^2-1}
\end{align*}
Let $u=x^2-1$. Observe that as $x\to 1, u\to 0$. Thus
\[\lim_{x\to 1}\frac{\sin(x^2-1)(x+1)}{x^2-1}=\lim_{u\to 0}\frac{\sin u}{u}\cdot \lim_{x\to 1} x+1=2\alpha\]

\subsubsection*{Problem 19}
Prove that if $f(x)=0$ for irrational $x$ and $f(x)=1$ for rational $x$, then $\lim_{x\to a}f(x)$ does not exist for any $a$.

\subsubsection*{Solution}
Let $\epsilon=\frac{1}{10}$. We handle two cases. First suppose $L<\frac{1}{2}$. Pick any rational $x$ from the interval $0<|x-a|<\delta$. Then $|f(x)-L|=|1-L|>\frac{1}{2}$. Thus $|f(x)-L|\geq\frac{1}{10}$.

\vs

Similarly, suppose $L>\frac{1}{2}$. Pick any irrational $x$ from the interval $0<|x-a|<\delta$. Then $|f(x)-L|=|0-L|>\frac{1}{2}$. Thus $|f(x)-L|\geq\frac{1}{10}$.


%%% Local Variables:
%%% TeX-master: "notes"
%%% End:

\section{Limits, Part II (Edge Cases)}

\subsection{Absence of limits}

What does it mean to say $L$ is not a limit of $f(x)$ at $a$? It flows
out of the definition-- there exist some $\epsilon$ such that for any
$\delta$ there exists an $x$ in $0<|x-a|<\delta$ such that $|f(x)-L|\geq\epsilon$.

\vs

A stronger version is to say there is no limit of $f(x)$ at $a$. To do
that we must prove that \textit{any} $L$ is not a limit of $f(x)$ at
$a$.

\subsubsection*{Example: Absolute value fraction}

Consider $f(x)=\frac{x}{|x|}$. It's easy to see that

\[f(x)=\begin{cases}
    -1 & \text{if } x<0\\
    1 & \text{if } x>0\\
\end{cases}\]

We will show there is no limit of $f(x)$ near $0$.

\paragraph{Weak version.}

First, let's prove a weak version-- that $\lim_{x\to 0}f(x)\neq 0$. That is
easy. Pick some reasonably small epsilon, say $\epsilon=\frac{1}{10}$. We
must show that for any $\delta$ there exists an $x$ in
$0<|x-a|<\delta$ such that $|f(x)-0|\geq \frac{1}{10}$.

\vs

Let's pick some arbitrary $x$ out of our permitted interval, say $x=\delta/2$. Then
\[|f(x)-0|=|f(\delta/2)|=\left|\frac{\delta/2}{|\delta/2|}\right|=1\geq\frac{1}{10}\]


\paragraph{Strong version.}

Now we prove that $\lim_{x\to 0}f(x)\neq L$ for \textit{any} $L$. Sticking
with $\epsilon=\frac{1}{10}$ we proceed as follows.

\vs

If $L<0$ take $x=\delta/2$. Then

\[|f(x)-L|=|f(\delta/2)-L|=\left|\frac{\delta/2}{|\delta/2|}-L\right|=|1-L|>\frac{1}{10}\]

Similarly if $L\geq 0$ take $x=-\delta/2$. Then

\[|f(x)-L|=|f(-\delta/2)-L|=\left|\frac{-\delta/2}{|-\delta/2|}-L\right|=|-1-L|>\frac{1}{10}\]

\subsubsection*{Example: Dirichlet function} \label{subsubsec:dirichlet}
The \textit{dirichlet} function $f$ is defined as follows:
\[
f(x) = 
\begin{cases} 
1 & \text{for rational } x,\\
0 & \text{for irrational } x.
\end{cases}
\]

We prove $\lim_{x\to a}f(x)$ does not exist for any $a$.

\paragraph{Proof.} Let $\epsilon=\frac{1}{10}$. Suppose for contradiction
there exists $L$ such that $\lim_{x\to a}f(x)=L$. There are two
possibilities: either $L\leq\frac{1}{2}$ or $L>\frac{1}{2}$.

\vs

First suppose $L\leq\frac{1}{2}$. Pick any rational $x$ from the interval
$0<|x-a|<\delta$. Then $|f(x)-L|=|1-L|\geq\frac{1}{2}$. Thus
$|f(x)-L|\geq\frac{1}{10}$.

\vs

Similarly, suppose $L>\frac{1}{2}$. Pick any irrational $x$ from the
interval $0<|x-a|<\delta$. Then $|f(x)-L|=|0-L|>\frac{1}{2}$. Thus
$|f(x)-L|\geq\frac{1}{10}$.

\vs

Thus $\lim_{x\to a}f(x)$ does not exist for any $a$, as desired.

\subsection{One-sided limits}
TODO

\subsection{Infinite limits}
TODO

\subsection{Limits at infinity}
TODO

%%% Local Variables:
%%% TeX-master: "notes"
%%% End:

\section{Continuity, Part I (On a Point)}

\subsection{Definition of continuity}

A function $f$ is \textbf{continuous} at $a$ when

\[\lim_{x\to a}f(x)=f(a)\]

Inlining the limits definition, $f$ is continuous at $a$ if for all
$\epsilon>0$ there exists $\delta>0$ such that $0<|x-a|<\delta$ implies
$|f(x)-f(a)|<\epsilon$.

\vs

We can simplify this definition slightly. Observe that in continuous
functions $f(a)$ exists, and at $x=a$ we get $f(x)-f(a)=0$. Thus we
can relax the constraint $0<|x-a|<\delta$ to $|x-a|<\delta$.

\vs

A function $f$ is \textbf{continuous on an interval} $(a, b)$ if it's
continuous at all $c\in(a,b)$\footnote{Closed intervals are a tiny bit
  harder, and I'm keeping them out for brevity.}.

\subsubsection*{Nonzero Neighborhood Lemma} \label{subsubsec:nonzero-lemma}

Armed with these definitions we can extend the half-value neighborhood
lemma (see \ref{subsubsec:half-value-lemma}) in a useful way. The
\textit{nonzero neighborhood lemma} will come in handy when we prove
the intermediate value theorem (see \ref{ivt}), so we may as well
prove the lemma now.

\vs

Suppose $f$ is continuous at $a$, and $f(a)\neq0$. Then there exists
$\delta>0$ such that:
\begin{enumerate}
\item if $f(a)<0$ then $f(x)<0$ for all $x$ in $|x-a|<\delta$.
\item if $f(a)>0$ then $f(x)>0$ for all $x$ in $|x-a|<\delta$.
\end{enumerate}

\textit{Intuitively} the lemma states that there is some interval
around $a$ on which $f(x)\neq0$ and has the same sign as $f(a)$.

\vs

\textbf{Proof.} The proof follows trivially from the half-value
neighborhood lemma.


\subsection{Recognizing continuous functions}
The following theorems allow us to tell at a glance that large classes
of functions are continuous (e.g. polynomials, rational functions,
etc.)

\subsubsection*{Five easy proofs}

\paragraph{Constants.} Let $f(x)=c$. Then $f$ is continuous at all $a$
because
\[\lim_{x\to a}f(x)=c=f(a)\]

\paragraph{Identity.} Let $f(x)=x$. Then $f$ is continuous at all $a$
because
\[\lim_{x\to a}f(x)=a=f(a)\]

\paragraph{Addition.} Let $f,g\in\R\to\R$ be continuous at $a$. Then
$f+g$ is continuous at $a$ because
\[\lim_{x\to a}(f+g)(x)=\lim_{x\to a}f(x)+\lim_{x\to a}g(x)=f(a)+g(a)=(f+g)(a)\]

\paragraph{Multiplication.} Let $f,g\in\R\to\R$ be continuous at $a$. Then
$f\cdot g$ is continuous at $a$ because
\[\lim_{x\to a}(fg)(x)=\lim_{x\to a}f(x)\cdot\lim_{x\to a}g(x)=f(a)\cdot g(a)=(fg)(a)\]

\paragraph{Reciprocal.} Let $g$ be continuous at $a$. Then $\frac{1}{g}$
is continuous at $a$ where $g(a)\neq 0$ because
\[\lim_{x\to a}\left(\frac{1}{g}\right)(x)=\frac{1}{\lim_{x\to a}g(x)}=\frac{1}{g(a)}=\left(\frac{1}{g}\right)(a)\]

\subsubsection*{Slightly harder proof: composition}

Let $f,g\in\R\to\R$. Let $g$ be continuous at $a$, and let $f$ be
continuous at $g(a)$. Then $f\circ g$ is continuous at $a$. Put
differently, we want to show
\[\lim_{x\to a}(f\circ g)(x)=(f\circ g)(a)\]

Unpacking the definitions, let $\epsilon>0$ be given. We want to show there
exists $\delta>0$ such that $|x-a|<\delta$ implies
\begin{align*}
    |(f\circ g)&(x)-(f\circ g)(a)|\\
    &=|f(g(x))-f(g(a))|<\epsilon
\end{align*}

By problem statement we have two continuities.

\vs

\textbf{First}, $f$ is continuous at $g(a)$, i.e.
$\lim_{X\to g(a)}f(X)=f(g(a))$. Thus there exists $\delta'>0$ such that
$|X-g(a)|<\delta'$ implies $|f(X)-f(g(a))|<\epsilon$.

\vs

\textbf{Second}, $g$ is continuous at $a$, i.e.
$\lim_{x\to a}g(x)=g(a)$. Thus there exists $\delta>0$ such that
$|x-a|<\delta$ implies $|g(x)-g(a)|<\epsilon$. Since we can make
$\epsilon$ be anything, we can set it to $\delta'$.

\vs

I.e. there exists $\delta>0$ such that $|x-a|<\delta$ implies
$|g(x)-g(a)|<\delta'$. Intuitively, $g(x)$ is close to $g(a)$. But by the
first continuity, any $X$ close to $g(a)$ implies
\[|f(X)-f(g(a))|<\epsilon\]

Thus $|f(g(x))-f(g(a))|<\epsilon$, as desired.

\subsection{Example: Stars over Babylon}
Stars over Babylon is a modification of the Dirichlet function (see
\ref{subsubsec:dirichlet}), defined as follows:
\[
f(x) = 
\begin{cases} 
  0, & \text{$x$ irrational}, 0<x<1\\
  1/q, & x=p/q \text{ in lowest terms}, 0<x<1.
\end{cases}
\]

\textbf{Claim:} for $0<a<1$, $\lim_{x\to a}f(x)=0$.

\vs

\textbf{Proof.} Let $\epsilon>0$ be given. We must find $\delta>0$ such that
$0<|x-a|<\delta$ implies $|f(x)-0|<\epsilon$. For \textit{any}
$\delta>0$, $0<|x-a|<\delta$ implies one of two cases for all $x$: either
$x$ is irrational or it is rational.

\vs

If $x$ is irrational, $|f(x)-0|=0<\epsilon$.

\vs

Otherwise, if $x=p/q$ in the lowest terms is rational, $f(x)=1/q$. Let
$n\in\mathcal{N}$ such that $1/n<\epsilon$. We will look for $\delta$ such that:
\[f\left(\frac{p}{q}\right)=\frac{1}{q}<\frac{1}{n}<\epsilon\]

\vs

Observe that when $q>n$, $f(\frac{p}{q})=\frac{1}{q}<\frac{1}{n}$.
Thus the only rationals that \textit{could} result in $f(\frac{p}{q})\geq1/n$ are
ones where $q\leq n$:
\[A=\{\frac{1}{2};\ \ \frac{1}{3},\frac{2}{3};\ \
  \frac{1}{4},\frac{3}{4};\ \
  \frac{1}{5},\frac{2}{5},\frac{3}{5},\frac{4}{5},\ \ \ldots,\ \ \frac{1}{n},\ldots,\frac{n-1}{n}\}\]

This set has a finite length, and thus \textit{one} $p/q\in A$ is
closest to $a$. Fix $\delta=|a-p/q|$ (i.e. anything less than this
distance). This guarantees $0<|x-a|<\delta$ implies $x\notin A$ for all $x$, and
thus $f(x)<1/n<\epsilon$ for all $x$, as desired.

\vs

\textbf{Claim:} $f(x)$ is continuous at all irrationals, discontinuous
at all rationals.

\vs

\textbf{Proof:} we've just proven for $0<a<1$, $\lim_{x\to a}f(x)=0$. By
definition $f(x)$ is zero for all irrationals, and nonzero for all
rationals. Thus $\lim_{x\to a}f(x)=f(x)$ for all irrationals, and
$\lim_{x\to a}f(x)\neq f(x)$ for all rationals.

%%% Local Variables:
%%% TeX-master: "notes"
%%% End:

\section{Complete ordered fields}

\subsection{Motivation}
The twelve ordered field axioms are sufficient to define limits,
continuity, and prove all the theorems in the previous sections. Since
the set $\mathcal{Q}$ of rational numbers is an ordered field\footnote{The proof
  is straightforward, so I'm not including it here.}, rationals have
been sufficient for the work we've done so far. However, we are about
to start proving slightly more sophisticated theorems about continous
functions, and ordered fields will quickly start breaking our
intuitions.

\vs

For example, consider the function $f(x)=x^{2}-2$ (a parabola shifted
down two units). It's easy to see $f$ is a continuous function, and
thus our intuition is that we should be able to draw it without
``lifting the tip of the pencil off the sheet of paper''. Upon
reflection however, it becomes obvious that in the universe limited to
ordered fields this is impossible. $f$ intersects the x-axis when
$x^{2}=2$, but every high school student knows
$\sqrt{2}\notin\mathcal{Q}$ (see \ref{sqrt2proof} for proof). Thus there is no
$x\in\mathcal{Q}$ such that $f(x)=0$. And since $\mathcal{Q}$ is an ordered field, it
follows ordered fields alone aren't sufficient to resolve this
problem.

\vs

The \textit{intermediate value theorem} (see \ref{ivt}) formalizes the
claim that a continuous function segment that starts below the x-axis
and ends above the x-axis intersects the x-axis. But as we can see
from the example above, this is not possible to prove with ordered
field axioms alone. So before we proceed with further study of
continuity, we need one more axiom called \textit{the completeness
  axiom}, which we introduce in this chapter.

\vs

Combined with the twelve ordered field axioms, the completeness axiom
forms \textit{complete ordered fields}. These objects are sufficient
to proceed with our study of calculus. We will see that rational
numbers $\mathcal{Q}$ are not a complete ordered field, whereas real numbers
$\mathcal{R}$ are.\footnote{Proof that $\mathcal{R}$ is a complete ordered field requires
  construction of $\mathcal{R}$, which doesn't happen in Spivak until the last
  chapters. Thus I will not be delving into that here and ask the
  reader (i.e., currently myself) to take this on faith.} Thus from
here $\mathcal{R}$-valued functions will become our primary object of study.

\subsection{Least upper bound}
\textbf{Definition:} $b$ is an \textbf{upper bound} for $S$ if
$s\leq b$ for all $s\in S$.

\vs

For example:
\begin{itemize}
\item Any $b\geq1$ is an upper bound for $S=\{x:0\leq x<1\}$. E.g. $1, 2,
  10$ are all upper bounds of $S$.
\item By convention, \textit{every} number is an upper bound for $\emptyset$.
\item The set $\mathcal{N}$ of natural numbers has no natural upper bound. The
  proof is easy. Suppose $b\in\mathcal{N}$ is an upper bound for
  $\mathcal{N}$. But $b+1\in\mathcal{N}$, and $b+1>b$, which is a contradiction. Thus
  $b$ isn't an upper bound for $\mathcal{N}$.\footnote{We need to do a little
    more work to show $\mathcal{N}$ has no upper bound, natural or not. Be
    patient! We will prove this by the end of the section.}
\end{itemize}

\vs

\textbf{Definition:} $x$ is a \textbf{least upper bound} of $A$, if
\begin{enumerate}
\item $x$ is an upper bound of $A$,
\item \textit{and} if $y$ is an upper bound of $A$, then $x\leq y$.
\end{enumerate}

A set can have only one least upper bound. The proof is easy. Suppose
$x$ and $x'$ are both least upper bounds of $S$. Then $x\leq x'$ and
$x'\leq x$. Thus $x=x'$. Consequently, we can use a convenient notation
$\sup A$ to denote the least upper bound of $A$.

\vs

Obligatory examples:
\begin{itemize}
\item Let $S=\{x:0\leq x<1\}$. Then $\sup S=1$.
\item By convention, the empty set $\emptyset$ has no least upper bound.
\end{itemize}

\subsection{Completeness axiom}
We are now ready to state the completeness axiom.

\vs

\textbf{Completeness [P13]:} If $A$ is a non-empty set of numbers that
has an upper bound, then it has a least upper bound.

\vs

\textbf{Claim:} rational numbers are not complete.

\textbf{Proof:} Let $C=\{x:x^{2}<2\text{ and }x\in\mathcal{Q}\}$. Suppose for
contradiction rational numbers are complete. Then there exists
$b\in\mathcal{Q}$ such that $b=\sup C$. Observe that
\begin{itemize}
\item $b^{2}\neq2$ as that would imply $b=\sqrt{2}$ and thus $b\notin\mathcal{Q}$.
\item $b^{2}\not<2$ as there would exist some $x\in C$ such that
  $b^{2}<x^{2}<2$. Thus $b<x$ and $b$ is not the upper bound.
\end{itemize}

Therefore $b^{2}>2$. But this implies there exists some
$x\in\mathcal{Q}$ such that $2<x^{2}<b^{2}$. Thus $x$ is greater than every
element in $C$, and $x<b$. So $b$ is not the \textit{least} upper
bound. We have a contradiction, therefore rational numbers are not
complete, as desired.

\vs

\textbf{Claim:} completeness cannot be derived from ordered fields.

\textbf{Proof:} $\mathcal{Q}$ is not complete and $\mathcal{Q}$ is an ordered field. Thus
completeness is not a property of ordered fields.

\vs

\textbf{Claim:} real numbers are complete.

\textbf{Proof [deferred]:} The completeness property can be derived
from the construction of real numbers $\mathcal{R}$, which makes reals a
\textbf{complete ordered field}. The proof requires we study the
actual construction of $\mathcal{R}$, which Spivak leaves until the last
chapters. Thus for the moment the proof will be taken on faith. In any
case, it is better to build calculus upon abstract complete ordered
fields than upon concrete real numbers.

\subsection{Consequences of completeness}

\subsubsection*{$\mathcal{N}$ is not bounded above}
We've shown $\mathcal{N}$ has no upper bound in $\mathcal{N}$. Now we
show $\mathcal{N}$ has no upper bound in $\mathcal{R}$.

\vs

Suppose for contradiction $\mathcal{N}$ has an upper bound. Since $\mathcal{N}\neq\emptyset$ then by
completeness $\mathcal{N}$ has a least upper bound. Let $\alpha=\sup \mathcal{N}$. Then:
\begin{align*}
  &\alpha\geq n \text{ for all } n\in\mathcal{N}\\
  \implies &\alpha\geq n+1 \text{ for all } n\in\mathcal{N}&&\text{since $n+1\in\mathcal{N}$ if $n\in\mathcal{N}$}\\
  \implies &\alpha-1\geq n \text{ for all } n\in\mathcal{N}
\end{align*}
Thus $\alpha-1$ is \textit{also} an upper bound for $\mathcal{N}$. This contradicts
that $\alpha=\sup \mathcal{N}$. Therefore $\mathcal{N}$ is not bounded above, as desired.

\vs

\subsubsection*{$\sqrt{2}$ exists}

We show $\sqrt{2}\in\mathcal{R}$. Let
$S=\{y\in\mathcal{R} : y^{2}<2\}$. Obviously $S$ is non-empty and has an upper
bound. Thus by completeness property it has a least upper bound. Let
$x=\sup S$. Note that $1\in S$ and $2$ is an upper bound of $S$. Thus
$1\leq x\leq2$. We show $x^{2}=2$ by showing $x^{2}\not<2$ and
$x^{2}\not>2$.

\vs

\textit{Case 1}. Suppose for contradiction $x^{2}<2$. Let $0<\epsilon<1$ be a
small number. Then
\begin{align*}
  {(x+\epsilon)}^{2}=&x^{2}+2\epsilon x+\epsilon^{2}\\
              &\leq x^{2}+4\epsilon+\epsilon&&\text{since $x<2$ and $\epsilon<1$}\\
              &=x^{2}+5\epsilon<2&&\text{since $x^{2}<2$ (by supposition), we
                             can pick}\\
              &&&\text{a small enough $\epsilon$ to make this true}
\end{align*}
Thus there exists $\epsilon$ such that ${(x+\epsilon)}^{2}<2$. By definition of
$S$ it follows $x+\epsilon\in S$, which contradics that $x$ is the least upper
bound. Therefore $x^{2}\not<2$

\vs

\textit{Case 2}. Suppose for contradiction $x^{2}>2$. Let $0<\epsilon<1$ be a
small number. Then
\begin{align*}
  {(x-\epsilon)}^{2}=&x^{2}-2\epsilon x+\epsilon^{2}\\
              &\geq x^{2}-2\epsilon x&&\text{since $\epsilon^{2}>0$}\\
              &\geq x^{2}-4\epsilon&&\text{since $x\leq2$}\\
              &>2&&\text{since $x^{2}>2$ (by supposition), we
                             can pick}\\
              &&&\text{a small enough $\epsilon$ to make this true}
\end{align*}

Thus ${(x-\epsilon)}^{2}>2$, which by definition of $S$ implies
$x-\epsilon>y$ for all $y\in S$. So $x-\epsilon$ is an upper bound of
$S$. We have a contradiction-- since $x-\epsilon<x$, it follows $x$ is not a
least upper bound. Therefore $x^{2}\not>2$ as desired.

\vs

Since $x^{2}\not<2$ and $x^{2}\not>2$, it follows $x^{2}=2$ as
desired.


\subsubsection*{Archimedean property}
Handwavy definition: the Archimedean property states that you can fill
the universe with tiny grains of sand.

\vs

Formal defition: let $\epsilon>0$ be small and let $r>0$ be large. Then there
exists $n\in\mathcal{N}$ such that $n\epsilon>r$.

\vs

\textbf{Proof:} suppose for contradiction the property is false. Then
there exist $\epsilon, r$ such that for all $n\in\mathcal{N}$, $n\epsilon\leq r$. Therefore
$n\leq\frac{r}{\epsilon}$. This implies $\mathcal{N}$ is bounded, which a contradiction.

\vs

A useful special case is when $r=1$. In this case the Archimedean
property can be restated as follows. Let $\epsilon>0$ be small. Then there
exists $n\in\mathcal{N}$ such that $n\epsilon>1$. Put differently, there exists
$n\in\mathcal{N}$ such that $\frac{1}{n}<\epsilon$.

\vs

A few more notes on the Archimedean property:
\begin{itemize}
\item Obviously the Archimedean property follows from completeness, as
  shown above.
\item The Archimedean property is true in $\mathcal{Q}$ and can be proven
  without being assumed\footnote{Excluding the proof here, but it's
    fairly simple}.
\item Completeness does not follow from the Archimedean property. The
  proof is easy: the Archimedean property holds on $\mathcal{Q}$, and we know
  $\mathcal{Q}$ is not complete as shown above.
\end{itemize}


\subsubsection*{Density}
Let $x,y\in\mathcal{R}$. Then $S$ is a \textbf{dense subset} of
$\mathcal{R}$ if there is an element of $S$ in $(x,y)$. Put differently, there
is an element of $S$ between any two points in $\mathcal{R}$.
\begin{itemize}
\item Obviously $\mathcal{R}$ is a dense subset of itself.
\item Integers are not a dense subset of $\mathcal{R}$. E.g. there is no integer
  between $1.1$ and $1.9$.
\item The set of positive numbers $\{x:x\in\mathcal{R}, x>0\}$ is not a dense
  subset of $\mathcal{R}$. E.g. there is no positive number between
  $-2$ and $-1$.
\end{itemize}

\textbf{Claim:} the set of rational numbers $\mathcal{Q}$ is dense.

\textbf{Proof:} let $x,y\in\mathcal{R}$ be given. Suppose we can show there exists
a rational in $(x, y)$ for $0\leq x<y$. Then:
\begin{itemize}
\item Given $x<y\leq0$, there is a rational $r$ in $(-y, -x)$. So $-r$ is
  in $(x,y)$.
\item Given $x<0<y$, there is a rational $r$ in $(0,y)$. So $r$ is of
  course also in $(x,y)$.
\end{itemize}
Thus all we must do is prove there exists a rational in $(x, y)$ for
$0\leq x<y$.

\vs

Let $0\leq x<y$ be given. By the Archimedean property there exists
$n\in\mathcal{N}$ such that $\frac{1}{n}<y-x$. Because (a)
$\mathcal{N}$ is unbounded and (b) $\mathcal{N}$ is well-ordered, there exists the least
integer $m\in\mathcal{N}$ such that $m\geq ny$.

\vs

\textit{First}, observe that
\begin{align*}
&m-1<ny&&\text{or $m$ wouldn't be the \textit{least} integer $m\geq
          ny$}\\
&\implies\frac{m-1}{n}<y\\
\end{align*}

\textit{Second}, suppose for contradiction $\frac{m-1}{n}\leq x$. Then
\begin{align*}
  &\frac{m-1}{n}\leq x\\
  &\implies \frac{m}{n}-\frac{1}{n}\leq x\\
  &\implies -\frac{1}{n}\leq x-\frac{m}{n}\\
  &\implies \frac{1}{n}\geq \frac{m}{n}-x\\
  &\implies \frac{1}{n}\geq y-x&&\text{recall $m\geq ny$, thus $\frac{m}{n}\geq
                               y$}
\end{align*}
This is a contradiction, thus $\frac{m-1}{n}>x$.

\vs

Therefore $\frac{m-1}{n}\in(x,y)$ as desired.

\vs

\textbf{Claim:} the set of irrational numbers $\mathcal{R}\setminus\mathcal{Q}$ is dense.

\textbf{Proof:} let $x,y\in\mathcal{R}$ be given. By density of the rationals
there exists $r\in\mathcal{Q}$ such that
$\frac{x}{\sqrt{2}}<r<\frac{y}{\sqrt{2}}$. Multiplying each side by
$\sqrt{2}$, we get $x<\sqrt{2}r<y$. We know $\sqrt{2}r$ is irrational.
Thus there exists an irrational number between any two numbers in
$\mathcal{R}$, and the set of irrationals $\mathcal{R}\setminus\mathcal{Q}$ is dense as desired.

\subsection{Appendix: sqrt(2) is irrational}\label{sqrt2proof}
Suppose $\sqrt{2}\in\mathcal{Q}$. Then there exist $a,b\in\mathcal{N}$ such that
$\left(\frac{a}{b}\right)^{2}=2$. Assume $a, b$ have no common divisor
(since we can obviously keep simplifying until this is the case).
Observe that both $a$ and $b$ cannot be even, otherwise we could
simplify further.

\vs

Now we have $a^{2}=2b^{2}$. Thus $a^{2}$ is even, $a$ must be
even\footnote{Even numbers have even squares because
  ${(2k)}^{2}=4k^{2}=2\cdot(2k^{2})$}, and there exists
$k\in\mathcal{N}$ such that $a=2k$. Then $a^{2}=4k^{2}=2b^{2}$ so
$2k^{2}=b^{2}$. Thus $b^{2}$ is even and so $b$ is even. Since both
$a$ and $b$ cannot be even, this is a contradiction. Thus
$\sqrt{2}\notin\mathcal{Q}$ as desired.

\subsection{Problems}
\subsubsection*{Problem 5a}
Suppose that $y-x>1$. Prove that there is an integer $k$ such that
$x<k<y$. Hint: let $l$ be the largest integer satisfying $l\leq x$, and
consider $l+1$.

\subsubsection*{Solution}

\subsubsection*{Problem 5b}
Suppose $x<y$. Prove that there is a rational number $r$ such that
$x<r<y$. Hint: if $1/n<y-x$, then $ny-nx>1$. (Query: Why have parts
(a) and (b) been postponed until this problem set?)

\subsubsection*{Solution}

\subsubsection*{Problem 5c}
Suppose $r<s$ are rational numbers. Prove that there is an irrational
number between $r$ and $s$. Hint: As a start, you know that there is
an irrational number between $0$ and $1$.

\subsubsection*{Solution}

\subsubsection*{Problem 5d}
Suppose that $x<y$. Prove that there is an irrational number between
$x$ and $y$. Hint: It is unnecessary to do any more work; this follows
from (b) and (c).

\subsubsection*{Solution}

\subsubsection*{Problems 6a, b}
A set $A$ of real numbers is said to be \textbf{dense} if every open
interval contains a point of $A$. For example, Problem 5 shows that
the set of rational numbers and the set of irrational numbers are each
dense.

\vs

(a) Prove that if $f$ is continuous and $f(x)=0$ for all numbers $x$
in a dense set $A$, then $f(x)=0$ for all $x$.

\vs

(b) Prove that if $f$ and $g$ are continous and $f(x)=g(x)$ for all
$x$ in a dense set $A$, then $f(x)=g(x)$ for all $x$.

\subsubsection*{Solution}

\subsubsection*{Problem 11a}
Suppose that $a_{1}, a_{2}, a_{3}, \ldots$ is a sequence of positive
numbers with $a_{n+1}\leq a_{n}/2$. Prove that for any $\epsilon>0$ there is
some $n$ with $a_{n}<\epsilon$.

\subsubsection*{Solution}




%%% Local Variables:
%%% TeX-master: "notes"
%%% End:

\section{Continuity, Part II}

\subsection{Intermediate Value Theorem} \label{ivt}
\subsection{Extreme Value Theorem}
\subsection{Appendix: IVT and EVT consequences}

%%% Local Variables:
%%% TeX-master: "notes"
%%% End:


\section{Derivatives, Part I (The Fundamentals)}

\subsection{Formal definitions}

\textbf{Definition:} the \textit{derivative} at $a$ of a function $f$,
denoted $f'(a)$, is defined as:
\[f'(a)=\lim_{h\to0}\frac{f(a+h)-f(a)}{h}\]

There are two \textit{intuitions} to convey about the derivative:
\begin{itemize}
\item \textit{First}, draw a line through points $(a, f(a))$ and
  $(a+h, f(a+h))$ for some small $h$. Then make $h$ ``infinitely
  small''. Our $f'(a)$ is the slope of that line.
\item \textit{Second}, \textbf{TODO:} physics intuition.
\end{itemize}

\textbf{Definition:} $f$ is called \textit{differentiable} at $a$ if
the limit $f'(a)$ exists.

\vs

The notation $f'(a)$ suggests $f'$ is a function. Indeed, we define
$f'$ as follows. Its domain is the set of all numbers $a$ where $f$ is
differentiable, and its value at such a point $a$ is the limit above.
Not surprisingly, we call $f'$ the \textit{derivative} of $f$. Note
that the domain of $f'$ could be much smaller than the domain of $f$.

\vs

We can apply the definition of the derivative to $f'$ yielding the
\textit{second derivative} $(f')'$, denoted $f''$ or $f^{(2)}$. The
domain of $f''$ is all points $a$ such that $f'$ is differentiable at
$a$. If $f''(a)$ exists, we say $f$ is \textit{twice differentiable}
at $a$.

\vs

\textbf{Theorem:} if $f$ is differentiable at $a$, then $f$ is
continuous at $a$.

\textbf{Proof: TODO} we must show that:
\[\lim_{x\to a}f(x)=f(a)\]

Observe that $\lim_{x\to a}f(x)$ is equivalent to $\lim_{h\to 0}f(a+h)$.
\textbf{Todo.} The proof is now straightforward:
\begin{align*}
  \lim_{h\to0}[f(a+h)-f(a)]&=\lim_{h\to0}\frac{f(a+h)-f(a)}{h}\cdot h\\
                         &=\lim_{h\to0}\frac{f(a+h)-f(a)}{h}\cdot \lim_{h\to
                           0}h\\
                         &=\lim_{h\to0}\frac{f(a+h)-f(a)}{h}\cdot 0\\
                         &=0
\end{align*}

It follows that
\begin{align*}
  &\lim_{h\to0}[f(a+h)-f(a)]=0\\
  &\implies \lim_{h\to0}f(a+h)-\lim_{h\to0}f(a)=0\\
  &\implies \lim_{h\to0}f(a+h)=\lim_{h\to0}f(a)=f(a)
\end{align*}

\subsection{Leibniz notation}

The notation $f'$ is called Lagrange's notation.\footnote{Wikipedia
  claims the notation was invented by Euler and Lagrange only
  popularized it.} It's supposed to be modern, and Spivak's book
standardizes on it. Another notation commonly in use is the older (but
often convenient, instructive, but also initially confusing) Leibniz
notation.

\subsubsection*{Historical interpretation}

Leibniz didn't know about limits, and thought the derivative is the
value of the quotient $\frac{f(a+h)-f(a)}{h}$ when $h$ is
``infinitesimally small''. He denoted this infinitesimally small
quantity by $dx$, and the corresponding difference $f(x+dx)-f(x)$ by
$df(x)$. Thus for a given function $f$ the Leibniz notation for its
derivative $f'$ is:
\[\frac{df(x)}{dx}=f'\]

Intuitively, we can think of $d$ in a historical context as ``delta''
or ``change''. Then we can interpret this notation as Leibniz did-- a
quotient of a tiny change in $f(x)$ and a tiny change in $x$.

\vs

Leibniz notation for the second derivative is
\[\frac{d\left(\frac{df(x)}{dx}\right)}{dx},\ \ \ \text{abbreviated
    to}\ \ \ \frac{d^2f(x)}{(dx)^2}, \ \ \ \text{or more often to}\ \
  \ \frac{d^2f(x)}{dx^2}.\]

\subsubsection*{Modern interpretation}

Complete ordered fields do not have a notion of infinitesimally small
quantities. Thus in a modern interpretation we treat
$\frac{df(x)}{dx}$ as a symbol denoting $f'$, \textit{not} as a
quotient of numbers. Nothing here is being divided, nothing can be
canceled out. In a modern interpretation $\frac{df(x)}{dx}$ is just
one thing that \textit{happens to look} like a quotient.

\vs

There are two notable ambiguities associated with the Leibniz
notation. First, $\frac{df(x)}{dx}$ is frequently abbreviated to
$\frac{df}{dx}$. Second, $\frac{df(x)}{dx}$ sometimes means the
function $f'$, and sometimes means the value $f'(x)$. The meaning of
the symbol often must be deteremined from the specific context.

\subsection{Low-level proofs}

In the next chapter we prove theorems that make finding derivatives
for many classes of functions easy. But for now we show four low-level
derivations directly from the definition. Here we will be looking at
constant functions, linear functions, quadratic, and cubic functions.

\subsubsection*{Constant functions}
Let $f(x)=c$. Then:
\[f'(a)=\lim_{h\to0}\frac{f(a+h)-f(a)}{h}=\lim_{h\to0}\frac{c-c}{h}=0\]

Thus $f$ is differentiable at $a$ for every number $a$, and $f'(a)=0$.

\subsubsection*{Linear functions}
Let $f(x)=cx+d$. Then:
\begin{align*}
  f'(a)&=\lim_{h\to0}\frac{f(a+h)-f(a)}{h}\\
       &=\lim_{h\to0}\frac{c(a+h)+d-(ca+d)}{h}\\
       &=\lim_{h\to0}\frac{ch}{h}=c
\end{align*}

Thus $f$ is differentiable at $a$ for every number $a$, and $f'(a)=c$.

\subsubsection*{Quadratic functions}
Let $f(x)=x^2$. Then:
\begin{align*}
  f'(a)&=\lim_{h\to0}\frac{f(a+h)-f(a)}{h}\\
       &=\lim_{h\to0}\frac{(a+h)^2-a^2}{h}\\
       &=\lim_{h\to0}\frac{a^2+2ah+h^2-a^2}{h}\\
       &=\lim_{h\to0}\frac{2ah+h^2}{h}\\
       &=\lim_{h\to0}2a+h\\
       &=\lim_{h\to0}2a
\end{align*}

Thus $f$ is differentiable at $a$ for every number $a$, and $f'(a)=2a$.

\subsubsection*{Cubic functions}
Let $f(x)=x^3$. Then:
\begin{align*}
  f'(a)&=\lim_{h\to0}\frac{f(a+h)-f(a)}{h}\\
       &=\lim_{h\to0}\frac{(a+h)^3-a^3}{h}\\
       &=\lim_{h\to0}\frac{a^3+3a^2h+3ah^2+h^3-a^3}{h}\\
       &=\lim_{h\to0}\frac{3a^2h+3ah^2+h^3}{h}\\
       &=\lim_{h\to0}3a^2+3ah+h^2\\
       &=3a^2
\end{align*}

Thus $f$ is differentiable at $a$ for every number $a$, and $f'(a)=3a^2$.

\subsection{Non-differentiability}
Continuous functions are ``nice''. Functions that are differentiable
everywhere are ``nicer''. Functions that are differentiable everywhere
and whose first derivative is differentiable everywhere are nicer
still. Thus to fully understand the derivative we must understand
examples where it does not exist.

\vs

We now turn our attention to functions that aren't differentiable at
some points $a$. We first look at four simple examples where there
isn't everywhere a first derivative. We then turn our attention to a
more subtle example-- a function that's differentiable in the first,
but not everywhere in the second derivative.

\subsubsection*{First derivative}

\textbf{Example 1}

Let $f(x)=|x|$. Consider $f'(0)$:
\[f'(0)=\lim_{h\to0}\frac{f(0+h)-f(0)}{h}=\lim_{h\to0}\frac{|h|}{h}\]

Observe that $\lim_{h\to0^+}\frac{|h|}{h}=1$ and
$\lim_{h\to0^-}\frac{|h|}{h}=-1$. This $\lim_{h\to0}\frac{|h|}{h}$ does
not exist, and $f$ is not differentiable at $0$. Note that $f$ is
differentiable at every other point: $f'(a)=-1$ for $a<0$ and
$f'(a)=-1$ for $a>0$.

\vs

\textbf{Example 2}

Let $f$ be defined as follows:
\[f(x)=\begin{cases}
  x^2,&x\leq 0\\
  x,&x\geq 0
\end{cases}\]

Now consider $f'(0)$:
\[f'(0)=\lim_{h\to0}\frac{f(0+h)-f(0)}{h}=\lim_{h\to0}\frac{f(h)}{h}\]

Observe that
\[\frac{f(h)}{h}=\begin{cases}
  \frac{h^2}{h}=h,&h\leq0\\
  \frac{h}{h}=1,&h\geq0
\end{cases}\]

Therefore $\lim_{h\to0^-}\frac{f(h)}{h}=0$ and
$\lim_{h\to0^+}\frac{f(h)}{h}=1$. Thus $\lim_{h\to0}\frac{f(h)}{h}$ does
not exist, and $f$ is not differentiable at $0$.

\vs

\textbf{Example 3}

Let $f(x)=\sqrt{|x|}$. Consider $f'(0)$:
\[f'(0)=\lim_{h\to0}\frac{f(a+h)-f(a)}{h}=\lim_{h\to0}\frac{\sqrt{|h|}}{h}\]

Observe that
\[\frac{\sqrt{|h|}}{h}=\begin{cases}
  \frac{\sqrt{-h}}{h}=-\frac{1}{\sqrt{-h}},&h<0\\
  \frac{\sqrt{h}}{h}=\frac{1}{\sqrt{h}},&h>0
\end{cases}\]

Therefore $\lim_{h\to0^+}\frac{\sqrt{|h|}}{h}=\infty$ and
$\lim_{h\to0^-}\frac{\sqrt{|h|}}{h}=-\infty$. Thus
$\lim_{h\to0}\frac{\sqrt{|h|}}{h}$ does not exist, and $f$ is not
differentiable at $0$.

\vs

\textbf{Example 4}

Let $f(x)=\sqrt[3]{x}$.


\subsubsection*{Second derivative}
We now come to our more subtle example-- a function that's
differentiable in the first but not everywhere in the second
derivative:

\[f(x)=\begin{cases}
  x^2,&x\geq0\\
  -x^2,&x\leq0
  \end{cases}\]

\subsection{Intersections}

%%% Local Variables:
%%% TeX-master: "notes"
%%% End:


\end{document}

%%% Local Variables:
%%% TeX-master: t
%%% End:
