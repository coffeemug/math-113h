\section{Complete ordered fields}

\subsection{Motivation}
The twelve ordered field axioms are sufficient to define limits,
continuity, and prove all the theorems in the previous sections. Since
the set $\mathcal{Q}$ of rational numbers is an ordered field\footnote{The proof
  is straightforward, so I'm not including it here.}, rationals have
been sufficient for the work we've done so far. However, we are about
to start proving slightly more sophisticated theorems about continous
functions, and ordered fields will quickly start breaking our
intuitions.

\vs

For example, consider the function $f(x)=x^{2}-2$ (a parabola shifted
down two units). It's easy to see $f$ is a continuous function, and
thus our intuition is that we should be able to draw it without
``lifting the tip of the pencil off the sheet of paper''. Upon
reflection however, it becomes obvious that in the universe limited to
ordered fields this is impossible. $f$ intersects the x-axis when
$x^{2}=2$, but every high school student knows
$\sqrt{2}\notin\mathcal{Q}$ (see \ref{sqrt2proof} for proof). Thus there is no
$x\in\mathcal{Q}$ such that $f(x)=0$. And since $\mathcal{Q}$ is an ordered field, it
follows ordered fields alone aren't sufficient to resolve this
problem.

\vs

The \textit{intermediate value theorem} (see \ref{ivt}) formalizes the
claim that a continuous function segment that starts below the x-axis
and ends above the x-axis intersects the x-axis. But as we can see
from the example above, this is not possible to prove with ordered
field axioms alone. So before we proceed with further study of
continuity, we need one more axiom called \textit{the completeness
  axiom}, which we introduce in this chapter.

\vs

Combined with the twelve ordered field axioms, the completeness axiom
forms \textit{complete ordered fields}. These objects are sufficient
to proceed with our study of calculus. We will see that rational
numbers $\mathcal{Q}$ are not a complete ordered field, whereas real numbers
$\mathcal{R}$ are.\footnote{Proof that $\mathcal{R}$ is a complete ordered field requires
  construction of $\mathcal{R}$, which doesn't happen in Spivak until the last
  chapters. Thus I will not be delving into that here and ask the
  reader (i.e., currently myself) to take this on faith.} Thus from
here $\mathcal{R}$-valued functions will become our primary object of study.

\subsection{Least upper bound}
\textbf{Definition:} $b$ is an \textbf{upper bound} for $S$ if
$s\leq b$ for all $s\in S$.

\vs

For example:
\begin{itemize}
\item Any $b\geq1$ is an upper bound for $S=\{x:0\leq x<1\}$. E.g. $1, 2,
  10$ are all upper bounds of $S$.
\item By convention, \textit{every} number is an upper bound for $\emptyset$.
\item The set $\mathcal{N}$ of natural numbers has no natural upper bound. The
  proof is easy. Suppose $b\in\mathcal{N}$ is an upper bound for
  $\mathcal{N}$. But $b+1\in\mathcal{N}$, and $b+1>b$, which is a contradiction. Thus
  $b$ isn't an upper bound for $\mathcal{N}$.\footnote{We need to do a little
    more work to show $\mathcal{N}$ has no upper bound, natural or not. Be
    patient! We will prove this by the end of the section.}
\end{itemize}

\vs

\textbf{Definition:} $x$ is a \textbf{least upper bound} of $A$, if
\begin{enumerate}
\item $x$ is an upper bound of $A$,
\item \textit{and} if $y$ is an upper bound of $A$, then $x\leq y$.
\end{enumerate}

A set can have only one least upper bound. The proof is easy. Suppose
$x$ and $x'$ are both least upper bounds of $S$. Then $x\leq x'$ and
$x'\leq x$. Thus $x=x'$. Consequently, we can use a convenient notation
$\sup A$ to denote the least upper bound of $A$.

\vs

Obligatory examples:
\begin{itemize}
\item Let $S=\{x:0\leq x<1\}$. Then $\sup S=1$.
\item By convention, the empty set $\emptyset$ has no least upper bound.
\end{itemize}

\subsection{Completeness axiom}
We are now ready to state the completeness axiom.

\vs

\textbf{Completeness [P13]:} If $A$ is a non-empty set of numbers that
has an upper bound, then it has a least upper bound.

\vs

\textbf{Claim:} rational numbers are not complete.

\textbf{Proof:} Let $C=\{x:x^{2}<2\text{ and }x\in\mathcal{Q}\}$. Suppose for
contradiction rational numbers are complete. Then there exists
$b\in\mathcal{Q}$ such that $b=\sup C$. Observe that
\begin{itemize}
\item $b^{2}\neq2$ as that would imply $b=\sqrt{2}$ and thus $b\notin\mathcal{Q}$.
\item $b^{2}\not<2$ as there would exist some $x\in C$ such that
  $b^{2}<x^{2}<2$. Thus $b<x$ and $b$ is not the upper bound.
\end{itemize}

Therefore $b^{2}>2$. But this implies there exists some
$x\in\mathcal{Q}$ such that $2<x^{2}<b^{2}$. Thus $x$ is greater than every
element in $C$, and $x<b$. So $b$ is not the \textit{least} upper
bound. We have a contradiction, therefore rational numbers are not
complete, as desired.

\vs

\textbf{Claim:} completeness cannot be derived from ordered fields.

\textbf{Proof:} $\mathcal{Q}$ is not complete and $\mathcal{Q}$ is an ordered field. Thus
completeness is not a property of ordered fields.

\vs

\textbf{Claim:} real numbers are complete.

\textbf{Proof [deferred]:} The completeness property can be derived
from the construction of real numbers $\mathcal{R}$, which makes reals a
\textbf{complete ordered field}. The proof requires we study the
actual construction of $\mathcal{R}$, which Spivak leaves until the last
chapters. Thus for the moment the proof will be taken on faith. In any
case, it is better to build calculus upon abstract complete ordered
fields than upon concrete real numbers.

\subsection{Consequences of completeness}

\subsubsection*{$\mathcal{N}$ is not bounded above}
We've shown $\mathcal{N}$ has no upper bound in $\mathcal{N}$. Now we
show $\mathcal{N}$ has no upper bound in $\mathcal{R}$.

\vs

Suppose for contradiction $\mathcal{N}$ has an upper bound. Since $\mathcal{N}\neq\emptyset$ then by
completeness $\mathcal{N}$ has a least upper bound. Let $\alpha=\sup \mathcal{N}$. Then:
\begin{align*}
  &\alpha\geq n \text{ for all } n\in\mathcal{N}\\
  \implies &\alpha\geq n+1 \text{ for all } n\in\mathcal{N}&&\text{since $n+1\in\mathcal{N}$ if $n\in\mathcal{N}$}\\
  \implies &\alpha-1\geq n \text{ for all } n\in\mathcal{N}
\end{align*}
Thus $\alpha-1$ is \textit{also} an upper bound for $\mathcal{N}$. This contradicts
that $\alpha=\sup \mathcal{N}$. Therefore $\mathcal{N}$ is not bounded above, as desired.

\vs

\subsubsection*{$\sqrt{2}$ exists}

We show $\sqrt{2}\in\mathcal{R}$. Let
$S=\{y\in\mathcal{R} : y^{2}<2\}$. Obviously $S$ is non-empty and has an upper
bound. Thus by completeness property it has a least upper bound. Let
$x=\sup S$. Note that $1\in S$ and $2$ is an upper bound of $S$. Thus
$1\leq x\leq2$. We show $x^{2}=2$ by showing $x^{2}\not<2$ and
$x^{2}\not>2$.

\vs

\textit{Case 1}. Suppose for contradiction $x^{2}<2$. Let $0<\epsilon<1$ be a
small number. Then
\begin{align*}
  {(x+\epsilon)}^{2}=&x^{2}+2\epsilon x+\epsilon^{2}\\
              &\leq x^{2}+4\epsilon+\epsilon&&\text{since $x<2$ and $\epsilon<1$}\\
              &=x^{2}+5\epsilon<2&&\text{since $x^{2}<2$ (by supposition), we
                             can pick}\\
              &&&\text{a small enough $\epsilon$ to make this true}
\end{align*}
Thus there exists $\epsilon$ such that ${(x+\epsilon)}^{2}<2$. By definition of
$S$ it follows $x+\epsilon\in S$, which contradics that $x$ is the least upper
bound. Therefore $x^{2}\not<2$

\vs

\textit{Case 2}. Suppose for contradiction $x^{2}>2$. Let $0<\epsilon<1$ be a
small number. Then
\begin{align*}
  {(x-\epsilon)}^{2}=&x^{2}-2\epsilon x+\epsilon^{2}\\
              &\geq x^{2}-2\epsilon x&&\text{since $\epsilon^{2}>0$}\\
              &\geq x^{2}-4\epsilon&&\text{since $x\leq2$}\\
              &>2&&\text{since $x^{2}>2$ (by supposition), we
                             can pick}\\
              &&&\text{a small enough $\epsilon$ to make this true}
\end{align*}

Thus ${(x-\epsilon)}^{2}>2$, which by definition of $S$ implies
$x-\epsilon>y$ for all $y\in S$. So $x-\epsilon$ is an upper bound of
$S$. We have a contradiction-- since $x-\epsilon<x$, it follows $x$ is not a
least upper bound. Therefore $x^{2}\not>2$ as desired.

\vs

Since $x^{2}\not<2$ and $x^{2}\not>2$, it follows $x^{2}=2$ as
desired.


\subsubsection*{Archimedean property}
Handwavy definition: the Archimedean property states that you can fill
the universe with tiny grains of sand.

\vs

Formal defition: let $\epsilon>0$ be small and let $r>0$ be large. Then there
exists $n\in\mathcal{N}$ such that $n\epsilon>r$.

\vs

\textbf{Proof:} suppose for contradiction the property is false. Then
there exist $\epsilon, r$ such that for all $n\in\mathcal{N}$, $n\epsilon\leq r$. Therefore
$n\leq\frac{r}{\epsilon}$. This implies $\mathcal{N}$ is bounded, which a contradiction.

\vs

A useful special case is when $r=1$. In this case the Archimedean
property can be restated as follows. Let $\epsilon>0$ be small. Then there
exists $n\in\mathcal{N}$ such that $n\epsilon>1$. Put differently, there exists
$n\in\mathcal{N}$ such that $\frac{1}{n}<\epsilon$.

\vs

A few more notes on the Archimedean property:
\begin{itemize}
\item Obviously the Archimedean property follows from completeness, as
  shown above.
\item The Archimedean property is true in $\mathcal{Q}$ and can be proven
  without being assumed\footnote{Excluding the proof here, but it's
    fairly simple}.
\item Completeness does not follow from the Archimedean property. The
  proof is easy: the Archimedean property holds on $\mathcal{Q}$, and we know
  $\mathcal{Q}$ is not complete as shown above.
\end{itemize}


\subsubsection*{Density}
Let $x,y\in\mathcal{R}$. Then $S$ is a \textbf{dense subset} of
$\mathcal{R}$ if there is an element of $S$ in $(x,y)$. Put differently, there
is an element of $S$ between any two points in $\mathcal{R}$.
\begin{itemize}
\item Obviously $\mathcal{R}$ is a dense subset of itself.
\item Integers are not a dense subset of $\mathcal{R}$. E.g. there is no integer
  between $1.1$ and $1.9$.
\item The set of positive numbers $\{x:x\in\mathcal{R}, x>0\}$ is not a dense
  subset of $\mathcal{R}$. E.g. there is no positive number between
  $-2$ and $-1$.
\end{itemize}

\textbf{Claim:} the set of rational numbers $\mathcal{Q}$ is dense.

\textbf{Proof:} let $x,y\in\mathcal{R}$ be given. Suppose we can show there exists
a rational in $(x, y)$ for $0\leq x<y$. Then:
\begin{itemize}
\item Given $x<y\leq0$, there is a rational $r$ in $(-y, -x)$. So $-r$ is
  in $(x,y)$.
\item Given $x<0<y$, there is a rational $r$ in $(0,y)$. So $r$ is of
  course also in $(x,y)$.
\end{itemize}
Thus all we must do is prove there exists a rational in $(x, y)$ for
$0\leq x<y$.

\vs

Let $0\leq x<y$ be given. By the Archimedean property there exists
$n\in\mathcal{N}$ such that $\frac{1}{n}<y-x$. Because (a)
$\mathcal{N}$ is unbounded and (b) $\mathcal{N}$ is well-ordered, there exists the least
integer $m\in\mathcal{N}$ such that $m\geq ny$.

\vs

\textit{First}, observe that
\begin{align*}
&m-1<ny&&\text{or $m$ wouldn't be the \textit{least} integer $m\geq
          ny$}\\
&\implies\frac{m-1}{n}<y\\
\end{align*}

\textit{Second}, suppose for contradiction $\frac{m-1}{n}\leq x$. Then
\begin{align*}
  &\frac{m-1}{n}\leq x\\
  &\implies \frac{m}{n}-\frac{1}{n}\leq x\\
  &\implies -\frac{1}{n}\leq x-\frac{m}{n}\\
  &\implies \frac{1}{n}\geq \frac{m}{n}-x\\
  &\implies \frac{1}{n}\geq y-x&&\text{recall $m\geq ny$, thus $\frac{m}{n}\geq
                               y$}
\end{align*}
This is a contradiction, thus $\frac{m-1}{n}>x$.

\vs

Therefore $\frac{m-1}{n}\in(x,y)$ as desired.

\vs

\textbf{Claim:} the set of irrational numbers $\mathcal{R}\setminus\mathcal{Q}$ is dense.

\textbf{Proof:} let $x,y\in\mathcal{R}$ be given. By density of the rationals
there exists $r\in\mathcal{Q}$ such that
$\frac{x}{\sqrt{2}}<r<\frac{y}{\sqrt{2}}$. Multiplying each side by
$\sqrt{2}$, we get $x<\sqrt{2}r<y$. We know $\sqrt{2}r$ is irrational.
Thus there exists an irrational number between any two numbers in
$\mathcal{R}$, and the set of irrationals $\mathcal{R}\setminus\mathcal{Q}$ is dense as desired.

\subsection{Appendix: sqrt(2) is irrational}\label{sqrt2proof}
Suppose $\sqrt{2}\in\mathcal{Q}$. Then there exist $a,b\in\mathcal{N}$ such that
$\left(\frac{a}{b}\right)^{2}=2$. Assume $a, b$ have no common divisor
(since we can obviously keep simplifying until this is the case).
Observe that both $a$ and $b$ cannot be even, otherwise we could
simplify further.

\vs

Now we have $a^{2}=2b^{2}$. Thus $a^{2}$ is even, $a$ must be
even\footnote{Even numbers have even squares because
  ${(2k)}^{2}=4k^{2}=2\cdot(2k^{2})$}, and there exists
$k\in\mathcal{N}$ such that $a=2k$. Then $a^{2}=4k^{2}=2b^{2}$ so
$2k^{2}=b^{2}$. Thus $b^{2}$ is even and so $b$ is even. Since both
$a$ and $b$ cannot be even, this is a contradiction. Thus
$\sqrt{2}\notin\mathcal{Q}$ as desired.


%%% Local Variables:
%%% TeX-master: "notes"
%%% End:
