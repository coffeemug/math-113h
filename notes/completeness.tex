\section{Completness property}

\subsection{Motivation}
The set $\mathcal{Q}$ of rational numbers satisfies the twelve ordered field
properties. This is sufficient to define limits, continuity, and prove
all the theorems in previous sections. However, we are about to start
proving slightly more sophisticated theorems about continous
functions, where $\mathcal{Q}$ quickly breaks our intuition.

\vs

For example, consider the function $f(x)=x^{2}-2$ (a parabola shifted
down two units). It's easy to see $f$ is a continuous function, and
thus our intuition is that we should be able to draw it without
``lifting the tip of the pencil off the paper''. Upon reflection
however, it becomes obvious that in the universe of $\mathcal{Q}$ this is
impossible. $f$ crosses the x-axis when $x^{2}=2$, but we know
$\sqrt{2}$ is not a rational number, i.e. $\sqrt{2}\notin\mathcal{Q}$. Thus there is
no $x\in\mathcal{Q}$ such that $f(x)=0$.

\vs

The \textit{intermediate value theorem} (see \ref{ivt}) formalizes the
claim that a continuous function stretch that starts below the x-axis
and ends above the x-axis crosses the x-axis. But as we see from the
example above, this is not possible to prove with ordered field
properties alone. Before we proceed we need one more property called
\textit{the completeness property}, which we introduce in this
chapter. We will see that $\mathcal{Q}$ does not satisfy this property, whereas
$\mathcal{R}$ does.



\subsection{Least upper bound}
\subsection{Appendix: completeness consequences}
\subsection{Appendix: square root of two irrationality}


%%% Local Variables:
%%% TeX-master: "notes"
%%% End:
