
\section{Derivatives, Part III (Consequences)}

I reordered some content from Spivak's chapter for clarity, but kept
theorem and corollary numbering scheme. Thus theorem numbers here
aren't in order (but they match the numbers in Spivak).

\subsection{Maxima and manima}
We start with some definitions. Let $f$ be a function and $A$ a set of
numbers contained in $f$'s domain.\footnote{$A$ need not have any
  additional properties. E.g. it may have holes, etc.} Then:

\vs

\textbf{Definition.} A point $x$ in $A$ is a \textbf{maximum
  point} for $f$ on $A$ if
\[f(x)\geq f(y)\qquad\text{for every $y$ in $A$}\]

\textbf{Definition.} A \textbf{critical point} of $f$ is a number $x$
such that $f'(x)=0$.\footnote{If $x$ is a maximum and/or critical
  point, then $f(x)$ is called a maximum and/or critical value of
  $f$.}

\vs---\vs

\textbf{Theorem 1a.} Let $f$ be any function defined on $(a,b)$. If $x$
is a maximum point for $f$ on $(a,b)$, and $f$ is differentiable at
$x$, then $f'(x)=0$.

\vs

\textit{Intuitively,} maximum and minimum points are also critical
points (but \textbf{not} the other way around-- $f(x)=x^3$ has
$f'(0)=0$ as an obvious counterexample\footnote{In this case this
  critical point is called the \textbf{saddle point}.}).

\vs

\textbf{Proof.} \textit{Informally}, suppose $a$ is a maximum point.
Draw a secant line between $a$ and $a_l$ (to its left), and another
line between $a$ and $a_r$ (to its right). The $a-a_l$ line will slope
up, the $a-a_r$ line will slope down. Thus at $a$ the slope crosses
from positive to negative, and is $0$.

\vs

\textit{Formally}, let $h\in\mathcal{R}$ such that $x+h\in(a,b)$. If $h<0$ it follows that:
\begin{align*}
  &f(x+h)\leq f(x)&\text{since $f(x)$ is a maximum value}\\
  &\implies f(x+h)-f(x)\leq 0\\
  &\implies\frac{f(x+h)-f(x)}{h}\geq0&\text{dividing by negative $h$}\\
  &\implies \lim_{h\to0^-}\frac{f(x+h)-f(x)}{h}\geq0&\text{see
                                                 \ref{subsec:onesided-limits}
                                                 and
                                                 \ref{subsubsec:nonzero-lemma}}
\end{align*}

Conversely, if $h>0$ it follows that:
\begin{align*}
  &\implies \frac{f(x+h)-f(x)}{h}\leq0&\text{dividing by positive $h$}\\
  &\implies \lim_{h\to0^+}\frac{f(x+h)-f(x)}{h}\leq0
\end{align*}

By hypothesis, $f$ is differentiable at $x$. Thus the two limits must
be equal to each other, and to $f'(x)$. Therefore $f'(x)\geq0$ and
$f'(x)\leq0$. Thus $f'(x)=0$ as desired.

\vs

\textbf{Theorem 1b.} Let $f$ be any function defined on $(a,b)$. If
$x$ is a \textit{minimum} point for $f$ on $(a,b)$, and $f$ is
differentiable at $x$, then $f'(x)=0$.

\vs

\textbf{Proof.} Let $g=-f$. Then $x$ is a maximum point of $g$. By 1a,
$g'(x)=0$, thus $(-f)'(x)=-1\cdot f'(x)=0$, and thus $f'(x)=0$ as desired.

\vs---\vs

The obvious (extremely valuable) consequences of these theorems is
that we can find minimum and maximum values of $f$ by solving for
$f'(x)=0$.

\subsection{Mean Value Theorem}

\textbf{Theorem 3 (Rolle's theorem).} Let $f$ be continuous on $[a,b]$
differentiable on $(a,b)$, and let $f(a)=f(b)$. Then there exists
$x\in(a,b)$ such that $f'(x)=0$.

\vs

\textbf{Proof.} There are two cases:
\begin{itemize}
\item \textit{Case 1.} Suppose the maximum or the minimum occurs at a
  point $x\in(a,b)$. Then $f'(x)=0$ by theorem 1, and we are done.
\item \textit{Case 2.} Suppose the maximum and the minimum both occur
  at endpoints. Since $f(a)=f(b)$, the maximum and the minimum values
  are equal and $f$ is constant. Then for any $x\in(a,b), f'(x)=0$ and
  we are done.
\end{itemize}

\textbf{Theorem 4 (Mean value theorem).}

Let $f$ be continuous on $[a,b]$ and differentiable on $(a,b)$. Then
there exists $x\in(a,b)$ such that:
\[f'(x)=\frac{f(b)-f(a)}{b-a}\]

Here are three intuitions:
\begin{enumerate}
\item \textit{Geometric intuition.} There exists a line tangent to $f$
  parallel to the line between the endpoints (i.e. line between
  $(a, f(a))$ and $(b, f(b))$).
\item \textit{Algebraic intuition.} There exists a point $x$ at which
  instantaneous rate of change of $f$ is equal to the average change
  of $f$ on $[a,b]$.
\item \textit{Physical example.} If you travel 60 miles in one hour,
  at some point you must have been travelling exactly 60 miles per
  hour.
\end{enumerate}

\textbf{Proof.}

Here's an informal proof outline:
\begin{itemize}
\item Take the line segment formed by endpoints $(a,f(a))$ and
  $(b,f(b))$.
\item Construct a function $g$ that for $x\in(a,b)$ returns the vertical
  distance between $f(x)$ and the line segment. (We'll show it's
  continuous and differentiable.)\footnote{It turns out not to matter
    whether $g$ computes the distance between $f$ and the line
    segment, or $f$ and the line segment shifted down by $f(a)$ (i.e.
    down to $x$-axis). So in practice we use the lattern form to avoid
    dealing with the $f(a)$ term in the linear equation.}
\item By Rolle's theorem, it has a flat tangent. It's easy to show
  algebraically (and visualize geometrically) this proves the MVT.
\end{itemize}

Formally, let\footnote{See \ref{subsubsec-point-slope-form} for how
  the point-slope form is used to construct the second term.}
\[h(x)=f(x)-\left[\frac{f(b)-f(a)}{b-a}(x-a)\right]\]

Observe $h$ is continuous on $[a,b]$ and differentiable on $(a,b)$.
Further:
\begin{align*}
  h(a)&=f(a)-\left[\frac{f(b)-f(a)}{b-a}\cdot0\right]=f(a)\\
  h(b)&=f(b)-\left[\frac{f(b)-f(a)}{b-a}(b-a)\right]\\
      &=f(b)-[f(b)-f(a)]\\
      &=f(a)
\end{align*}

Thus we can apply Rolle's Theorem $h$ to conclude there is $x\in(a,b)$
such that:\footnote{Note the derivative of a line is its slope, thus
  $\frac{d}{dx}\left[\frac{f(b)-f(a)}{b-a}(x-a)\right]=\frac{f(b)-f(a)}{b-a}$.}

\begin{align*}
  &0=h'(x)=f'(x)-\frac{f(b)-f(a)}{b-a}\\
  &\implies f'(x)=\frac{f(b)-f(a)}{b-a}
\end{align*}

QED.


\subsection{MVT consequences}
\textbf{Corollary 1.} If $f$ is defined on an interval and $f'(x)=0$
for all $x$ in the interval, then $f$ is constant on the interval.

\vs

\textit{Intuitively,} if the velocity of a particle is always zero,
the particle must be standing still.

\vs

\textbf{Proof.} Let $a\neq b$ be any two points on the interval. Then
there is $x\in(a,b)$ such that $f'(x)=\frac{f(b)-f(a)}{b-a}$. But
$f'(x)=0$ for all $x$ on the interval, thus $0=\frac{f(b)-f(a)}{b-a}$.
Thuf $f(a)=f(b)$ for any $a,b$ (i.e. $f$ is constant on the interval
as desired).

\vs---\vs

\textbf{Corollary 2.} If $f,g$ are defined on the same interval, and
$f'(x)=g'(x)$ for all $x$ in the interval, then there is
$c\in\mathcal{R}$ such that $f=g+c$.

\vs

\textbf{Proof.} Observe that
\begin{align*}
  &f'(x)=g'(x)\\
  &\implies f'(x)-g'(x)=0\\
  &(f-g)'(x)=0
\end{align*}

By corollary 1, $(f-g)$ is constant, i.e. $f=g+c$ as desired.

\vs---\vs

\textbf{Definition.} A function is \textbf{increasing} on an interval
if $f(a)<f(b)$ whenever $a,b$ are two numbers in the interval with
$a<b$.\footnote{The decreasing function definition is obvious.}

\vs

\textbf{Corollary 3a.} If $f'(x)>0$ for all $x$ on an interval, then
$f$ is increasing on the interval.

\vs

\textbf{Proof.} Let $a<b$ be two points on an interval. Then there
exists $x\in(a,b)$ such that
\[f'(x)=\frac{f(b)-f(a)}{b-a}\]

But $f'(x)>0$ for all $x\in(a,b)$, thus
\[\frac{f(b)-f(a)}{b-a}>0\]

We know $b-a>0$, thus $f(b)>f(a)$ as desired.

\vs---\vs

\textbf{Corollary 3b.} If $f'(x)<0$ for all $x$ on an interval, then
$f$ is decreasing on the interval.

\vs

\textbf{Proof.} The proof is an obvious modification of 3a.

\subsection{Local maxima and manima}
\textbf{Definition.} A point $x$ in $A$ is a \textbf{local maximum
  point} for $f$ on $A$ if there is some $\delta>0$ such that $x$ is a
maximum point for $f$ on $A\cap(x-\delta, x+\delta)$.

\vs

\textbf{Theorem 2.} If $f$ is defined on $(a,b)$ and has a local
maximum (or minimum) at $x$, and $f$ is differentiable at $x$, then
$f'(x)=0$.

\vs

\textbf{Proof.} The proof is a trivial application of theorem 1 to $f$
on $(x-\delta, x+\delta)$.

\subsection{Graph sketching}

\subsection{L'H\^opital's rule}
We will build up to L'H\^opital's rule in the following way:

\begin{enumerate}
\item First, we'll cover a special case of L'H\^opital's rule (theorem
  7, no holes in derivatives).
\item To prove the L'H\^opital's rule we'll need Cauchy's Mean Value
  Theorem (a generalization of the Mean Value Theorem). When stated
  outright it can be hard to parse, so we'll build up to it informally
  next.
\item Formally prove Cauchy's Mean Value Theorem.
\item Prove L'H\^opital's rule.
\item We then reprove theorem 7 in a simpler way using L'H\^opital's
  rule.
\end{enumerate}

---\vs

\textbf{Theorem 7.} Suppose (1) $f$ is continuous at $a$, (2) $f'(x)$
exists for all $x$ in $0<|x-a|<\delta$, and (3) $\lim_{x\to a}f'(x)$ exists.
Then $f'(a)$ exists and
\[f'(a)=\lim_{x\to a}f'(x)\]

\textit{Intuitively,} derivatives cannot have holes. Put differently,
$f'$ \textit{can} be discontinuous at $a$ by fluctuating wildly near
$a$\footnote{In \ref{subsec-sine-poly} we've already seen an example
  of this:
  \[f(x)=\begin{cases}
    x^2\sin \frac{1}{x},&x\neq0\\
    0,&x=0.
  \end{cases}\]}, but
not by being undefined at $a$, or by being defined at $a$ to be far from its limit
near $a$.

\vs

\textbf{Proof 1.}\footnote{We will give a second proof in terms of
  L'H\^opital's rule at the end of this chapter.} Informally, for
``nice'' functions like $f$, the mean value theorem applies on tiny
scales of $\delta-\epsilon$ limits.

\vs

By derivative definition
\[f'(a)=\lim_{h\to 0}\frac{f(a+h)-f(a)}{h}\]

For sufficiently small $h$, both positive and negative, by supposition:
\begin{itemize}
\item $f$ will be continuous on $[a,a+h]$
\item $f$ will be differentiable on $(a,a+h)$
\end{itemize}

These conditions are sufficient for the mean value theorem, and thus
there exists $\alpha_h\in(a, a+h)$ such that:
\[\frac{f(a+h)-f(a)}{h}=f'(\alpha_h)\]

Putting this together:\footnote{Spivak observes the last equation in
  the proof is handwavy and needs a proper $\delta-\epsilon$ proof. I need to move
  on, so leaving this as a TODO.}
\begin{align*}
  f'(a)&=\lim_{h\to0}\frac{f(a+h)-f(a)}{h}&\text{by derivative definition}\\
       &=\lim_{h\to0}f'(\alpha_h)&\text{by mean value theorem}\\
       &=\lim_{x\to a}f'(x)&\text{$\alpha_h\in(a,a+h)$, thus as $h\to0$, $\alpha_h\to a$}
\end{align*}

\vs---\vs

\textbf{Theorem 8 (Cauchy's MVT), handwavy version.}

\vs

Let $f,g$ be continuous on $[a,b]$ and differentiable on $(a,b)$.
\textit{Intuitively,} theorem 8 states that there exists a point
$x\in(a,b)$ where $\frac{f'(x)}{g'(x)}$ (i.e. the ratio of
\textit{instantaneous} changes of $f$ and $g$) is the same as the
ratio of \textit{average} changes of $f$ and $g$ on $[a,b]$.

\vs

A more formal version of this is:
\begin{align*}
  \frac{f'(x)}{g'(x)}&=\frac{f(b)-f(a)}{b-a}\div \frac{g(b)-g(a)}{b-a}\\
                     &=\frac{f(b)-f(a)}{b-a}\cdot \frac{b-a}{g(b)-g(a)}\\
                     &=\frac{f(b)-f(a)}{g(b)-g(a)}
\end{align*}

when $g'(x)\neq 0$ and $g(b)-g(a)\neq0$.

\vs

There are two additional considerations. First, if $g(x)=x$ then
$g'(x)=1$, and the theorem simplifies to $f'(x)=\frac{f(b)-f(a)}{b-a}$
(i.e. we obtain the mean value theorem).

\vs

Second, to avoid division by zero constraints, formally the Cauchy
theorem is expressed as a multiplication rather than division of
terms:

\begin{align*}
  &\frac{f'(x)}{g'(x)}=\frac{f(b)-f(a)}{g(b)-g(a)}\\
  &\implies [f(b)-f(a)]g'(x)=[g(b)-g(a)]f'(x)
\end{align*}

With this buildup, we're ready to prove Cauchy's MVT.

\vs

\textbf{Theorem 8 (Cauchy's MVT).} Let $f,g$ be continuous on $[a,b]$
and differentiable on $(a,b)$. Then there exists $x\in(a,b)$ such that
\[[f(b)-f(a)]g'(x)=[g(b)-g(a)]f'(x)\]

\vs

\textbf{Proof.} Let
\[h(x)=f(x)[g(b)-g(a)]-g(x)[f(b)-f(a)]\]

Then $h$ is continuous on $[a,b]$ and differentiable on $(a,b)$.
Further observe that\footnote{If you plug $a$ and $b$ into $h(x)$, I
  promise this works (I checked).}
\[h(a)=f(a)g(b)-f(b)g(a)=h(b)\]

Thus $h(a)=h(b)$, Rolle's theorem applies, and there exists
$x\in(a,b)$ such that $h'(x)=0$. Taking the derivative of
$h$\footnote{Note to derive $h'$ we treat $g(b)-g(a)$ and $f(b)-f(a)$
  as constants.}:
\[0=h'(x)=f(x)'[g(b)-g(a)]-g(x)'[f(b)-f(a)]\]

---\vs

\textbf{Theorem 9 (L'H\^opital's rule.)} Suppose that (1)
$\lim_{x\to a}f(x)=0$ and $\lim_{x\to a}g(x)=0$, and (2)
$\lim_{x\to a}\frac{f'(x)}{g'(x)}$ exists. Then
$\lim_{x\to a}\frac{f(x)}{g(x)}$ exists, and
\[\lim_{x\to a}\frac{f(x)}{g(x)}=\lim_{x\to a}\frac{f'(x)}{g'(x)}\]

\textbf{Proof.}

\vs---\vs

\textbf{Theorem 7, proof 2.} We can now offer a second proof for
Theorem 7, as it turns out to be a special case of L'H\^opital's rule.

\vs

Recall, the theorem asserts the following. Suppose (1) $f$ is
continuous at $a$, (2) $f'(x)$ exists for all $x$ in $0<|x-a|<\delta$, and
(3) $\lim_{x\to a}f'(x)$ exists. Then $f'(a)$ exists and
\[f'(a)=\lim_{x\to a}f'(x)\]

\vs

By derivative definition:
\[f'(a)=\lim_{h\to0}\frac{f(a+h)-f(a)}{h}\]

Let $x=a+h$. We can rewrite the equation above as follows:
\[f'(a)=\lim_{x\to a}\frac{f(x)-f(a)}{x-a}\]

Clearly $\lim_{x\to a}[f(x)-f(a)]=0$ and $\lim_{x\to a}[x-a]=0$. Thus by
L'H\^opital's rule:
\[f'(a)=\lim_{x\to a}\frac{f'(x)}{1}=\lim_{x\to a}f'(x),\]

as desired.

%%% Local Variables:
%%% TeX-master: "notes"
%%% End:
