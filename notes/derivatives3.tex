
\section{Derivatives, Part III (Leibniz notation)}

The notation $f'$ that we've used so far is called the Lagrange
notation.\footnote{Wikipedia claims the notation was invented by Euler
  and Lagrange only popularized it.} However, there is another
notation for the derivative in common use. You may have already seen
something like $\frac{dy}{dx}$. This is called the Leibniz notation.

\vs

The Leibniz notation has many of what Spivak calls ``vagaries''. It
has multiple interpretations-- formal and informal. The informal
interpretation doesn't map to modern mathematics, but can
\textit{sometimes} be useful (while at other times misleading). The
full, unambigous Leibniz notation is verbose, so in practice people
end up taking liberties with it. As a consequence, its meaning must
often be discerned from the context.

\vs

This flexibility makes the notation very useful in science and
engineering, but also makes it difficult to learn. For clarity Spivak
standardizes on the Lagrange notation and banishes Leibniz notation to
the problems. But since the Leibniz notation is so common, I take a
different approach and explore it here in a dedicated chapter.

\vs

I will first introduce the full unambigous Leibniz notation and
discuss its historical and modern interpretations. I wil then discuss
its various ``vagaries''-- shortcuts that people take in practice.
Finally, I'll do a bunch of practice problems that might show up in
science and engineering to get used to the notation.

\subsection{Historical interpretation}

We start with the historical interpretation, where the notation began.
Leibniz didn't know about limits. He thought the derivative is the
value of the quotient $\frac{f(x+h)-f(x)}{h}$ when $h$ is
``infinitesimally small''. He denoted this infinitesimally small
quantity of $h$ by $dx$, and the corresponding difference
$f(x+dx)-f(x)$ by $df(x)$. Thus for a given function $f$ the Leibniz
notation for its derivative $f'$ is:
\[\frac{df(x)}{dx}=f'\]

Intuitively, we can think of $d$ in a historical context as ``delta''
or ``change''. Then we can interpret this notation as Leibniz did-- a
quotient of a tiny change in $f(x)$ and a tiny change in $x$. There
are two important notes.

\vs

\textit{First}, $d$ is not a value. If it were, you could cancel out
$d$'s in the numerator and the denomenator. But you can't. Instead
think of $d$ as an operator. When applied to $f(x)$ or $x$, it
produces an infinitesimally small quantity. Alternatively you can
think of $df(x)$ and $dx$ as one symbol that happens to look like
multiplication, but isn't.\footnote{I read somewhere that in his
  notebooks Leibniz experimented with extending $d$ with a squiggle on
  top that went over $x$ to indicate that $d$ is not a value, but I
  haven't been able to verify if that's true.}

\vs

\textit{Second}, note that $\frac{df(x)}{dx}$ denotes a function
equivalent to $f'$, \textit{not} a value equivalent to $f'(x)$. To
denote the value of the derivative function at $a$ we use the
following notation:
\[\left. \frac{d f(x)}{dx} \right|_{x=a}=\lim_{h\to0}\frac{f(a+h)-f(a)}{h}=f'(a)\]

To summarize, the \textbf{full and unambiguous Leibniz notation} is as
follows:
\[\frac{df(x)}{dx}=f' \qquad\text{and}\qquad \left. \frac{d f(x)}{dx} \right|_{x=a}=f'(a)\]


\subsection{Modern interpretation}

Complete ordered fields do not have a notion of infinitesimally small
quantities. Thus in a modern interpretation we treat
$\frac{df(x)}{dx}$ as a symbol denoting $f'$, \textit{not} as a
quotient of numbers. Nothing here is being divided, nothing can be
canceled out. In a modern interpretation $\frac{df(x)}{dx}$ is just
one thing that \textit{happens to look} like a quotient but isn't,
anymore than $f'$ is a quotient.

\vs



\vs

Leibniz notation for the second derivative is
\[\frac{d\left(\frac{df(x)}{dx}\right)}{dx},\ \ \ \text{abbreviated
    to}\ \ \ \frac{d^2f(x)}{(dx)^2}, \ \ \ \text{or more often to}\ \
  \ \frac{d^2f(x)}{dx^2}.\]

\subsection{Shortcuts and vagaries}

There are two notable ambiguities associated with the Leibniz
notation. First, $\frac{df(x)}{dx}$ is frequently abbreviated to
$\frac{df}{dx}$. Second, $\frac{df(x)}{dx}$ sometimes means the
function $f'$, and sometimes means the value $f'(x)$. The meaning of
the symbol often must be deteremined from the specific context.

\subsection{Chain rule}

In the next chapter we will learn that $(f\circ g)'(x)=f'(g(x))\cdot g'(x)$.
In Leibniz notation it ought to look like this:
\[\frac{d f(g(x))}{dx} = \left. \frac{d f(y)}{dy} \right|_{y=g(x)} \cdot
  \frac{d g(x)}{dx}\]

But nobody does it this way. Usually people state that if $y=g(x)$ and
$z=f(y)$ then:
\[\frac{dz}{dx}=\frac{dz}{dy}\cdot \frac{dy}{dx}\]

\subsection{Implicit differentiation}

\subsection{Notation practice}


%%% Local Variables:
%%% TeX-master: "notes"
%%% End:
