
\section{Derivatives, Part III (Consequences)}

\subsection{Maxima and manima, part I}
We start with some definitions. Let $f$ be a function and $A$ a set of
numbers contained in $f$'s domain.\footnote{$A$ need not have any
  additional properties. E.g. it may have holes, etc.} Then:

\vs

\textbf{Definition.} A point $x$ in $A$ is a \textbf{maximum
  point} for $f$ on $A$ if
\[f(x)\geq f(y)\qquad\text{for every $y$ in $A$}\]

\textbf{Definition.} A point $x$ in $A$ is a \textbf{local maximum
  point} for $f$ on $A$ if there is some $\delta>0$ such that $x$ is a
maximum point for $f$ on $A\cap(x-\delta, x+\delta)$.

\vs

\textbf{Definition.} A \textbf{critical point} of $f$ is a number $x$
such that $f'(x)=0$.\footnote{If $x$ is a maximum and/or critical
  point, then $f(x)$ is called a maximum and/or critical value of
  $f$.}

\vs---\vs

\textbf{Theorem 1.} Let $f$ be any function defined on $(a,b)$. If $x$
is a maximum (or minimum) point for $f$ on $(a,b)$, and $f$ is
differentiable at $x$, then $f'(x)=0$.

\vs

\textit{Intuitively,} maximum and minimum points are also critical
points (but \textbf{not} the other way around-- $f(x)=x^3$ has
$f'(0)=0$ as an obvious counterexample\footnote{In this case this
  critical point is called the \textbf{saddle point}.}).

\vs

\textbf{Proof.}

\subsection{Mean value theorem}
\subsection{Increasing and decreasing functions}
\subsection{Graph sketching}
\subsection{Maxima and manima, part II}
\subsection{L'H\^opital's rule}
\subsection{Convexity and concavity}

%%% Local Variables:
%%% TeX-master: "notes"
%%% End:
