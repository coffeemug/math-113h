
\section{Derivatives, Part III (Consequences)}

\subsection{Maxima and manima, part I}
We start with some definitions. Let $f$ be a function and $A$ a set of
numbers contained in $f$'s domain.\footnote{$A$ need not have any
  additional properties. E.g. it may have holes, etc.} Then:

\vs

\textbf{Definition.} A point $x$ in $A$ is a \textbf{maximum
  point} for $f$ on $A$ if
\[f(x)\geq f(y)\qquad\text{for every $y$ in $A$}\]

\textbf{Definition.} A point $x$ in $A$ is a \textbf{local maximum
  point} for $f$ on $A$ if there is some $\delta>0$ such that $x$ is a
maximum point for $f$ on $A\cap(x-\delta, x+\delta)$.

\vs

\textbf{Definition.} A \textbf{critical point} of $f$ is a number $x$
such that $f'(x)=0$.\footnote{If $x$ is a maximum and/or critical
  point, then $f(x)$ is called a maximum and/or critical value of
  $f$.}

\vs---\vs

\textbf{Theorem 1a.} Let $f$ be any function defined on $(a,b)$. If $x$
is a maximum point for $f$ on $(a,b)$, and $f$ is differentiable at
$x$, then $f'(x)=0$.

\vs

\textit{Intuitively,} maximum and minimum points are also critical
points (but \textbf{not} the other way around-- $f(x)=x^3$ has
$f'(0)=0$ as an obvious counterexample\footnote{In this case this
  critical point is called the \textbf{saddle point}.}).

\vs

\textbf{Proof.} \textit{Informally}, suppose $a$ is a maximum point.
Draw a secant line between $a$ and $a_l$ (to its left), and another
line between $a$ and $a_r$ (to its right). The $a-a_l$ line will slope
up, the $a-a_r$ line will slope down. Thus at $a$ the slope crosses
from positive to negative, and is $0$.

\vs

\textit{Formally}, let $h\in\mathcal{R}$ such that $x+h\in(a,b)$. If $h<0$ it follows that:
\begin{align*}
  &f(x+h)\leq f(x)&\text{since $f(x)$ is a maximum value}\\
  &\implies f(x+h)-f(x)\leq 0\\
  &\implies\frac{f(x+h)-f(x)}{h}\geq0&\text{dividing by negative $h$}\\
  &\implies \lim_{h\to0^-}\frac{f(x+h)-f(x)}{h}\geq0&\text{see
                                                 \ref{subsec:onesided-limits}
                                                 and
                                                 \ref{subsubsec:nonzero-lemma}}
\end{align*}

Conversely, if $h>0$ it follows that:
\begin{align*}
  &\implies \frac{f(x+h)-f(x)}{h}\leq0&\text{dividing by positive $h$}\\
  &\implies \lim_{h\to0^+}\frac{f(x+h)-f(x)}{h}\leq0
\end{align*}

By hypothesis, $f$ is differentiable at $x$. Thus the two limits must
be equal to each other, and to $f'(x)$. Therefore $f'(x)\geq0$ and
$f'(x)\leq0$. Thus $f'(x)=0$ as desired.

\vs

\textbf{Theorem 1b.} Let $f$ be any function defined on $(a,b)$. If
$x$ is a \textit{minimum} point for $f$ on $(a,b)$, and $f$ is
differentiable at $x$, then $f'(x)=0$.

\vs

\textbf{Proof.} Let $g=-f$. Then $x$ is a maximum point of $g$. By 1a,
$g'(x)=0$, thus $(-f)'(x)=-1\cdot f'(x)=0$, and thus $f'(x)=0$ as desired.

\vs

\textbf{Theorem 2.} If $f$ is defined on $(a,b)$ and has a local
maximum (or minimum) at $x$, and $f$ is differentiable at $x$, then
$f'(x)=0$.

\vs

\textbf{Proof.} The proof is a trivial application of theorem 1 to $f$
on $(x-\delta, x+\delta)$.

\vs---\vs

The obvious (extremely valuable) consequences of these theorems is
that we can find minimum and maximum values of $f$ by solving for
$f'(x)=0$.

\subsection{Mean value theorem}

\textbf{Theorem 3 (Rolle's theorem).} Let $f$ be continuous on $[a,b]$
differentiable on $(a,b)$, and let $f(a)=f(b)$. Then there exists
$x\in(a,b)$ such that $f'(x)=0$.

\vs

\textbf{Proof.} There are two cases:
\begin{itemize}
\item \textit{Case 1.} Suppose the maximum or the minimum occurs at a
  point $x\in(a,b)$. Then $f'(x)=0$ by theorem 1, and we are done.
\item \textit{Case 2.} Suppose the maximum and the minimum both occur
  at endpoints. Since $f(a)=f(b)$, the maximum and the minimum values
  are equal and $f$ is constant. Then for any $x\in(a,b), f'(x)=0$ and
  we are done.
\end{itemize}

\textbf{Theorem 4 (Mean value theorem).}

TODO

\subsection{Increasing and decreasing functions}
\subsection{Graph sketching}
\subsection{Maxima and manima, part II}
\subsection{L'H\^opital's rule}
\subsection{Convexity and concavity}

%%% Local Variables:
%%% TeX-master: "notes"
%%% End:
