\section{Continuity}

\subsection{Definition of continuity}

A function $f$ is \textbf{continuous} at $a$ when

\[\lim_{x\to a}f(x)=f(a)\]

Inlining the limits definition, $f$ is continuous at $a$ if for all
$\epsilon>0$ there exists $\delta>0$ such that $0<|x-a|<\delta$ implies
$|f(x)-f(a)|<\epsilon$.

\vs

We can simplify this definition slightly. Observe that in continuous
functions $f(a)$ exists, and at $x=a$ we get $f(x)-f(a)=0$. Thus we
can relax the constraint $0<|x-a|<\delta$ to $|x-a|<\delta$.

\vs

A function $f$ is \textbf{continuous on an interval} $(a, b)$ if it's
continuous at all $c\in(a,b)$\footnote{Closed intervals are a tiny bit
  harder, and I'm keeping them out for brevity.}.

\subsection{Recognizing continuous functions}
The following theorems allow us to tell at a glance that large classes
of functions are continuous (e.g. polynomials, rational functions,
etc.)

\subsubsection*{Five easy proofs}

\paragraph{Constants.} Let $f(x)=c$. Then $f$ is continuous at all $a$
because
\[\lim_{x\to a}f(x)=c=f(a)\]

\paragraph{Identity.} Let $f(x)=x$. Then $f$ is continuous at all $a$
because
\[\lim_{x\to a}f(x)=a=f(a)\]

\paragraph{Addition.} Let $f,g\in\R\to\R$ be continuous at $a$. Then
$f+g$ is continuous at $a$ because
\[\lim_{x\to a}(f+g)(x)=\lim_{x\to a}f(x)+\lim_{x\to a}g(x)=f(a)+g(a)=(f+g)(a)\]

\paragraph{Multiplication.} Let $f,g\in\R\to\R$ be continuous at $a$. Then
$f\cdot g$ is continuous at $a$ because
\[\lim_{x\to a}(fg)(x)=\lim_{x\to a}f(x)\cdot\lim_{x\to a}g(x)=f(a)\cdot g(a)=(fg)(a)\]

\paragraph{Reciprocal.} Let $g$ be continuous at $a$. Then $\frac{1}{g}$
is continuous at $a$ where $g(a)\neq 0$ because
\[\lim_{x\to a}\left(\frac{1}{g}\right)(x)=\frac{1}{\lim_{x\to a}g(x)}=\frac{1}{g(a)}=\left(\frac{1}{g}\right)(a)\]

\subsubsection*{Slightly harder proof: composition}

Let $f,g\in\R\to\R$. Let $g$ be continuous at $a$, and let $f$ be
continuous at $g(a)$. Then $f\circ g$ is continuous at $a$. Put
differently, we want to show
\[\lim_{x\to a}(f\circ g)(x)=(f\circ g)(a)\]

Unpacking the definitions, let $\epsilon>0$ be given. We want to show there
exists $\delta>0$ such that $|x-a|<\delta$ implies
\begin{align*}
    |(f\circ g)&(x)-(f\circ g)(a)|\\
    &=|f(g(x))-f(g(a))|<\epsilon
\end{align*}

By problem statement we have two continuities.

\vs

\textbf{First}, $f$ is continuous at $g(a)$, i.e.
$\lim_{X\to g(a)}f(X)=f(g(a))$. Thus there exists $\delta'>0$ such that
$|X-g(a)|<\delta'$ implies $|f(X)-f(g(a))|<\epsilon$.

\vs

\textbf{Second}, $g$ is continuous at $a$, i.e.
$\lim_{x\to a}g(x)=g(a)$. Thus there exists $\delta>0$ such that
$|x-a|<\delta$ implies $|g(x)-g(a)|<\epsilon$. Since we can make
$\epsilon$ be anything, we can set it to $\delta'$.

\vs

I.e. there exists $\delta>0$ such that $|x-a|<\delta$ implies
$|g(x)-g(a)|<\delta'$. Intuitively, $g(x)$ is close to $g(a)$. But by the
first continuity, $X$ close to $g(a)$ implies
\[|f(X)-f(g(a))|<\epsilon\]

Thus $|f(g(x))-f(g(a))|<\epsilon$, as desired.

\subsection{Appendix: example functions}

%%% Local Variables:
%%% TeX-master: "index"
%%% End:
