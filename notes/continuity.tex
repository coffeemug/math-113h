\section{Continuity, Part I}

\subsection{Definition of continuity}

A function $f$ is \textbf{continuous} at $a$ when

\[\lim_{x\to a}f(x)=f(a)\]

Inlining the limits definition, $f$ is continuous at $a$ if for all
$\epsilon>0$ there exists $\delta>0$ such that $0<|x-a|<\delta$ implies
$|f(x)-f(a)|<\epsilon$.

\vs

We can simplify this definition slightly. Observe that in continuous
functions $f(a)$ exists, and at $x=a$ we get $f(x)-f(a)=0$. Thus we
can relax the constraint $0<|x-a|<\delta$ to $|x-a|<\delta$.

\vs

A function $f$ is \textbf{continuous on an interval} $(a, b)$ if it's
continuous at all $c\in(a,b)$\footnote{Closed intervals are a tiny bit
  harder, and I'm keeping them out for brevity.}.

\subsubsection*{Nonzero Neighborhood Lemma} \label{subsubsec:nonzero-lemma}

Armed with these definitions we can extend the half-value neighborhood
lemma (see \ref{subsubsec:half-value-lemma}) in a useful way. The
\textit{nonzero neighborhood lemma} will come in handy when we prove
the intermediate value theorem (see \ref{ivt}), so we may as well
prove the lemma now.

\vs

Suppose $f$ is continuous at $a$, and $f(a)\neq0$. Then there exists
$\delta>0$ such that:
\begin{enumerate}
\item if $f(a)<0$ then $f(x)<0$ for all $x$ in $|x-a|<\delta$.
\item if $f(a)>0$ then $f(x)>0$ for all $x$ in $|x-a|<\delta$.
\end{enumerate}

\textit{Intuitively} the lemma states that there is some interval
around $a$ on which $f(x)\neq0$ and has the same sign as $f(a)$.

\vs

\textbf{Proof.} The proof follows trivially from the half-value
neighborhood lemma.


\subsection{Recognizing continuous functions}
The following theorems allow us to tell at a glance that large classes
of functions are continuous (e.g. polynomials, rational functions,
etc.)

\subsubsection*{Five easy proofs}

\paragraph{Constants.} Let $f(x)=c$. Then $f$ is continuous at all $a$
because
\[\lim_{x\to a}f(x)=c=f(a)\]

\paragraph{Identity.} Let $f(x)=x$. Then $f$ is continuous at all $a$
because
\[\lim_{x\to a}f(x)=a=f(a)\]

\paragraph{Addition.} Let $f,g\in\R\to\R$ be continuous at $a$. Then
$f+g$ is continuous at $a$ because
\[\lim_{x\to a}(f+g)(x)=\lim_{x\to a}f(x)+\lim_{x\to a}g(x)=f(a)+g(a)=(f+g)(a)\]

\paragraph{Multiplication.} Let $f,g\in\R\to\R$ be continuous at $a$. Then
$f\cdot g$ is continuous at $a$ because
\[\lim_{x\to a}(fg)(x)=\lim_{x\to a}f(x)\cdot\lim_{x\to a}g(x)=f(a)\cdot g(a)=(fg)(a)\]

\paragraph{Reciprocal.} Let $g$ be continuous at $a$. Then $\frac{1}{g}$
is continuous at $a$ where $g(a)\neq 0$ because
\[\lim_{x\to a}\left(\frac{1}{g}\right)(x)=\frac{1}{\lim_{x\to a}g(x)}=\frac{1}{g(a)}=\left(\frac{1}{g}\right)(a)\]

\subsubsection*{Slightly harder proof: composition}

Let $f,g\in\R\to\R$. Let $g$ be continuous at $a$, and let $f$ be
continuous at $g(a)$. Then $f\circ g$ is continuous at $a$. Put
differently, we want to show
\[\lim_{x\to a}(f\circ g)(x)=(f\circ g)(a)\]

Unpacking the definitions, let $\epsilon>0$ be given. We want to show there
exists $\delta>0$ such that $|x-a|<\delta$ implies
\begin{align*}
    |(f\circ g)&(x)-(f\circ g)(a)|\\
    &=|f(g(x))-f(g(a))|<\epsilon
\end{align*}

By problem statement we have two continuities.

\vs

\textbf{First}, $f$ is continuous at $g(a)$, i.e.
$\lim_{X\to g(a)}f(X)=f(g(a))$. Thus there exists $\delta'>0$ such that
$|X-g(a)|<\delta'$ implies $|f(X)-f(g(a))|<\epsilon$.

\vs

\textbf{Second}, $g$ is continuous at $a$, i.e.
$\lim_{x\to a}g(x)=g(a)$. Thus there exists $\delta>0$ such that
$|x-a|<\delta$ implies $|g(x)-g(a)|<\epsilon$. Since we can make
$\epsilon$ be anything, we can set it to $\delta'$.

\vs

I.e. there exists $\delta>0$ such that $|x-a|<\delta$ implies
$|g(x)-g(a)|<\delta'$. Intuitively, $g(x)$ is close to $g(a)$. But by the
first continuity, any $X$ close to $g(a)$ implies
\[|f(X)-f(g(a))|<\epsilon\]

Thus $|f(g(x))-f(g(a))|<\epsilon$, as desired.

\subsection{Example: Stars over Babylon}
Stars over Babylon is a modification of the Dirichlet function (see
\ref{subsubsec:dirichlet}), defined as follows:
\[
f(x) = 
\begin{cases} 
  0, & \text{$x$ irrational}, 0<x<1\\
  1/q, & x=p/q \text{ in lowest terms}, 0<x<1.
\end{cases}
\]

\textbf{Claim:} for $0<a<1$, $\lim_{x\to a}f(x)=0$.

\vs

\textbf{Proof.} Let $\epsilon>0$ be given. We must find $\delta>0$ such that
$0<|x-a|<\delta$ implies $|f(x)-0|<\epsilon$. For \textit{any}
$\delta>0$, $0<|x-a|<\delta$ implies one of two cases for all $x$: either
$x$ is irrational or it is rational.

\vs

If $x$ is irrational, $|f(x)-0|=0<\epsilon$.

\vs

Otherwise, if $x=p/q$ in the lowest terms is rational, $f(x)=1/q$. Let
$n\in\mathcal{N}$ such that $1/n<\epsilon$. We will look for $\delta$ such that:
\[f\left(\frac{p}{q}\right)=\frac{1}{q}<\frac{1}{n}<\epsilon\]

\vs

Observe that when $q>n$, $f(\frac{p}{q})=\frac{1}{q}<\frac{1}{n}$.
Thus the only rationals that \textit{could} result in $f(\frac{p}{q})\geq1/n$ are
ones where $q\leq n$:
\[A=\{\frac{1}{2};\ \ \frac{1}{3},\frac{2}{3};\ \
  \frac{1}{4},\frac{3}{4};\ \
  \frac{1}{5},\frac{2}{5},\frac{3}{5},\frac{4}{5},\ \ \ldots,\ \ \frac{1}{n},\ldots,\frac{n-1}{n}\}\]

This set has a finite length, and thus \textit{one} $p/q\in A$ is
closest to $a$. Fix $\delta=|a-p/q|$ (i.e. anything less than this
distance). This guarantees $0<|x-a|<\delta$ implies $x\notin A$ for all $x$, and
thus $f(x)<1/n<\epsilon$ for all $x$, as desired.

\vs

\textbf{Claim:} $f(x)$ is continuous at all irrationals, discontinuous
at all rationals.

\vs

\textbf{Proof:} we've just proven for $0<a<1$, $\lim_{x\to a}f(x)=0$. By
definition $f(x)$ is zero for all irrationals, and nonzero for all
rationals. Thus $\lim_{x\to a}f(x)=f(x)$ for all irrationals, and
$\lim_{x\to a}f(x)\neq f(x)$ for all rationals.

%%% Local Variables:
%%% TeX-master: "notes"
%%% End:
