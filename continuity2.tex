\section{Continuity, Part II}

\subsection{Intermediate Value Theorem} \label{ivt}
\subsection{Extreme Value Theorem}
\subsection{Appendix: IVT and EVT consequences}
\subsection{Problems}
\subsection*{Problem 3a (from chapter 8)}
Let $f$ be a continous function on $[a, b]$ with $f(a)<0<f(b)$. The
proof of Theorem 1 showed that there is a smallest $x$ in $[a,b]$ with
$f(x)=0$. Is there necessarily a second smallest $x$ in $[a,b]$ with
$f(x)=0$? Show that there is a largest $x$ in $[a,b]$ with $f(x)=0$.
(Try to give an easy proof by considering a new function closely
related to $f$.)

\subsubsection*{Solution}



%%% Local Variables:
%%% TeX-master: "notes"
%%% End:
